\begin{Interfaz}
  
  \textbf{se explica con}: \tadNombre{Red}.

  
  \section*{Operaciones básicas de \tadNombre{Red}}

  \InterfazFuncion{IniciarRed}{}{red}
  {$res \igobs iniciarRed()$}
  [$\Theta()$]
  [genera la red vacía]

  \InterfazFuncion{AgregarComputadora}{\Inout{r}{\tadNombre{Red}}, \In{c}{\tadNombre{Compu}}}{}
  [$r \igobs r_0 \land (\forall c' : compu) (c' \in computadoras(r) \impluego ip(c) \neq ip(c'))$] 
  {$computadoras(r) = Ag(c, computadoras(r_0)) \yluego \\
     (\forall c_1, c_2 : compu) (c_1 \in computadoras(r_0) \land c_2 \in computadoras(r_0) \impluego \\ 
   conectadas(r_0, c_1, c_2) = conectadas(r, c_1, c_2) \land interfazUsada(r_0, c_1, c_2) = interfazUsada(r, c_1, c_2)$)}
  [$\Theta()$]
  [agrega el la computadora $c$ a la red $r$.]
  [el elemento $c$ agrega por copia.]
  
  
  \InterfazFuncion{conectar}{\Inout{r}{\tadNombre{Red}}, \In{c_1}{\tadNombre{Compu}},  \In{i_1}{\tadNombre{Interfaz}}, \In{c_2}{\tadNombre{Interfaz}}, \In{i_2}{\tadNombre{Interfaz}}}{}
 [$r \igobs r_0 \land c_1 \in computadoras(r) \land c_2 \in computadoras(r) \land ip(c_1) \neq ip(c_2) \land \neg conectadas?(r, c_1, c_2) \land \neg usaInterfaz?(r,c_1, i_1) \land \neg usaInterfaz?(r, c_2, i_2)$]
  {$computadoras(r) = computadoras(r_0) \yluego \\
    (\forall c'_1, c'_2 : compu) (c'_1 \in computadoras(r_0) \land c'_2 \in computadoras(r_0) \land \{c_1, c_2\} \neq \{c'_1, c'_2\} \impluego \\ 
   conectadas(r_0, c'_1, c'_2) = conectadas(r, c'_1, c'_2) \land interfazUsada(r_0, c'_1, c'_2) = interfazUsada(r, c'_1, c'_2)) \land \\
   conectadas(r_0, c_1, c_2) \land interfazUsada(r, c_1, c_2) = i_1 \land interfazUsada(r, c_2, c_1) = i_2$}
  [$\Theta()$]
  [conecta las computadoras $c_1$ y $c_2$ mediante las interfaces $i_1$ e $i_2$ respectivamente en la red $r$.]


  \InterfazFuncion{computadoras}{\In{r}{\tadNombre{Red}}}{}
  {$res = computadoras(r)$}
  [$\Theta()$]
  [devuelve un conjunto con todas las computadoras de la red $r$.]
  [?]


  \InterfazFuncion{conectadas?}{\In{r}{\tadNombre{Red}}, \In{c_1}{\tadNombre{Compu}}, \In{c_2}{\tadNombre{Compu}}}{\tadNombre{bool}}
  [$c_1 \in computadoras(r) \land c_2 \in computadoras(r)$]
  {$res = conectadas?(r,c_1,c_2)$}
  [$\Theta()$]
  [dice si las computadoras $c_1$ y $c_2$ están conectadas en la red $r$]


  \InterfazFuncion{interfazUsada}{\In{r}{\tadNombre{Red}}, \In{c_1}{\tadNombre{Compu}}, \In{c_2}{\tadNombre{Compu}}}{\tadNombre{bool}}
  [$conectadas?(r,c_1,c_2)$]
  {$res = interfazUsada(r,c_1,c_2)$}
  [$\Theta()$]
  [devuelve la interfaz mediante la cual $c_1$ se conecta a $c_2$  en la red $r$.]


  \InterfazFuncion{vecinos}{\In{r}{\tadNombre{Red}}, \In{c}{\tadNombre{Compu}}}{\tadNombre{conj(compu)}}
  [$c \in computadoras(r)$]
  {$res = vecinos(r,c)$}
  [$\Theta()$]
  [devuelve las computadoras conectadas a $c$ en la red $r$.]
  [?]


  \InterfazFuncion{usaInterfaz?}{\In{r}{\tadNombre{Red}}, \In{c}{\tadNombre{Compu}}, \In{i}{\tadNombre{Interfaz}}}{\tadNombre{bool}}
  [$c \in computadoras(r)$]
  {$res = usaInterfaz?(r,c,i)$}
  [$\Theta()$]
  [devuelve true si y solo si $c$ está conectada a otra pc mediante la interfaz $i$ en la red $r$.]


  \InterfazFuncion{caminosMinimos}{\In{r}{\tadNombre{Red}}, \In{c_1}{\tadNombre{Compu}}, \In{c_2}{\tadNombre{Compu}}}{\tadNombre{conj(secu(compu))}}
  [$c_1 \in computadoras(r) \land c_2 \in computadoras(r)$]
  {$res = caminosMinimos(r,c_1,c_2)$}
  [$\Theta()$]
  [devuelve un conjunto con los caminos más cortos entre las computadoras $c_1$ y $c_2$ en la red $r$.]

%%

  \InterfazFuncion{hayCamino?}{\In{r}{\tadNombre{Red}}, \In{c_1}{\tadNombre{Compu}}, \In{c_2}{\tadNombre{Compu}}}{\tadNombre{bool}}
  [$c_1 \in computadoras(r) \land c_2 \in computadoras(r)$]
  {$res = hayCamino?(r,c_1,c_2)$}
  [$\Theta()$]
  [devuelve true si y solo si hay un camino entre las computadoras $c_1$ y $c_2$ en la red $r$.]



\end{Interfaz}

\begin{Representacion}
  
  
  
  \section*{Representación de la red}

  \begin{Estructura}{red}[estr]
    \begin{Tupla}[estr]
      \tupItem{vecinos}{dicc(id, dicc(id, interfaz))}\\
      \tupItem{compus}{dicc(id, compu)}%
    \end{Tupla}

  \end{Estructura}

  \Rep[estr][e]{claves(e.vecinos) $=$ claves(e.compus) $\yluego$ \\
  ($\forall i_1, i_2$ : id) def?($i_2$, (obtener($i_1$, e.vecinos))) $\impluego$ def?($i_2$, e.vecinos) $\yluego$ def?($i_1$, obtener($i_2$, e.vecinos)) $\yluego$ \\
obtener($i_2$, obtener($i_1$, e.vecinos)) $\in$  interfaces(obtener($i_1$, e.compus))
$\land$\\ 
($\forall i, i_1, i_2$ : id) def?($i$, e.vecinos) $\impluego$ \\
(def?($i_1$, obtener($i$, e.vecinos)) $\land$ def?($i_2$, obtener($i$, e.vecinos)) $\land$ $i_1 \neq i_2$ $\impluego$ obtener($i_1$, obtener($i$, e.vecinos)) $\neq$ obtener($i_2$, obtener($i$, e.vecinos)))

}\mbox{}
  
   
   \AbsFc[estr]{red}[e]{
     r : red / computadoras(r) = significados(e.compus) $\land$ \\ 
     ($\forall c_1, c_2$ : computadora) conectadas($r, c_1, c_2$) = def?(id($c_1$), obtener(id($c_2$), e.vecinos)) $\land$ \\
     ($\forall c_1, c_2$: computadora) ($\forall i$ : interfaz) conectadas($r,c_1,c_2$) $\impluego$ interfazUsada($r, c_1, c_2$) = obtener(id($c_1$), obtener(id($c_2$), e.vecinos))
  }

\end{Representacion}



\begin{Algoritmos}


\begin{algorithm}
  \caption{Algoritmos de \tadNombre{RED}}
\begin{algorithmic}[1]
  \Procedure{IniciarRed() $\to$ $res$ : red}{}
  \State vecinos $\gets$ Vacia()
  \State interfaces $\gets$ Vacia()
  \State $res \gets$ (vecinos, interfaces)
  \EndProcedure
\end{algorithmic}
\end{algorithm}


\begin{algorithm}
\caption{Agregar Computadora}
\begin{algorithmic}[1]
  \Procedure{AgregarComputadora(\textbf{in/out} $r$ : red, \textbf{in} $c$ : compu)}{}
  \State definir(r.vecinos, id($c$), Vacia())
  \State definir(r.interfaces, id($c$), $c$)
 \EndProcedure
\end{algorithmic}
\end{algorithm}



\begin{algorithm}
%\caption{Conectar}
\begin{algorithmic}[1]
  \Procedure{Conectar(\textbf{in/out} $r$ : red, \textbf{in} $c_1$ : compu, \textbf{in} $c_2$ : compu, \textbf{in} $i_1$ : interfaz, \textbf{in} $i_2$ : interfaz)}{}
  \State $x \gets$ obtener(r.vecinos, $c_1$)
  \State definir($x$, id($c_2$), $i_1$)
  \State $x \gets$ obtener(r.vecinos, $c_2$)
  \State definir($x$, id($c_1$), $i_2$)
 \EndProcedure
\end{algorithmic}




%\caption{Computadoras}
\begin{algorithmic}[1]
  \Procedure{Computadoras(\textbf{in} $r$ : red) $\to res$ : conj(compu)}{} 
  \State $i :\tadNombre{itdicc} \gets$ NuevoIterador($r$.vecinos)
  \State $res :\tadNombre{conj(compu)} \gets$ Vacio()
  \While{haySiguiente($i$)}
    \State Agregar($res$, Obtener(Siguiente($i$), $r$.vecinos))
    \State Avanzar($i$)
  \EndWhile
 \EndProcedure
\end{algorithmic}


%\caption{Conectadas?}
\begin{algorithmic}[1]
  \Procedure{Conectadas?(\textbf{in} $r$ : red, \textbf{in} $c_1$ : compu, \textbf{in} $c_2$ : compu)  $\to res$ : bool}{}
  \State $x \gets$ Obtener(id($c_1$), $r$.vecinos)
  \State $res \gets$ def?(id($c_2$), $x$)
 \EndProcedure
\end{algorithmic}


%\caption{InterfazUsada}
\begin{algorithmic}[1]
  \Procedure{InterfazUsada(\textbf{in} $r$ : red, \textbf{in} $c_1$ : compu, \textbf{in} $c_2$ : compu)  $\to res$ : interfaz}{}
  \State $x \gets$ Obtener(id($c_1$), $r$.vecinos)
  \State $res \gets$ Obtener(id($c_2$), $x$)
 \EndProcedure
\end{algorithmic}


%\caption{Vecinos}
\begin{algorithmic}[1]
  \Procedure{Vecinos(\textbf{in} $r$ : red, \textbf{in} $c$ : compu)  $\to res$ : conj(compu)}{}
  \State $x \gets$ Obtener(id($c$), $r$.vecinos)
  \State $res \gets$ Claves($x$)
 \EndProcedure
\end{algorithmic}


%\caption{UsaInterfaz?}
\begin{algorithmic}[1]
  \Procedure{UsaInterfaz?(\textbf{in} $r$ : red, \textbf{in} $c$ : compu,  \textbf{in} $i$ : interfaz) $\to res$ : bool}{} 
  \State $x \gets$ Obtener(id($c$), $r$.vecinos)
  \State $i :\tadNombre{itdicc} \gets$ NuevoIterador($x$)
  \State $res \gets$ False
  \While{haySiguiente($i$)}
    \If {Obtener(Siguiente($i$), $x$) = $i$} $res \gets$ True  \EndIf
    \State Avanzar($i$)
  \EndWhile
 \EndProcedure
\end{algorithmic}



%\caption{CaminosMinimos}
\begin{algorithmic}[1]
  \Procedure{CaminosMinimos(\textbf{in} $r$ : red, \textbf{in} $c_1$ : compu,  \textbf{in} $c_2$ : compu) $\to res$ : conj(secu(compu))}{} 
   \State En construcción 
\EndProcedure
\end{algorithmic}


%\caption{HayCamino?}
\begin{algorithmic}[1]
  \Procedure{HayCamino?(\textbf{in} $r$ : red, \textbf{in} $c_1$ : compu,  \textbf{in} $c_2$ : compu) $\to res$ : bool}{} 
   \State $res \gets$ $\neg$ esVacio?(CaminosMinimos($r$,$c_1$, $c_2$))
\EndProcedure
\end{algorithmic}


\end{algorithm}



\end{Algoritmos}







