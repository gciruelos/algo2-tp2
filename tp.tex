\documentclass[a4paper,10pt, nofootinbib]{article}
\usepackage[width=15.5cm, left=3cm, top=2.5cm, right=1cm, left=2cm, height= 24.5cm]{geometry}
\usepackage[spanish]{babel}
\usepackage[utf8]{inputenc}
\usepackage[T1]{fontenc}
\usepackage{xspace}
\usepackage{xargs}
\usepackage{ifthen}
\usepackage{caratula}
\usepackage{fancyhdr}
\usepackage{aed2-tad,aed2-symb,aed2-itef}
\usepackage[bottom]{footmisc}
\usepackage{modulos}
\usepackage{algorithm}
\usepackage[noend]{algpseudocode}

\newcommand{\moduloNombre}[1]{\textbf{#1}}


\let\NombreFuncion=\textsc
\let\TipoVariable=\texttt
\let\ModificadorArgumento=\textbf
\newcommand{\res}{$res$\xspace}
\newcommand{\tab}{\hspace*{7mm}}

\newcommand{\Ogr}{\mathcal{O}}


\newcommand{\footnoteMio}[2]{\textsuperscript{#1} \lfoot{\footnotesize \parbox{16cm}{\textsuperscript{#1}#2}}}

\newcommandx{\TipoFuncion}[3]{%
  \NombreFuncion{#1}(#2) \ifx#3\empty\else $\to$ \res\,: \TipoVariable{#3}\fi%
}
\newcommand{\In}[2]{\ModificadorArgumento{in} \ensuremath{#1}\,: \TipoVariable{#2}\xspace}
\newcommand{\Out}[2]{\ModificadorArgumento{out} \ensuremath{#1}\,: \TipoVariable{#2}\xspace}
\newcommand{\Inout}[2]{\ModificadorArgumento{in/out} \ensuremath{#1}\,: \TipoVariable{#2}\xspace}
\newcommand{\Aplicar}[2]{\NombreFuncion{#1}(#2)}

\newcommand{\Titulo}[1]{
  \vspace*{1ex}\par\noindent\textbf{\large #1}\par
}
\newcommand{\DRef}{\ensuremath{\rightarrow}}

%%Información para la carátula
\materia{Algoritmos y Estructuras de Datos II}

\titulo{\LARGE Trabajo Práctico Nº2}

\grupo{Grupo 18}

\integrante{Ciruelos Rodríguez, Gonzalo}{063/14}{gonzalo.ciruelos@gmail.com}
\integrante{Costa, Manuel José Joaquín}{035/14}{manuc94@hotmail.com}
\integrante{Gatti, Mathias Nicolás}{477/14}{mathigatti@gmail.com}
\integrante{Ginsberg, Mario Ezequiel}{145/14}{ezequielginsberg@gmail.com}

\def\cuatrimestre{1}

%%fancyhdr
\pagestyle{fancy}
\thispagestyle{fancy}
\addtolength{\headheight}{1pt}
\lhead{Algoritmos y Estructuras de Datos II: TP1}
\rhead{$1^{\mathrm{er}}$ cuatrimestre de 2015}
\cfoot{\thepage\ / 11}
\renewcommand{\footrulewidth}{0.4pt}



\setlength{\parskip}{0.8em}


\begin{document}
\maketitle
\thispagestyle{empty}

\section{Aclaraciones}
\begin{enumerate}
  \item Dado que consultando con algunos JTP nos sugirieron que en las funciones de DCNet en las que devolvemos computadoras, nos manejemos solamente con sus ips (es decir, no devolvamos una tupla \texttt{(ip, conj(interfaz))} sino simplemente una \texttt{ip}). La información de las interfaces obviamente no se 'perdió', sigue estando en la red, sólo que no se está manipulando y devolviendo todo el tiempo.
   
    Además nos dijeron qué cambios deberíamos hacerle al TAD DCNet para que sea consistente con estos cambios. Los cambios que deberían hacerse son simples, solo debería devolverse la ip de las computadoras, por ejemplo, en \texttt{recorridoPaquete}, en la que se devuelve una secuencia de computadoras, debería devolverse una secuencia de ip.
    
    Para hacer la lectura más fácil, decidimos usar intercambiablemente las palabras \texttt{compu} e \texttt{id}, y ambas deben ser intepretadas como el string hostname de la computadora (su ip).
    
    \item Al trabajar con $p$:\texttt{puntero($tupla(campo_1$, $\dots$, $campo_n$))}, usamos la notación $p\to campo_i$ que equivale a decir $(*p).campo_i$. 
    \item Al calcular complejidades de operaciones que tienen condicionales, siempre hacemos la rama que tiene mayor complejidad, pues lo que nos interesa es la complejidad en el peor caso. 
	\item Cuando ingresamos un paquete, éste ya viene con su id, en lugar de generarlo nosotros (como sí pasaba en un ejercicio visto en clase). Esta es una decisión que tomamos para respetar lo más posible el tad. 

\end{enumerate}

\clearpage


\section{Módulo Red}
\begin{Interfaz}
  
  \textbf{se explica con}: \tadNombre{Red}.

  
  \section*{Operaciones básicas de \tadNombre{Red}}

  \InterfazFuncion{IniciarRed}{}{red}
  {$res \igobs iniciarRed()$}
  [$\Theta()$]
  [genera la red vacía]

  \InterfazFuncion{AgregarComputadora}{\Inout{r}{\tadNombre{Red}}, \In{c}{\tadNombre{Compu}}}{}
  [$r \igobs r_0 \land (\forall c' : compu) (c' \in computadoras(r) \impluego ip(c) \neq ip(c'))$] 
  {$computadoras(r) = Ag(c, computadoras(r_0)) \yluego \\
     (\forall c_1, c_2 : compu) (c_1 \in computadoras(r_0) \land c_2 \in computadoras(r_0) \impluego \\ 
   conectadas(r_0, c_1, c_2) = conectadas(r, c_1, c_2) \land interfazUsada(r_0, c_1, c_2) = interfazUsada(r, c_1, c_2)$)}
  [$\Theta()$]
  [agrega el la computadora $c$ a la red $r$.]
  [el elemento $c$ agrega por copia.]
  
  
  \InterfazFuncion{conectar}{\Inout{r}{\tadNombre{Red}}, \In{c_1}{\tadNombre{Compu}},  \In{i_1}{\tadNombre{Interfaz}}, \In{c_2}{\tadNombre{Interfaz}}, \In{i_2}{\tadNombre{Interfaz}}}{}
 [$r \igobs r_0 \land c_1 \in computadoras(r) \land c_2 \in computadoras(r) \land ip(c_1) \neq ip(c_2) \land \neg conectadas?(r, c_1, c_2) \land \neg usaInterfaz?(r,c_1, i_1) \land \neg usaInterfaz?(r, c_2, i_2)$]
  {$computadoras(r) = computadoras(r_0) \yluego \\
    (\forall c'_1, c'_2 : compu) (c'_1 \in computadoras(r_0) \land c'_2 \in computadoras(r_0) \land \{c_1, c_2\} \neq \{c'_1, c'_2\} \impluego \\ 
   conectadas(r_0, c'_1, c'_2) = conectadas(r, c'_1, c'_2) \land interfazUsada(r_0, c'_1, c'_2) = interfazUsada(r, c'_1, c'_2)) \land \\
   conectadas(r_0, c_1, c_2) \land interfazUsada(r, c_1, c_2) = i_1 \land interfazUsada(r, c_2, c_1) = i_2$}
  [$\Theta()$]
  [conecta las computadoras $c_1$ y $c_2$ mediante las interfaces $i_1$ e $i_2$ respectivamente en la red $r$.]


  \InterfazFuncion{computadoras}{\In{r}{\tadNombre{Red}}}{}
  {$res = computadoras(r)$}
  [$\Theta()$]
  [devuelve un conjunto con todas las computadoras de la red $r$.]
  [?]


  \InterfazFuncion{conectadas?}{\In{r}{\tadNombre{Red}}, \In{c_1}{\tadNombre{Compu}}, \In{c_2}{\tadNombre{Compu}}}{\tadNombre{bool}}
  [$c_1 \in computadoras(r) \land c_2 \in computadoras(r)$]
  {$res = conectadas?(r,c_1,c_2)$}
  [$\Theta()$]
  [dice si las computadoras $c_1$ y $c_2$ están conectadas en la red $r$]


  \InterfazFuncion{interfazUsada}{\In{r}{\tadNombre{Red}}, \In{c_1}{\tadNombre{Compu}}, \In{c_2}{\tadNombre{Compu}}}{\tadNombre{bool}}
  [$conectadas?(r,c_1,c_2)$]
  {$res = interfazUsada(r,c_1,c_2)$}
  [$\Theta()$]
  [devuelve la interfaz mediante la cual $c_1$ se conecta a $c_2$  en la red $r$.]


  \InterfazFuncion{vecinos}{\In{r}{\tadNombre{Red}}, \In{c}{\tadNombre{Compu}}}{\tadNombre{conj(compu)}}
  [$c \in computadoras(r)$]
  {$res = vecinos(r,c)$}
  [$\Theta()$]
  [devuelve las computadoras conectadas a $c$ en la red $r$.]
  [?]


  \InterfazFuncion{usaInterfaz?}{\In{r}{\tadNombre{Red}}, \In{c}{\tadNombre{Compu}}, \In{i}{\tadNombre{Interfaz}}}{\tadNombre{bool}}
  [$c \in computadoras(r)$]
  {$res = usaInterfaz?(r,c,i)$}
  [$\Theta()$]
  [devuelve true si y solo si $c$ está conectada a otra pc mediante la interfaz $i$ en la red $r$.]


  \InterfazFuncion{caminosMinimos}{\In{r}{\tadNombre{Red}}, \In{c_1}{\tadNombre{Compu}}, \In{c_2}{\tadNombre{Compu}}}{\tadNombre{conj(secu(compu))}}
  [$c_1 \in computadoras(r) \land c_2 \in computadoras(r)$]
  {$res = caminosMinimos(r,c_1,c_2)$}
  [$\Theta()$]
  [devuelve un conjunto con los caminos más cortos entre las computadoras $c_1$ y $c_2$ en la red $r$.]

%%

  \InterfazFuncion{hayCamino?}{\In{r}{\tadNombre{Red}}, \In{c_1}{\tadNombre{Compu}}, \In{c_2}{\tadNombre{Compu}}}{\tadNombre{bool}}
  [$c_1 \in computadoras(r) \land c_2 \in computadoras(r)$]
  {$res = hayCamino?(r,c_1,c_2)$}
  [$\Theta()$]
  [devuelve true si y solo si hay un camino entre las computadoras $c_1$ y $c_2$ en la red $r$.]



\end{Interfaz}

\begin{Representacion}
  
  
  
  \section*{Representación de la red}

  \begin{Estructura}{red}[estr]
    \begin{Tupla}[estr]
      \tupItem{vecinos}{dicc(id, dicc(id, interfaz))}\\
      \tupItem{interfaces}{dicc(id, conj(interfaz))}%
    \end{Tupla}

  \end{Estructura}

  \Rep[estr][e]{claves(e.vecinos) $=$ claves(e.interfaces) $\yluego$ \\
  ($\forall i_1, i_2$ : id) def?($i_2$, (obtener($i_1$, e.vecinos))) $\impluego$ def?($i_2$, e.vecinos) $\yluego$ def?($i_1$, obtener($i_2$, e.vecinos)) $\yluego$ \\
obtener($i_2$, obtener($i_1$, e.vecinos)) $\in$ obtener($i_1$, e.interfaces)
$\land$\\ 
($\forall i, i_1, i_2$ : id) def?($i$, e.vecinos) $\impluego$ \\
(def?($i_1$, obtener($i$, e.vecinos)) $\land$ def?($i_2$, obtener($i$, e.vecinos)) $\land$ $i_1 \neq i_2$ $\impluego$ obtener($i_1$, obtener($i$, e.vecinos)) $\neq$ obtener($i_2$, obtener($i$, e.vecinos)))

}\mbox{}
  
   
   \AbsFc[estr]{red}[e]{
     r : red / computadoras(r) = claves(e.vecinos) $\land$ \\ 
     ($\forall c_1, c_2$ : computadora) conectadas($r, c_1, c_2$) = def?(id($c_1$), obtener(id($c_2$), e.vecinos)) $\land$ \\
     ($\forall c_1, c_2$: computadora) ($\forall i$ : interfaz) conectadas($r,c_1,c_2$) $\impluego$ interfazUsada($r, c_1, c_2$) = obtener(id($c_1$), obtener(id($c_2$), e.vecinos))
  }

\end{Representacion}








\clearpage

\section{Módulo DiccTrie}
\begin{Interfaz}
  \textbf{parámetros formales}\hangindent=2\parindent\\
  \parbox{1.7cm}{\textbf{géneros}} $\sigma$\\
  \parbox[t]{1.7cm}{\textbf{función}}\parbox[t]{\textwidth-2\parindent-1.7cm}{    	
    \InterfazFuncion{Copiar}{\In{s}{$\sigma$}}{$\sigma$}
    {$res \igobs s$}
    [$\Theta(copy(s))$]
    [función de copia de $\sigma$]
    }
      
  \textbf{se explica con}: \tadNombre{diccionario($string, \sigma$)}.

  \textbf{géneros}: \TipoVariable{diccTrie($\sigma$), itClavesDiccTrie($\sigma$)}.

  \section*{Operaciones básicas DiccTrie($\sigma$)}

  \InterfazFuncion{Vacío}{}{diccTrie($\sigma$)}
  {$res$ \igobs vacío()}
  [$\Ogr(1)$]
  [genera un diccionario vacío.]

  \InterfazFuncion{Definir}{\Inout{d}{diccTrie($\sigma$)}, \In{k}{$string$}, \In{s}{$\sigma$}}{}
  [$d \igobs d_0$]  
  {$d$ \igobs definir($k, s, d_0$)}
  [$\Ogr(|k|)$]
  [define la clave $k$ con el sinificado $s$.]
  [$s$ se define por referencia.]
    
  \InterfazFuncion{Definido?}{\In{d}{diccTrie($\sigma$)}, \In{k}{$string$}}{bool}
  {$res$ \igobs def?($k$, $d$) }
  [$\Ogr(|k|)$]
  [devuelve $true$ si la clave $k$ está definida en el diccionario.]
  
  \InterfazFuncion{Obtener}{\In{d}{diccTrie($\sigma$)}, \In{k}{$string$}}{$\sigma$}
  [def?($k$, $d$)]  
  {alias($res$ \igobs obtener($k$, $d$))}
  [$\Ogr(|k|)$,]
  [devuelve el significado de la clave $k$ en $d$.]
  [$res$ es modificable si y sólo si $d$ lo es]
  
  \InterfazFuncion{Borrar}{\In{d}{diccTrie($\sigma$)}, \In{k}{$string$}}{}
  [$d$ \igobs $d_0 \land$ def?($k$, $d_0$) ]
  {$d \igobs$ borrar($k$, $d_0$)}
  [$\Ogr(|k|)$]
  [elimina la entrada $k$ del diccionario.] 
  
  \section*{Operaciones del iterador}

  \InterfazFuncion{CrearItClaves}{\In{d}{diccTrie($\sigma$)}}{itDiccTrie($\sigma$)}
  {alias(esPermutación?(SecuSuby($res$), clavesExtendidas($d$))) $\land$ vacia?(Anteriores($res$))}
  [$\Ogr(n)$]
  [crea un iterador del diccionario, de forma que pueda ser recorrido completamente aplicando iterativamente Siguiente.]
  
  \InterfazFuncion{HaySiguiente?}{\In{it}{itDiccTrie($\sigma$)}}{bool}
  {$res$ $\igobs$ haySiguiente($it$)}
  [$\Ogr(1)$]
  [devuelve $true$ si y sólo si quedan elementos para iterar.]  

  \InterfazFuncion{SiguienteClave}{\In{it}{itDiccTrie($\sigma$)}}{string}
  [haySiguiente?($it$)]  
  {alias($res$ $\igobs$ siguiente($it$).clave)}
  [$\Ogr(1)$]
  [devuelve la clave de la entrada del diccionario apuntada por el iterador.]  
  [$res$ no es modificable.]
  
  \InterfazFuncion{SiguienteSignificado}{\In{it}{itDiccTrie($\sigma$)}}{$\sigma$}
  [haySiguiente?($it$)]  
  {alias($res$ $\igobs$ siguiente($it$).significado)}
  [$\Ogr(1)$]
  [devuelve el significado de la entrada del diccionario apuntada por el iterador.]  
  [$res$ no es modificable.]  
  
  \InterfazFuncion{Avanzar}{\Inout{it}{itDiccTrie($\sigma$)}}{}
  [$it \igobs it_0 \land$ haySiguiente?($it_0$)]  
  {$it$ $\igobs$ avanzar($it_0$)}
  [$\Ogr(1)$]
  [avanza a la posición siguiente del iterador.] 
  
\end{Interfaz}

\clearpage

\section{Módulo DiccAVL($\kappa, \sigma$)}
\begin{Interfaz}
  \textbf{parámetros formales}\hangindent=2\parindent\\
  \parbox{1.7cm}{\textbf{géneros}} $\sigma$\\
  \parbox[t]{1.7cm}{\textbf{función}}\parbox[t]{\textwidth-2\parindent-1.7cm}{    	
    \InterfazFuncion{Copiar}{\In{s}{$\sigma$}}{$\sigma$}
    {$res \igobs s$}
    [$\Theta(copy(s))$]
    [función de copia de $\sigma$]
    }
      
  \textbf{se explica con}: \tadNombre{diccionario($\kappa, \sigma$)}.

  \textbf{géneros}: \TipoVariable{diccAVL($\kappa, \sigma$)}.

  \section*{Operaciones básicas DiccAVL($\kappa, \sigma$)}

  \InterfazFuncion{Vacío}{}{diccAVL($\kappa, \sigma$)}
  {$res$ \igobs vacío()}
  [$\Theta(1)$]
  [genera un diccionario vacío.]

  \InterfazFuncion{Definir}{\Inout{d}{diccAVL($\kappa, \sigma$)}, \In{k}{$\kappa$}, \In{s}{$\sigma$}}{}
  [$d \igobs d_0$]  
  {$d$ \igobs definir($k, s, d_0$)}
  [$\Theta(log(n))$\footnote{Dentro de este módulo, n es la cantidad de entradas del diccionario.}]
  [define la clave $k$ con el sinificado $s$.]
  [$k$ y $s$ se definen por referencia.]
    
  \InterfazFuncion{Definido?}{\In{d}{diccAVL($\kappa, \sigma$)}, \In{k}{$\kappa$}}{bool}
  {$res$ \igobs def?($k$, $d$) }
  [$\Theta(log(n))$]
  [devuelve $true$ si la clave $k$ está definida en el diccionario.]
  
  \InterfazFuncion{Obtener}{\In{d}{diccAVL($\kappa, \sigma$)}, \In{k}{$\kappa$}}{$\sigma$}
  [def?($k$, $d$)]  
  {alias($res$ \igobs obtener($k$, $d$))}
  [$\Theta(log(n))$,]
  [devuelve el significado de la clave $k$ en $d$.]
  [$res$ es modificable si y sólo si $d$ lo es]
  
  \InterfazFuncion{Borrar}{\In{d}{diccAVL($\kappa, \sigma$)}, \In{k}{$\kappa$}}{}
  [$d$ \igobs $d_0 \land$ def?($k$, $d_0$) ]
  {$d \igobs$ borrar($k$, $d_0$)}
  [$\Theta(log(n))$]
  [elimina la entrada $k$ del diccionario.] 
  
\end{Interfaz}

\begin{Representacion}
  
  
  
  \section*{Representación de DiccAVL($\kappa, \sigma$)}

  \begin{Estructura}{diccAVL($\kappa, \sigma$)}[puntero(puntero(nodo))]
    \begin{Tupla}[nodo]
      \tupItem{clave}{$\kappa$}
      \tupItem{\\significado}{$\sigma$}
      \tupItem{\\izq}{puntero(nodo)}
      \tupItem{\\der}{puntero(nodo)}
      \tupItem{\\altura}{nat}
    \end{Tupla}

  \end{Estructura}

  \Rep[puntero(puntero(nodo))][a]{
  					   $a \neq $NULL \yluego \\
  					   \hspace*{4em}$*a$ = NULL \oluego \\
  					   \hspace*{4em}($\exists n:$nat) finaliza($*a$, $n$) \yluego \\
  					   \hspace*{4em}$(*a)\rightarrow altura$\footnote{La notación $p\rightarrow campo$, donde $p$ es un puntero a una tupla y $campo$ algun campo de esa tupla, equivale a decir ($*p$).$campo$.} = alto($*a$) $\land$ \\
  					   \hspace*{4em}diferenciaDeAlturas($(*a)\rightarrow izq$, $(*a)\rightarrow der$)$\leq$ 1 $\land$ \\
  					   \hspace*{4em}($\forall c:\kappa$) ($c\in$ ClavesDelAvl($(*a)\rightarrow izq$) $\Rightarrow (*a)\rightarrow clave \leq c)$ $\land $\\
  					   \hspace*{4em}($\forall c:\kappa$) ($c\in$ ClavesDelAvl($(*a)\rightarrow der$) $\Rightarrow (*a)\rightarrow clave \leq c)$ $\land $\\
  					   \hspace*{4em}Rep($(*a) \rightarrow izq$) $\land$ Rep($(*a) \rightarrow der$)}
  					   
  
  \AbsFc[puntero(puntero(nodo))]{dicc($\kappa, \sigma$)}[a]{
	$d$:dicc($\kappa, \sigma$) |
	($\forall c:\kappa$) (def?($c, d$) $\Leftrightarrow$ $c \in$ clavesDelAvl($*a$)) \yluego\\
	\hspace*{5.9em}($\forall c:\kappa$) (def?($c, d$) \impluego obtener($c, d$) = darSignificado($*a, c$))
  }  
  
  \tadOperacion{finaliza}{puntero(nodo), nat}{bool}{}
	\tadAxioma{finaliza($p,n$)}{$n > 0$ \yluego ($p$ = NULL \oluego (finaliza($p\rightarrow izq$, $n-1$) $\land$ finaliza($p\rightarrow der$, $n-1$)))}

  \tadOperacion{alto}{puntero(nodo)}{nat}{}
	\tadAxioma{alto($p$)}{\IF $p = $NULL THEN 0 ELSE $1 + $max(alto($p\rightarrow izq$), alto($p\rightarrow der$))FI}
 
  \tadOperacion{diferenciaDeAlturas}{puntero(nodo) , puntero(nodo)}{nat}{}
	\tadAxioma{diferenciaDeAlturas($p, p'$)}{\IF $p\rightarrow altura \leq p'\rightarrow altura$
											THEN $(p'\rightarrow altura) - (p\rightarrow altura)$
											ELSE $(p\rightarrow altura) - (p'\rightarrow altura)$
											FI} 
 
  \tadOperacion{clavesDelAvl}{puntero(nodo)}{conj($\kappa$)}{}
	\tadAxioma{clavesDelAvl($p$)}{\IF $p = $NULL THEN $\emptyset$ ELSE Ag($p\rightarrow clave$, clavesDelAvl($p\rightarrow izq$) $\cup$ clavesDelAvl($p\rightarrow der$))FI} 
    
   
  \tadOperacion{darSignificado}{puntero(nodo)/p , $\kappa$/c}{$\sigma$}{$c \in$ clavesDelAvl($p$)}
	\tadAxioma{darSignificado($p, c$)}{\IF $p\rightarrow clave$ = $c$ THEN $p\rightarrow significado$ ELSE 
											{\IF $c\in$ clavesDelAvl($p\rightarrow izq$)
											THEN darSignificado($p\rightarrow izq$, $c$)
											ELSE darSignificado($p\rightarrow der$, $c$)
											FI}
										FI} 

\end{Representacion}

\begin{Algoritmos}

\begin{algorithm}
\caption{actualizarAltura}
\begin{algorithmic}[1]
\Procedure{actualizarAltura}{\texttt{in/out} $p$ : \texttt{puntero(nodo)}}
	\If{$p\neq$ NULL}
    \State $(p\to altura)\gets 1 + $max($(p\to izq)\to altura, (p\to der)\to altura)$
  \EndIf
\EndProcedure
\end{algorithmic}
\end{algorithm}

\begin{algorithm}
\caption{rotarSimple}
\begin{algorithmic}[1]
\Procedure{rotarSimple}{\texttt{in/out} $a$ : \texttt{puntero(puntero(nodo))}, \texttt{in} $rota\_izq$ : \texttt{bool}}
	\State puntero(nodo) $a_1$
  \If{$rota\_izq$}
    \State $a_1 \gets (*a\to izq)$
    \State $(*a\to izq) \gets (a_1\to der)$
    \State $(a_1\to der)\gets *a$
  \Else
    \State $a_1 \gets (*a\to der)$
    \State $(*a\to der) \gets (a_1\to izq)$
    \State $(a_1\to izq)\gets *a$
  \EndIf

  \State actualizarAltura($*a$)
  \State actualizarAltura($a_1$)
  
  \State $*a \gets a_1$
\EndProcedure
\end{algorithmic}
\end{algorithm}

\begin{algorithm}
\caption{rotarDoble}
\begin{algorithmic}[1]
\Procedure{rotarDoble}{\texttt{in/out} $a$ : \texttt{puntero(puntero(nodo))}, \texttt{in} $rota\_izq$ : \texttt{bool}}
	\If{$rota\_izq$}
    \State rotarSimple($\&(*a\to izq), false$)
    \State rotarSimple($a, true$)
  \Else
		\State rotarSimple($\&(*a\to der), true$)
    \State rotarSimple($a, false$)
  \EndIf
\EndProcedure
\end{algorithmic}
\end{algorithm}

\begin{algorithm}
\caption{balancear}
\begin{algorithmic}[1]
\Procedure{balancear}{\texttt{in/out} $a$}
	\If{$*a \neq$ NULL}
 		\If{$(*a\to izq)\to altura \geq (*a\to der)\to altura$}		
 			\If{$(*a\to izq)\to altura - (*a\to der)\to altura = 2$}
 				\Comment desequilibrio hacia la izquierda
 				\If{$((*a\to izq)\to izq)\to altura \geq ((*a\to izq)\to der)\to altura $}
 		   		\Comment desequilibrio simple
 		   		\State rotarSimple($a, true$)
 		   	\Else \Comment desequilibrio doble
 		   		\State rotarDoble($a, true$)
 		   	\EndIf			
			\EndIf	
			\Else
			\If{$(*a\to der)\to altura - (*a\to izq)\to altura = 2$}	
				\Comment desequilibrio hacia la derecha
				\If{$((*a\to der)\to der)\to altura \geq ((*a\to der)\to der)\to altura $}
 		   		\Comment desequilibrio simple
 		   		\State rotarSimple($a, false$)
 		   	\Else \Comment desequilibrio doble
 		   		\State rotarDoble($a, false$)
 		   	\EndIf
			\EndIf
		\EndIf
  \EndIf
\EndProcedure
\end{algorithmic}
\end{algorithm}

\begin{algorithm}
\caption{Vacio}
\begin{algorithmic}[1]
\Procedure{iVacio}{}$\to res$ : \texttt{puntero(puntero(nodo))}
	\State $*res \gets$ NULL
\EndProcedure
\end{algorithmic}
\end{algorithm}

\begin{algorithm}
\caption{Definir}
\begin{algorithmic}[1]
\Procedure{iDefinir}{\texttt{in/out} $d$ : \texttt{puntero(puntero(nodo))}, \texttt{in} $k$ : \texttt{$\kappa$}, \texttt{in} $s$ : \texttt{$\sigma$}}
	\If{$*d =$ NULL}
    \State nodo $nuevo \gets$ $(k, s,$ NULL, NULL, $1)$
  \Else
		\If{$k < *d\to clave$}
			\State Definir($\&(*d\to izq), k, s$)
		\Else
			\State Definir($\&(*d\to der), k, s$)
		\EndIf
  \EndIf
  \State balancear($d$)
  \State actualizarAltura($*d$)
\EndProcedure
\end{algorithmic}
\end{algorithm}

\begin{algorithm}
\caption{Definido?}
\begin{algorithmic}[1]
\Procedure{iDefinido?}{\texttt{in} $d$ : \texttt{puntero(puntero(nodo))}, \texttt{in} $k$ : \texttt{$\kappa$}}$\to res$ : \texttt{bool}
	\If{$*d = NULL$}
	 \State $res\gets false$
	 \Else	
		\If{$k = *d\to clave$}
			\State $res\gets true$
		\Else
			\If{$k < *d\to clave$}	
				\State Definido?($\&(*d\to izq), k$)		
			\Else
				\State Definido?($\&(*d\to der), k$)
			\EndIf
		\EndIf
	\EndIf
\EndProcedure
\end{algorithmic}
\end{algorithm}

\begin{algorithm}
\caption{Obtener}
\begin{algorithmic}[1]
\Procedure{iObtener}{\texttt{in} $d$ : \texttt{puntero(puntero(nodo))}, \texttt{in} $k$ : \texttt{$\kappa$}}$\to res$ : \texttt{$\sigma$}	
	\If{$k = *d\to clave$}
		\State $res\gets (*d\to significado)$
	\Else
		\If{$k < *d\to clave$}	
			\State Obtener($\&(*d\to izq), k$)		
		\Else
			\State Obtener($\&(*d\to der), k$)
		\EndIf
	\EndIf

\EndProcedure
\end{algorithmic}
\end{algorithm}

\begin{algorithm}
\caption{Borrar}
\begin{algorithmic}[1]
\Procedure{iBorrar}{\texttt{in/out} $d$ : \texttt{puntero(puntero(nodo))}, \texttt{in} $k$ : \texttt{$\kappa$}}
	\State puntero(nodo) $aux$
	\If{$k < (*d\to clave)$}
		\State Borrar($\&(*d\to izq), k$)
	\Else 
		\If{$k > *d\to clave$}
			\State Borrar($\&(*d\to der), k$)
		\Else 
			\If{$*d\to izq =$ NULL $\land *d\to der =$ NULL}
				\Comment Es una hoja
				\State delete($*d$)
				\State $*d\gets$ NULL
			\Else
				\If{$*d\to izq =$ NULL}
					\Comment subárbol izquierdo vacío
					\State $aux \gets *d$
					\State $*d\gets (*d\to der)$
					\State delete($aux$)
				\Else
					\If{$*d\to der =$ NULL}
						\Comment subárbol derecho vacío
						\State $aux \gets *d$
						\State $*d\gets (*d\to izq)$
						\State delete($aux$)
					\Else
						\Comment el árbol tiene dos hijos
						
					\EndIf
				\EndIf
			\EndIf 
		\EndIf
	\EndIf
	\State balancear($d$)
	\State actualizarAltura($*d$)
\EndProcedure
\end{algorithmic}
\end{algorithm}

\end{Algoritmos}
\clearpage

\section{Módulo ConjAVL($\alpha$)}
\begin{Interfaz}
  \textbf{parámetros formales}\hangindent=2\parindent\\
  \parbox{1.7cm}{\textbf{géneros}} $\alpha$\\
    \parbox[t]{1.7cm}{\textbf{función}}\parbox[t]{\textwidth-2\parindent-1.7cm}{    	
    \InterfazFuncion{$\bullet <_{\alpha} \bullet$}{\In{a}{$\alpha$}, \In{a'}{$\alpha$}}{$bool$}
    {$res \igobs (a <_{\alpha} a')$}
    [$\Ogr(comp_{\alpha}(a, a'))$]
    [relación de orden total de $\alpha$]
    }	
	
  \textbf{se explica con}: \tadNombre{conj($\alpha$)}.

  \textbf{géneros}: \TipoVariable{conjAVL($\alpha$)}.

  \section*{Operaciones básicas ConjAVL($\alpha$)}

  \InterfazFuncion{Vacio}{}{conjAVL($\alpha$)}
  {$res \igobs \emptyset$}
  [$\Ogr(1)$]
  [genera un conjunto vacío.]

  \InterfazFuncion{Agregar}{\Inout{c}{conjAVL($\alpha$)}, \In{a}{$\alpha$}}{}
  [$c \igobs c_0$]  
  {$c$ \igobs Ag($a,c_0$)}
  [$\Ogr(log(n) \times comp_{\alpha}(a, a'))$\footnote{Dentro de este módulo, n es la cantidad de elementos del conjunto.}]
  [agrega el elemento $a$ al conjunto $c$.]
  [$a$ se agrega por referencia.]
    
  \InterfazFuncion{Pertenece?}{\In{c}{conjAVL($\alpha$)}, \In{a}{$\alpha$}}{bool}
  {$res$ \igobs ($a\in c$) }
  [$\Ogr(log(n) \times comp_{\alpha}(a, a'))$]
  [devuelve $true$ si y sólo si el elemento $a$ pertenece al conjunto.]
  
  \InterfazFuncion{Eliminar}{\In{c}{conjAVL($\alpha$)}, \In{a}{$\alpha$}}{}
  [$c$ \igobs $c_0 \land a \in c$ ]
  {$c \igobs (c_0 - \{a\})$}
  [$\Ogr(log(n) \times comp_{\alpha}(a, a'))$]
  [elimina el elemento $a$ del conjunto.] 
  
  \underline{Nota:} Si $comp_{\alpha}(a, a')$ es constante entonces todas las operaciones del conjunto salvo Vacio son $\Ogr(log(n))$
  \newpage
\end{Interfaz}

\begin{Representacion}
    
  \section*{Representación de ConjAVL($\alpha$)}

  \begin{Estructura}{conjAVL($\alpha$)}[diccAVL($\alpha$, nat)]
  \end{Estructura}
  
  \Rep[diccAVL($\alpha${,} nat)][d]{true}
  
  \AbsFc[diccAVL($\alpha${,} nat)]{conjAVL($\alpha$)}[d]{
	$c$ : conj($\alpha$) | ($\forall a: \alpha$) ($a\in c \Leftrightarrow a\in claves(d)$)}
\end{Representacion}


\begin{Algoritmos}

\begin{algorithm}
\caption{Vacio}
\begin{algorithmic}[1]
\Procedure{iVacio}{}$\to res$ : \texttt{diccAVL($\alpha$, nat)}
	\State $res \gets$ Vacio()
	\Comment $\Ogr(1)$
\EndProcedure
\end{algorithmic}
\underline{Complejidad:} $\Ogr(1)$
\end{algorithm}

\begin{algorithm}
\caption{Agregar}
\begin{algorithmic}[1]
\Procedure{iAgregar}{\texttt{in/out} $c$ : \texttt{diccAVL($\alpha$, nat)}, \texttt{in} $a$ : \texttt{$\alpha$}}
	\State Definir($c, a, 0$)
	\Comment $\Ogr(log(n) \times comp_{\alpha}(a, a'))$
\EndProcedure
\end{algorithmic}
\underline{Complejidad:} $\Ogr(log(n) \times comp_{\alpha}(a, a'))$
\end{algorithm}

\begin{algorithm}
\caption{Pertenece?}
\begin{algorithmic}[1]
\Procedure{iPertenece?}{\texttt{in/out} $c$ : \texttt{diccAVL($\alpha$, nat)}, \texttt{in} $a$ : \texttt{$\alpha$}}$\to res$ : \texttt{bool}
	\State $res \gets$ Definido?($c, a$)
	\Comment $\Ogr(log(n) \times comp_{\alpha}(a, a'))$
\EndProcedure
\end{algorithmic}
\underline{Complejidad:} $\Ogr(log(n) \times comp_{\alpha}(a, a'))$
\end{algorithm}

\begin{algorithm}
\caption{Eliminar}
\begin{algorithmic}[1]
\Procedure{iEliminar}{\texttt{in/out} $c$ : \texttt{diccAVL($\alpha$, nat)}, \texttt{in} $a$ : \texttt{$\alpha$}}
	\State Borrar($c, a$)
	\Comment $\Ogr(log(n) \times comp_{\alpha}(a, a'))$
\EndProcedure
\end{algorithmic}
\underline{Complejidad:} $\Ogr(log(n) \times comp_{\alpha}(a, a'))$
\end{algorithm}
\end{Algoritmos}

\clearpage


\section{Módulo Cola de Prioridad($\alpha$)}





\begin{Interfaz}

  \textbf{se explica con}: colaPrior($\alpha$)

  \textbf{usa}: Nat, bool
  
  \textbf{genero}: colaPrior($\alpha$)


  \subsubsection{Operaciones de Cola de Prioridad \footnote{asumimos que la complejidad de comparar $\alpha$ con $<_\alpha$ es $\Ogr(1)$ para no sobrecargar de notación, la complejidad expresada en su totalidad está en la parte de algoritmos}}

  \InterfazFuncion{Vacia}{}{colaPrior($\alpha$)}
  [true]
  {$res$ $\igobs$ vacia}
  [O(1)]
  [Crea una cola de prioridad]\\ 
  
  \InterfazFuncion{Vacia?}{\In{c}{colaPrior($\alpha$)}}{bool}
  [true]
  {$res$ $\igobs$ vacia?(c)}
  [O(1)]
  [Dice si la cola no tiene ningun elemento]\\ 

  \InterfazFuncion{Proximo}{\In{c}{colaPrior($\alpha$)}}{$\alpha$}
  [$\neg$vacia?($c$)]
  {$res$ $\igobs$ proximo($c_0$)}
  [O(1)]
  [Devuelve el elemento mas prioritario]\\   
  
  \InterfazFuncion{Desencolar}{\Inout{c}{colaPrior($\alpha$)}}{$\alpha$}
  [$\neg$vacia?($c$) $\land$ $c$ $\igobs$ $c_0$]
  {$res$ $\igobs$ proximo($c_0$) $\land$ $c$ $\igobs$ desencolar($c_0$)}
  [O(log(tamano(c)))]
  [Quita el elemento mas prioritario]\\   
  
  \InterfazFuncion{Encolar}{\Inout{c}{colaPrior($\alpha$)}, \In{a}{$\alpha$}}{}
  [$c$ $\igobs$ $c_0$ $\land$ $\neg$esta($a$, $c$)] %agregar el esta
  {$c$ $\igobs$ encolar(a,$c_0$)}
  [O(log(tamano(c)))]
  [Agrega al elemento a a la cola de prioridad]
  [El iterador se invalida si, y solo si se elimina el elemento siguiente del iterador sin llamar a la funcion Eliminar del mismo]\\ 

\end{Interfaz}


\subsection{Representación de la cola de prioridad}
\begin{Representacion}
  
  \begin{Estructura}{colaPrior($\alpha$)}[ab($\alpha$)]

  \begin{Tupla}[ab($\alpha$)]
    \tupItem{tam}{nat}%
    \tupItem{\\ cabeza}{puntero(nodo($\alpha$))}% 
  \end{Tupla}

  ~

  \begin{Tupla}[nodo($\alpha$)]
    \tupItem{padre}{puntero(nodo($\alpha$))}%
    \tupItem{\\ izq}{puntero(nodo($\alpha$))}%
    \tupItem{\\ der}{puntero(nodo($\alpha$))}%
    \tupItem{\\ dato}{puntero($\alpha$)}%
  \end{Tupla}

  \end{Estructura}



  \Rep[ab($\alpha$)][h]{$tam$ = tamano($cabeza$) $\land$ propiedadMinab($cabeza$)}\mbox{}
  

  \tadOperacion{tamano}{puntero(nodo($\alpha$))}{nat}{}
  \tadAxioma{tamano(NULL)}{0}
  \tadAxioma{tamano(\& $x$)}{1 + tamano($x$.izq) + tamano($x$.der)}

  \tadOperacion{propiedadMinHeap)}{puntero(nodo($\alpha$))}{bool}{}
  \tadAxioma{propiedadMinHeap(NULL)}{true}
  \tadAxioma{propiedadMinHeap(\& $x$)}{$x.dato < x.izq \to dato \land x.dato < x.der \to dato \land$ \\ propiedadMinHeap($x$.izq) $\land$ propiedadMinHeap($x$.der)}
   
  \AbsFc[ab($\alpha$)]{colaPrior($\alpha$)}[h]{
    $c$ : colaPrior($\alpha$) / vacia?($c$) = ($h.cabeza$ = NULL) $\land$\\ 
    $\neg$ vacia?($c$) $\Rightarrow$ proximo($c$) = $h.cabeza \to dato$ $\land$\\ 
  desencolar($c$) = eliminar($h$, proximo($c$))}




\end{Representacion}






\begin{Algoritmos}



\begin{algorithm}
\caption{Vacia}
\begin{algorithmic}[1]
  \Procedure{iVacia}{\ } $\to res$ : \texttt{ab($\alpha$)}
  \State $tam \gets$ 0 \Comment $\Ogr(1)$
  \State $cabeza \gets$ NULL \Comment $\Ogr(1)$
  \State $res \gets$ ($tam$,$cabeza$) \Comment $\Ogr(1)$
 \EndProcedure
 \underline{Complejidad:} $\Ogr(1)$
 \underline{Justificación:} $\Ogr(1)+\Ogr(1)+\Ogr(1) = \Ogr(1)$
\end{algorithmic}
\end{algorithm}


\begin{algorithm}
\caption{Vacia?}
\begin{algorithmic}[1]
  \Procedure{iVacia?}{\texttt{in} h : \texttt{ab($\alpha$)}} $\to res$ : \texttt{heap}
  \State $res \gets$ $h.cabeza$ = NULL \Comment $\Ogr(1)$ 
 \EndProcedure
 \underline{Complejidad:} $\Ogr(1)$
\end{algorithmic}
\end{algorithm}


\begin{algorithm}
\caption{Proximo}
\begin{algorithmic}[1]
  \Procedure{Proximo}{\texttt{in} h : \texttt{ab($\alpha$)}} $\to res$ : $\alpha$
  \State $res \gets$ $h.cabeza \to dato$ \Comment $\Ogr(1)$ 
 \EndProcedure
 \underline{Complejidad:} $\Ogr(1)$
\end{algorithmic}
\end{algorithm}



\begin{algorithm}
\caption{Desencolar}
\begin{algorithmic}[1]
  \Procedure{iDesencolar}{\texttt{in/out} h : \texttt{ab($\alpha$)}} $\to res$ : $\alpha$
  \State $res \gets$ h.cabeza$\to$ dato

  \If{$tam$ = 1} \Comment $\Ogr(1)$
    \State delete($cabeza$) \Comment $\Ogr(1)$
    \State $cabeza \gets$ NULL \Comment $\Ogr(1)$
  \Else
    \State lista($nat$) $recorridoHastaUltimo \gets$ Vacia() \Comment $\Ogr(1)$
    \State $x \gets h.tam$ \Comment $\Ogr(1)$
    \While{x > 1} \Comment Se repite $\Ogr(log(x)) = \Ogr(log(tam))$ veces
      \State agregarAdelante($recorridoHastaUltimo$, $x$\% 2) \Comment $\Ogr(copy(\alpha))$
      \State $x \gets \frac{x}{2}$ \Comment $\Ogr(1)$
    \EndWhile
    \State puntero(nodo) $ultimo \gets h.cabeza$ \Comment $\Ogr(1)$
    \State itLista($nat$) $i \gets$ crearIt(recorridoHastaUltimo) \Comment $\Ogr(1)$
    \While{haySiguiente?($i$)}  \Comment Se repite $\Ogr(|recorridoHastaUltimo|) = \Ogr(log(tam))$ veces
      \If {Siguiente(i) = 0} \ $p \gets p.izq$ \Comment $\Ogr(1)$
      \Else \  $p \gets p.izq$ \Comment $\Ogr(1)$
      \EndIf
      \State $i \gets$ Avanzar($i$) \Comment $\Ogr(1)$
    \EndWhile
    
    \If{($p \to padre)\to der$ = $p$} \Comment $\Ogr(1)$
      \State $p \to padre \to der \gets$ NULL \Comment $\Ogr(1)$
    \Else
      \State $p \to padre \to izq \gets$ NULL \Comment $\Ogr(1)$
    \EndIf
  
    \State $p \to der \gets cabeza \to der$ \Comment $\Ogr(1)$
    \State $p \to izq \gets cabeza \to izq$ \Comment $\Ogr(1)$
    \State $p \to padre \gets$ NULL \Comment $\Ogr(1)$
  
    \State puntero(nodo) $nodoActual \gets p$
    \While{$(nodoActual \to izq \neq$ NULL $\land$ $nodoActual \to dato$ > $nodoActual \to izq)$ $\lor$ $(nodoActual \to der \neq$ NULL $\land$ $nodoActual \to dato$ > $nodoActual \to der)$}  \Comment Se repite $\Ogr(log(tam))$ veces.
    \If{$nodoActual \to der$ = NULL} \Comment $\Ogr(1)$
        \If{$nodoActual \to izq \to dato$ < $nodoActual \to dato$} \Comment $\Ogr(comparar(\alpha, <_\alpha))$
          \State Intercambiar($h$, $nodoActual$, $nodoActual \to izq$) \Comment $\Ogr(1)$
        \Else 
          \If{$nodoActual \to izq \to dato < nodoActual \to dato \land nodoActual \to izq \to dato < nodoActual \to der \to dato$}  \Comment $\Ogr(comparar(\alpha, <_\alpha))$ 
            \State Intercambiar($h$, $nodoActual$, $nodoActual \to izq$) \Comment $\Ogr(1)$
          \Else
             \State Intercambiar($h$, $nodoActual$, $nodoActual \to der$) \Comment $\Ogr(1)$
          \EndIf
        \EndIf
      \EndIf
    \EndWhile
   \EndIf
  \State $tam--$ \Comment $\Ogr(1)$
 \EndProcedure
 \underline{Complejidad:} $\Ogr(log(tam) \cdot max\{copy(\alpha), comparar(\alpha, <_\alpha)\})$
 \underline{Justificación:} $\Ogr(log(tam)) \cdot \Ogr(copy(\alpha)) + 3\Ogr(1) + \Ogr(log(tam)) \cdot \Ogr(1) + \Ogr(log(tam)) \cdot (6\Ogr(1) + \Ogr(comparar(\alpha, <_\alpha)) = \Ogr(log(tam)) \cdot \Ogr(copy(\alpha)) + \Ogr(log(tam)) \cdot \Ogr(comparar(\alpha, <_\alpha) = \Ogr(log(tam)) \cdot (\Ogr(copy(\alpha)) + \Ogr(comparar(\alpha, <_\alpha)) = \Ogr(log(tam) \cdot max\{copy(\alpha), comparar(\alpha, <_\alpha)\}) $.
\end{algorithmic}
\end{algorithm}




\begin{algorithm}
\caption{Encolar}
\begin{algorithmic}[1]
  \Procedure{iEncolar}{\texttt{in/out} $h$ : \texttt{ab($\alpha$)}, \texttt{in} $a$ : $\alpha$} $\to res$ : $\alpha$
  \State $nuevoNodo \gets$ Nodo(NULL, NULL, NULL, $a$)

  \If{$tam$ = 0}
    \State $cabeza \gets p$
  \Else

    \State lista($nat$) $recorridoHastaUltimo \gets$ Vacia()
    \State $x \gets h.tam+1$
    \While{x > 1}
      \State agregarAdelante($recorridoHastaUltimo$, $x$\% 2)
      \State $x \gets \frac{x}{2}$
    \EndWhile
    \State puntero(nodo) $padreNuevo \gets h.cabeza$
    \State itLista($nat$) $i \gets$ crearIt(recorridoHastaUltimo)
    
    \Repeat
      \If {Siguiente(i) = 0} \ $padreNuevo \gets padreNuevo.izq$
      \Else \  $padreNuevo \gets padreNuevo.izq$
      \EndIf
      \State $i \gets$ Avanzar($i$)
    \Until{Siguiente?($i$) $\neq$ ultimo(recorridoHastaUltimo)}
    
    \State $nuevo \to padre \gets padreNuevo$

    \If{ultimo(recorridoHastaUltimo) = 0}
      \State $padreNuevo \to izq \gets nuevoNodo$
    \Else
       \State $padreNuevo \to der \gets nuevoNodo$
    \EndIf

    \State puntero(nodo) $nodoActual \gets nuevoNodo$

    \While{$(nodoActual \to padre \neq NULL \land nodoActual \to padre \to dato > nodoActual \to dato$}

    Intercambiar($h$, $nodoActual \to padre$, $nodoActual$)
    \EndWhile
   \EndIf
  \State $tam++$
 \EndProcedure
\end{algorithmic}
\end{algorithm}





\begin{algorithm}
\caption{Intercambiar}
\begin{algorithmic}[1]
  \Procedure{iIntercambiar}{\texttt{in/out} $h$ : \texttt{ab($\alpha$)}, \texttt{in/out} $padre$ : \texttt{puntero(nodo($\alpha$))}, \texttt{in/out} $hijo$ : \texttt{puntero(nodo($\alpha$))}}
  \If{$hijo$ = $padre \to izq$} \Comment $\Ogr(1)$
    \State puntero(nodo) $derechoPadre \gets padre \to der$ \Comment $\Ogr(1)$
    \State $padre \to der \gets hijo \to der$ \Comment $\Ogr(1)$
    \State $padre \to izq \gets hijo \to izq$ \Comment $\Ogr(1)$
    \State $hijo \to izq \gets padre$ \Comment $\Ogr(1)$
    \State $hijo \to der \gets derechoPadre$ \Comment $\Ogr(1)$
  \Else
    \State puntero(nodo) $izquierdoPadre \gets padre \to izq$ \Comment $\Ogr(1)$
    \State $padre \to izq \gets hijo \to izq$ \Comment $\Ogr(1)$
    \State $padre \to der \gets hijo \to der$ \Comment $\Ogr(1)$
    \State $hijo \to der \gets padre$ \Comment $\Ogr(1)$
    \State $hijo \to izq \gets izquierdoPadre$ \Comment $\Ogr(1)$
  \EndIf

  \State $hijo \to padre \gets padre \to padre$ \Comment $\Ogr(1)$
  \State $padre \to padre \gets hijo$ \Comment $\Ogr(1)$
  
  \If{$hijo \to padre$ = NULL} \Comment $\Ogr(1)$
    \State$h$.cabeza $\gets hijo$ \Comment $\Ogr(1)$
  \EndIf 


 \EndProcedure
 \underline{Complejidad:} $\Ogr(1)$
 \underline{Justificación:} Queda claro que se producen siempre a lo sumo 8 operaciones que con costo $\Ogr(1)$.
\end{algorithmic}
\end{algorithm}










\end{Algoritmos}












\clearpage

\section{Módulo DCNet}
\begin{Interfaz}
  
  \textbf{se explica con}: \tadNombre{dcnet}.

  \textbf{géneros}: \TipoVariable{dcnet}.

  \section*{Operaciones básicas de DCNet}

  \InterfazFuncion{IniciarDCNet}{\In{r}{\tadNombre{red}}}{\tadNombre{dcnet}}
  [ $True$ ]
  {$ res \igobs IniciarDCNet(r)  $ }
  [$\Theta(1)$]
  [Genera una DCNet con las computadoras y conexiones de la red pasada como parámetro.]

  \InterfazFuncion{CrearPaquete}{\Inout{s}{\tadNombre{dcnet}}, \In{p}{\tadNombre{paquete}}}{}
  [$s \igobs s_0 \land 
   \neg((\exists p':paquete) (paqueteEnTransito?(s_0,p') \land id(p') \igobs id(p)) \land\\ 
   \hspace*{3em} origen(p) \in computadoras(red(s_0)) \yluego
   destino(p) \in computadoras(red(s_0)) \yluego\\
   \hspace*{3em} hayCamino?(red(s), origen(p), destino(p))$]
  {$s \igobs crearPaquete(s_0, p)$}
  [$\Theta()$]
  [Agrega el paquete $p$ a la cola de la computadora $p.origen$.]
  
  \InterfazFuncion{AvanzarSegundo}{\Inout{s}{\tadNombre{dcnet}}}{}
  [$s \igobs s_0$]  
  {$res \igobs avanzarSegundo(s_0)$}
  [$\Theta()$]
  [Avanza un segundo; todas las computadoras que tengan paquetes por enviar los envían.]
  
  
  \InterfazFuncion{red}{\In{s}{\tadNombre{dcnet}}}{\tadNombre{red}}
  [$ True $]  
  {$res \igobs red(s_0)$}
  [$\Theta(1)$]
  [Expresa la red que esta contenida en la DCNet.]

  
  \InterfazFuncion{caminoRecorrido}{\In{s}{\tadNombre{dcnet}}, \In{p}{\tadNombre{paquete}}}{\tadNombre{secu(compu)}}
  [$ paqueteEnTransito?(s,p) $]
  {$res \igobs caminoRecorrido(s, p)$}
  [$\Theta()$]
  [Expresa el camino de computadoras recorrido por el paquete $p$ desde su inicio hasta posicion actual.]


  \InterfazFuncion{cantidadEnviados}{\In{s}{\tadNombre{dcnet}}, \In{c}{\tadNombre{compu}}}{\tadNombre{nat}}
  [$ c \in computadoras(red(s)) $]  
  {$res \igobs cantidadEnviados(s, c)$}
  [$\Theta()$]
  [Expresa la cantidad de mensajes enviados por la compu $c$.]

  
  \InterfazFuncion{enEspera}{\In{s}{\tadNombre{dcnet}}, \In{c}{\tadNombre{compu}}}{\tadNombre{conj(paquete)}}
  [$ c \in computadoras(red(s)) $]  
  {$res \igobs enEspera(s, c)$}
  [$\Theta()$]
  [Expresa el camino de computadoras recorrido por el paquete $p$ desde su inicio hasta posicion actual.]


  \InterfazFuncion{PaqueteEnTransito}{\In{s}{\tadNombre{dcnet}}, \In{p}{\tadNombre{paquete}}}{\tadNombre{bool}}
  [$ True $]  
  {$res \igobs paqueteEnTransito?(s, p)$}
  [$\Theta()$]
  [Expresa si el paquete $p$ esta en alguna computadora.]
  
  
  \InterfazFuncion{LaQueMasEnvio}{\In{s}{\tadNombre{dcnet}}}{\tadNombre{compu}}
  [$ True $]
  {$res \igobs laQuemasEnvio(s)$}
  [$\Theta(1)$]
  [Devuelve la computadora que mas paquetes envio.]
  

\end{Interfaz}

\begin{Representacion}
  
  \section*{Representación de DCNet}

  \begin{Estructura}{dcnet}[net]
    \begin{Tupla}[net]
      \tupItem{proximaEnCamino}{dicc(compu, dicc(compu, compu))}%
      \tupItem{\\ paquetes}{diccTrie(compu,infoPaquetes)}%
      \tupItem{\\ caminosRecorridos}{lista(lista(compu))}
      \tupItem{\\ laQueMasEnvio}{tupla($cuantosEnvio$ : nat, $cualCompu$ : compu)}
      \tupItem{\\ red}{red}
    \end{Tupla}
  
  
  \begin{Tupla}[infoPaquetes]
      \tupItem{colas}{colaPrior($p$ : paq)}
      \tupItem{\\ diccPaqCamino}{diccAVL($p$ : paq, $camino$ : itLista )}
      \tupItem{\\ conjPaquetes}{conj(paq)}
      \tupItem{\\ cantEnviados}{nat}                       
    \end{Tupla}
  \end{Estructura}


  \Rep[net][n]{computadoras(n.red) = claves(n.paquetes) $\yluego$ ($\forall$ c : compu) [((c $\in$ claves(n.paquetes) $\rightarrow$ (($\forall$ p : paquete) p $\in$ definicion(n.paquetes, c).conjPaquetes $\rightarrow$ p $\in$ definicion(n.paquetes,c).colas $\land$ p $\in$ claves(definicion(paquetes, c).diccPaqCamino) $\land$  ($\forall$ p : paquete) p $\in$ claves(definicion(paquetes, c).diccPaqCamino) $\rightarrow$ p $\in$ definicion(n.paquetes, c).conjPaquetes  $\land$  (($\forall$ p : paquete) p $\in$ definicion(n.paquetes, c).colas $\rightarrow $ p $\in$ definiciones(paquetes,c).colas $yluego$ definicion(definicion(n.paquetes,c).diccPaqCamino,p).camino $\in$ caminosMinimos(n.red, p.origen, c)] $\land$ ($\forall$ c1, c2 : compu) estanConectadas(n.red, c1,c2) $\rightarrow$ estaDefinida?(n.proximaEnCamino, c1) $\yluego$ estaDefinida?(definicion(n.proximaEnCamino, c1),c2) $\yluego$ ($\exists$ sc : secu(compu)) sc $\in$ caminosMinimos(n.red, c1, c2) $\land$ definicion(definicion(n.proximaEnCamino,c1),c2) = primero(fin(sc)) $\land$ (n.laQueMasEnvio).cualCompu $\in$ claves(paquetes) $\yluego$ definicion(n.paquetes, (n.laQueMasEnvio).cualCompu).cantEnviados = (n.laQueMasEnvio).cuantosEnvio $\land$ ($\forall$ c : compu) c $\in$ claves(n.paquetes) $\rightarrow$ (n.laQueMasEnvio).cuantosEnvio  $\geq$ definicion(n.paquetes, c).cantEnviados $\yluego$ sinrepetidos(definicion(paquetes, c).colas)         } \mbox{}

 
  \AbsFc[net]{dcnet}[n]{dnt: dcnet | red(dnt) = n.red  $\land$ (($\forall$ p : paquete) (paqueteEnTransito?(dnt,p) $\Rightarrow$ (($\exists$ c : compu) p $\in$ $\pi_2$(definicion(n, c))) $\yluego$
 ($\forall$ p : paquete) ($\exists$ c : compu) c in computadoras(dnt) $\Rightarrow$ siguiente(definicion($\pi_2$(definicion(n.paquetes,c)),p) = caminorecorrido(c,p)  $\yluego$
($\forall$ c : compu) (c $\in$ computadoras(red(dcn)) $\Rightarrow$ (cantidadEnviados(dcn, c) = $\pi_3$(definicion(n.paquetes, c) $\yluego$ 
($\forall$ c : compu) c $\in$ computadoras(dnt) $\Rightarrow$ enEspera(dnt, c) = $\pi_2$(definicion(n.paquetes,c))}

\end{Representacion}



\begin{Algoritmos}


\begin{algorithm}
\caption{Iniciar DCNet}
\begin{algorithmic}[1]
  \Procedure{IniciarDCNet(\textbf{in} $r$ : red) $\to res$ : dcnet}{}
  \State $red \gets r$ \Comment $\Ogr(1)$
  \State $caminosRecorridos \gets$ Vacia() \Comment $\Ogr(1)$ 
  \State $laQueMasEnvio \gets$ (0, NULL) \Comment $\Ogr(1)$
  \State conj(compu) $compus \gets$ Computadoras($red$) \Comment $\Ogr(1)$
   \State itConj $it \gets$ CrearIt($compus$) \Comment $\Ogr(1)$
  \State $proximaEnCamino \gets$ Vacio() \Comment $\Ogr(1)$
  \While{HaySiguiente?(it)} \Comment $\Ogr(1)$
    \State \tadNombre{itConj} $it2 \gets$ CrearIt($compus$) \Comment $\Ogr(1)$
    \State diccTrie(compu, puntero(compu)) $diccActual \gets$ Vacio() \Comment $\Ogr(1)$
	\State definir(paquetes, siguiente(it), tupla(vacio(), vacio(), vacio(), 0)) \Comment $\Ogr(log \ n)$
    \While{HaySiguiente?(it2)} \Comment $\Ogr(n)$
      \State conj(lista(compu)) $camMinimos \gets$ CaminosMinimos(red, SiguienteClave(it), SiguienteClave(it2)) \Comment $\Ogr(1)$
      \State itConj $it3 \gets$ CrearIt(camMinimos) \Comment $\Ogr(1)$
      \State $caminoMinimo \gets$ siguiente(it3)
      \State Fin(caminoMinimo)
      \State puntero(compu) $siguiente \gets$ \& Primero($caminoMinimo$)
      \State Definir($diccActual$, SiguienteClave(it2), id($siguiente$))
      \State Avanzar(it2)            
    \EndWhile
    \State definir($proximaEnCamino$, id(SiguienteClave($it$), $diccActual$)))
    \State Avanzar($it$)
  \EndWhile
   
  \EndProcedure
\end{algorithmic}
\end{algorithm}


\begin{algorithm}
\caption{Crear Paquete}
\begin{algorithmic}[1]
  \Procedure{CrearPaquete(\textbf{in/out} $s$ : dcnet, \textbf{in} $p$ : paquete)}{}
   \If {p.origen != p.destino}
   \State lista(compu) $nuevoCaminoRecorrido \gets$ Vacio()
   \State AgregarAtras(nuevoCaminoRecorrido, origen(p))
   \State itLista(lista(compu)) $it \gets$ AgregarAtras(caminosRecorridos, nuevoCaminoRecorrido)
   \State $losPaquetes \gets $ Obtener($paquetes$, p.origen)
   \State Encolar(losPaquetes.cola, $p$)
   \State Agregar(losPaquetes.conjPaquetes, $p$)
   \State Definir(diccPaqCamino, $p$, it)
   \EndIf
   \EndProcedure
\end{algorithmic}
\end{algorithm}


\begin{algorithm}
\caption{Avanzar Segundo}
\begin{algorithmic}[1]
  \Procedure{AvanzarSegundo(\textbf{in/out} $s$ : dcnet)}{}
  
  \State lista(tupla(paq, itLista(lista(id)))) $paquetesAEnviar \gets$ Vacio()
  
  \State itDiccTrie $it \gets$ CrearIt($paquetes$)
  \While{HaySiguientes?(it)} \Comment $\Ogr(n)$
  	\State infoCompu $s \gets$ SiguienteSignificado($it$) 
  	\If{$\neg$Vacía?($s$.$cola$)}
  		\State paquete $estePaquete \gets$ Desencolar($s$.$cola$)
  		\State Eliminar($s$.conjPaquetes, $estepaquete$)
		\State AgregarAtras(paquetesAEnviar(tupla($estePaquete$, definicion(s.diccPaqCamino, $estePaquete$))))
		\State Eliminar(s.diccPaqCamino, $estePaquete$) 
		\State cantEnviados++
		\If{cantEnviados > $s$.laQueMasEnvio.cuantosEnvio}
		   \State $laQueMasEnvio.cuantosEnvio \gets$ $s$.cantEnviados
		   \State $laQueMasEnvio.cualCompu \gets$ ultimo(Siguiente($pi_1$($estePaquete$)))
		\EndIf
	\EndIf
	\State Avanzar($it$)
  \EndWhile 
   
  \State itLista(tupla(paq, itLista(lista(compu)))) $it_2 \gets$ CrearIt($paquetesAEnviar$)
  \While{HaySiguientes?($it_2$)}
      \State tupla(paq, itLista(lista(compu)) $p \gets$ Siguiente($it_2$)
      \State compu $proximaCompu \gets$ Obtener(Obtener($proximaEnCamino$, $Ultimo(Siguiente(\pi_1(p)))$), destino($p$))
      \If {proximaCompu != $\pi_0(p)$}
	\State $paquetesDeProximaCompu \gets$ Obtener($paquetes$, proximaCompu)
      	\State AgregarAtras(Siguiente($\pi_1(p)$), proximaCompu)
      	\State Encolar($paquetesDeProximaCompu$.$cola$, $\pi_0(p)$)
      	\State Agregar($paquetesDeProximacompu$.$conjPaquetes$, $\pi_0$(p))
	\State Definir($paquetesDeProximacompu.diccPaqCamino$,$\pi_0$(p),$\pi_1$(p))      
      \EndIf
      \State Avanzar($it_2$)
  \EndWhile
 
 \EndProcedure
\end{algorithmic}
\end{algorithm}



\begin{algorithm}
\caption{Red}
\begin{algorithmic}[1]
 \Procedure{red(\textbf{in} $s$ : dcnet)}{}
 \State res $\gets$ red
 \EndProcedure
\end{algorithmic}
\end{algorithm}

\begin{algorithm}
\caption{Camino Recorrido}
\begin{algorithmic}[1]
  \Procedure{caminoRecorrido(\textbf{in} $s$ : dcnet, \textbf{in} $p$ : paquete) $\to res$ : secu(compu)}{}
   \State itDiccTrie $it$ $\gets$ CrearIt(paquetes)
   \While{$\neg$($p \in$ enEspera(siguiente(it)))} 
   \State avanzar(it)
   \EndWhile
   \State res $\gets$ siguiente(Obtener($\pi_1$(Obtener(paquetes,siguiente(it)),p))
  \EndProcedure
\end{algorithmic}
\end{algorithm}


\begin{algorithm}
\caption{Cantidad Enviados}
\begin{algorithmic}[1]
  \Procedure{cantidadEnviados(\textbf{in} $s$ : dcnet, \textbf{in} $c$ : compu)  $\to res$ : nat}{}
   \State res $\gets$ $\pi_3$(Obtener(paquetes, c))
  \EndProcedure
\end{algorithmic}
\end{algorithm}


\begin{algorithm}
\caption{Paquete En Transito}
\begin{algorithmic}[1]
  \Procedure{paqueteEnTransito(\textbf{in} $s$ : dcnet, \textbf{in} $p$ : paquete)  $\to res$ : bool}{}
   \State itDiccTrie $it$ $\gets$ CrearIt(paquetes)
   \While{HaySiguiente?(it) $\yluego$ $\neg$($p \in$ enEspera(siguiente(it)))} 
   \State avanzar(it)
	\EndWhile
	\If {HaySiguiente?(it)}
	\State res $\gets$   True
	\Else 
	\State res $\gets$ False
	\EndIf 
  \EndProcedure
\end{algorithmic}
\end{algorithm}




\begin{algorithm}
\caption{En Espera}
\begin{algorithmic}[1]
  \Procedure{enEspera(\textbf{in} $s$ : dcnet, \textbf{in} $c$ : compu)  $\to res$ : conj(compu)}{}
    \State res $\gets$ $\pi_2(Obtener(\pi_1(net),c))$ \Comment $\Ogr(1)+\Ogr(1)+\Ogr(L)+\Ogr(1)$
  \EndProcedure
  \underline{Complejidad:} $\Ogr(L)$
 \underline{Justificación:} $\Ogr(1)+\Ogr(1)+\Ogr(L)+\Ogr(1) = \Ogr(L)$

\end{algorithmic}
\end{algorithm}


\begin{algorithm}
\caption{La Que Más Envio}
\begin{algorithmic}[1]
  \Procedure{laQueMasEnvio(\textbf{in} $s$ : dcnet)  $\to res$ : compu}{}     
   \State res $\gets$   $\pi_1(\pi_3(net))$ \Comment $\Ogr(1)$
  \EndProcedure
  \underline{Complejidad:} $\Ogr(1)$
\end{algorithmic}
\end{algorithm}



\end{Algoritmos}



\clearpage

\end{document}
