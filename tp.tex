\documentclass[a4paper,10pt, nofootinbib]{article}
\usepackage[width=15.5cm, left=3cm, top=2.5cm, right=1cm, left=2cm, height= 24.5cm]{geometry}
\usepackage[spanish]{babel}
\usepackage[utf8]{inputenc}
\usepackage[T1]{fontenc}
\usepackage{xspace}
\usepackage{xargs}
\usepackage{ifthen}
\usepackage{caratula}
\usepackage{fancyhdr}
\usepackage{aed2-tad,aed2-symb,aed2-itef}
\usepackage[bottom]{footmisc}
\usepackage{modulos}
\usepackage{algorithm}
\usepackage[noend]{algpseudocode}

\newcommand{\moduloNombre}[1]{\textbf{#1}}


\let\NombreFuncion=\textsc
\let\TipoVariable=\texttt
\let\ModificadorArgumento=\textbf
\newcommand{\res}{$res$\xspace}
\newcommand{\tab}{\hspace*{7mm}}

\newcommand{\Ogr}{\mathcal{O}}


\newcommand{\footnoteMio}[2]{\textsuperscript{#1} \lfoot{\footnotesize \parbox{16cm}{\textsuperscript{#1}#2}}}

\newcommandx{\TipoFuncion}[3]{%
  \NombreFuncion{#1}(#2) \ifx#3\empty\else $\to$ \res\,: \TipoVariable{#3}\fi%
}
\newcommand{\In}[2]{\ModificadorArgumento{in} \ensuremath{#1}\,: \TipoVariable{#2}\xspace}
\newcommand{\Out}[2]{\ModificadorArgumento{out} \ensuremath{#1}\,: \TipoVariable{#2}\xspace}
\newcommand{\Inout}[2]{\ModificadorArgumento{in/out} \ensuremath{#1}\,: \TipoVariable{#2}\xspace}
\newcommand{\Aplicar}[2]{\NombreFuncion{#1}(#2)}

\newcommand{\Titulo}[1]{
  \vspace*{1ex}\par\noindent\textbf{\large #1}\par
}
\newcommand{\DRef}{\ensuremath{\rightarrow}}

%%Información para la carátula
\materia{Algoritmos y Estructuras de Datos II}

\titulo{\LARGE Trabajo Práctico Nº2}

\grupo{Grupo 18}

\integrante{Ciruelos Rodríguez, Gonzalo}{063/14}{gonzalo.ciruelos@gmail.com}
\integrante{Costa, Manuel José Joaquín}{035/14}{manuc94@hotmail.com}
\integrante{Gatti, Mathias Nicolás}{477/14}{mathigatti@gmail.com}
\integrante{Ginsberg, Mario Ezequiel}{145/14}{ezequielginsberg@gmail.com}

\def\cuatrimestre{1}

%%fancyhdr
\pagestyle{fancy}
\thispagestyle{fancy}
\addtolength{\headheight}{1pt}
\lhead{Algoritmos y Estructuras de Datos II: TP1}
\rhead{$1^{\mathrm{er}}$ cuatrimestre de 2015}
\cfoot{\thepage\ / 11}
\renewcommand{\footrulewidth}{0.4pt}



\setlength{\parskip}{0.8em}


\begin{document}
\maketitle
\thispagestyle{empty}

\section{Aclaraciones}
\begin{enumerate}
  \item Dado que consultando con algunos JTP nos sugirieron que en las funciones de DCNet en las que devolvemos computadoras, nos manejemos solamente con sus ips (es decir, no devolvamos una tupla \texttt{(ip, conj(interfaz))} sino simplemente una \texttt{ip}). La información de las interfaces obviamente no se 'perdió', sigue estando en la red, sólo que no se está manipulando y devolviendo todo el tiempo.
   
    Además nos dijeron qué cambios deberíamos hacerle al TAD DCNet para que sea consistente con estos cambios. Los cambios que deberían hacerse son simples, solo debería devolverse la ip de las computadoras, por ejemplo, en \texttt{recorridoPaquete}, en la que se devuelve una secuencia de computadoras, debería devolverse una secuencia de ip.
    
    Para hacer la lectura más fácil, decidimos usar intercambiablemente las palabras \texttt{compu} e \texttt{id}, y ambas deben ser intepretadas como el string hostname de la computadora (su ip).
    
    \item Al trabajar con $p$:\texttt{puntero($tupla(campo_1$, $\dots$, $campo_n$))}, usamos la notación $p\to campo_i$ que equivale a decir $(*p).campo_i$. 
    \item Al calcular complejidades de operaciones que tienen condicionales, siempre hacemos la rama que tiene mayor complejidad, pues lo que nos interesa es la complejidad en el peor caso. 
	\item Cuando ingresamos un paquete, éste ya viene con su id, en lugar de generarlo nosotros (como sí pasaba en un ejercicio visto en clase). Esta es una decisión que tomamos para respetar lo más posible el tad. 

\end{enumerate}

\clearpage


\section{Módulo Red}
\begin{Interfaz}
  
  \textbf{se explica con}: \tadNombre{Red}.

  
  \section*{Operaciones básicas de \tadNombre{Red}}

  \InterfazFuncion{IniciarRed}{}{red}
  {$res \igobs iniciarRed()$}
  [$\Ogr(1)$]
  [genera la red vacía]

  \InterfazFuncion{AgregarComputadora}{\Inout{r}{\tadNombre{Red}}, \In{c}{\tadNombre{Compu}}, \In{is}{\tadNombre{Compu}}}{}
  [$r \igobs r_0 \land (\forall c' : compu) (c' \in computadoras(r) \impluego ip(c) \neq ip(c'))$] 
  {$computadoras(r) = Ag(c, computadoras(r_0)) \yluego \\
     (\forall c_1, c_2 : compu) (c_1 \in computadoras(r_0) \land c_2 \in computadoras(r_0) \impluego \\ 
   conectadas(r_0, c_1, c_2) = conectadas(r, c_1, c_2) \land interfazUsada(r_0, c_1, c_2) = interfazUsada(r, c_1, c_2)$)}
  [$\Ogr(L)$]
  [agrega el la computadora $c$ a la red $r$.]
  [el elemento $c$ agrega por copia.]
  
  
  \InterfazFuncion{conectar}{\Inout{r}{\tadNombre{Red}}, \In{c_1}{\tadNombre{Compu}},  \In{i_1}{\tadNombre{Interfaz}}, \In{c_2}{\tadNombre{Interfaz}}, \In{i_2}{\tadNombre{Interfaz}}}{}
 [$r \igobs r_0 \land c_1 \in computadoras(r) \land c_2 \in computadoras(r) \land ip(c_1) \neq ip(c_2) \land \neg conectadas?(r, c_1, c_2) \land \neg usaInterfaz?(r,c_1, i_1) \land \neg usaInterfaz?(r, c_2, i_2)$]
  {$computadoras(r) = computadoras(r_0) \yluego \\
    (\forall c'_1, c'_2 : compu) (c'_1 \in computadoras(r_0) \land c'_2 \in computadoras(r_0) \land \{c_1, c_2\} \neq \{c'_1, c'_2\} \impluego \\ 
   conectadas(r_0, c'_1, c'_2) = conectadas(r, c'_1, c'_2) \land interfazUsada(r_0, c'_1, c'_2) = interfazUsada(r, c'_1, c'_2)) \land \\
   conectadas(r_0, c_1, c_2) \land interfazUsada(r, c_1, c_2) = i_1 \land interfazUsada(r, c_2, c_1) = i_2$}
  [$\Ogr(L)$]
  [conecta las computadoras $c_1$ y $c_2$ mediante las interfaces $i_1$ e $i_2$ respectivamente en la red $r$.]


  \InterfazFuncion{computadoras}{\In{r}{\tadNombre{Red}}}{\tadNombre{conj(compu)}}
  {$res = computadoras(r)$}
  [$\Ogr(n)$]
  [devuelve un conjunto con todas las computadoras de la red $r$.]
  [?]


  \InterfazFuncion{conectadas?}{\In{r}{\tadNombre{Red}}, \In{c_1}{\tadNombre{Compu}}, \In{c_2}{\tadNombre{Compu}}}{\tadNombre{bool}}
  [$c_1 \in computadoras(r) \land c_2 \in computadoras(r)$]
  {$res = conectadas?(r,c_1,c_2)$}
  [$\Ogr(L)$]
  [dice si las computadoras $c_1$ y $c_2$ están conectadas en la red $r$]


  \InterfazFuncion{interfazUsada}{\In{r}{\tadNombre{Red}}, \In{c_1}{\tadNombre{Compu}}, \In{c_2}{\tadNombre{Compu}}}{\tadNombre{bool}}
  [$conectadas?(r,c_1,c_2)$]
  {$res = interfazUsada(r,c_1,c_2)$}
  [$\Ogr(L)$]
  [devuelve la interfaz mediante la cual $c_1$ se conecta a $c_2$  en la red $r$.]


  \InterfazFuncion{vecinos}{\In{r}{\tadNombre{Red}}, \In{c}{\tadNombre{Compu}}}{\tadNombre{conj(compu)}}
  [$c \in computadoras(r)$]
  {$res = vecinos(r,c)$}
  [$\Ogr(n+L)$]
  [devuelve las computadoras conectadas a $c$ en la red $r$.]
  [?]


  \InterfazFuncion{usaInterfaz?}{\In{r}{\tadNombre{Red}}, \In{c}{\tadNombre{Compu}}, \In{i}{\tadNombre{Interfaz}}}{\tadNombre{bool}}
  [$c \in computadoras(r)$]
  {$res = usaInterfaz?(r,c,i)$}
  [$\Ogr(nL)$]
  [devuelve true si y solo si $c$ está conectada a otra pc mediante la interfaz $i$ en la red $r$.]


  \InterfazFuncion{caminosMinimos}{\In{r}{\tadNombre{Red}}, \In{c_1}{\tadNombre{Compu}}, \In{c_2}{\tadNombre{Compu}}}{\tadNombre{conj(secu(compu))}}
  [$c_1 \in computadoras(r) \land c_2 \in computadoras(r)$]
  {$res = caminosMinimos(r,c_1,c_2)$}
  [$\Ogr(n \cdot n^2!)$]
  [devuelve un conjunto con los caminos más cortos entre las computadoras $c_1$ y $c_2$ en la red $r$.]

%%

  \InterfazFuncion{hayCamino?}{\In{r}{\tadNombre{Red}}, \In{c_1}{\tadNombre{Compu}}, \In{c_2}{\tadNombre{Compu}}}{\tadNombre{bool}}
  [$c_1 \in computadoras(r) \land c_2 \in computadoras(r)$]
  {$res = hayCamino?(r,c_1,c_2)$} 
  [$\Ogr(n \cdot n^2!)$]
  [devuelve true si y solo si hay un camino entre las computadoras $c_1$ y $c_2$ en la red $r$.]



\end{Interfaz}

\begin{Representacion}
  
  
  
  \section*{Representación de la red}

  \begin{Estructura}{red}[redEstr]
    \begin{Tupla}[topo]
      \tupItem{vecinos}{dicc(compu, dicc(compu, interfaz))}\\
      \tupItem{interfaces}{dicc(compu, conj(interfaces))}%
    \end{Tupla}

  \end{Estructura}

  \Rep[redEstr][e]{claves(e.vecinos) $=$ claves(e.compus) $\yluego$ \\
  ($\forall i_1, i_2$ : id) def?($i_2$, (obtener($i_1$, e.vecinos))) $\impluego$ def?($i_2$, e.vecinos) $\yluego$ def?($i_1$, obtener($i_2$, e.vecinos)) $\yluego$ \\
obtener($i_2$, obtener($i_1$, e.vecinos)) $\in$  interfaces(obtener($i_1$, e.compus))
$\land$\\ 
($\forall i, i_1, i_2$ : id) def?($i$, e.vecinos) $\impluego$ \\
(def?($i_1$, obtener($i$, e.vecinos)) $\land$ def?($i_2$, obtener($i$, e.vecinos)) $\land$ $i_1 \neq i_2$ $\impluego$ obtener($i_1$, obtener($i$, e.vecinos)) $\neq$ obtener($i_2$, obtener($i$, e.vecinos)))

}\mbox{}
  
   
   \AbsFc[redEstr]{red}[e]{
     r : red / computadoras(r) = significados(e.compus) $\land$ \\ 
     ($\forall c_1, c_2$ : computadora) conectadas($r, c_1, c_2$) = def?(id($c_1$), obtener(id($c_2$), e.vecinos)) $\land$ \\
     ($\forall c_1, c_2$: computadora) ($\forall i$ : interfaz) conectadas($r,c_1,c_2$) $\impluego$ interfazUsada($r, c_1, c_2$) = obtener(id($c_1$), obtener(id($c_2$), e.vecinos))
  }

\end{Representacion}



\begin{Algoritmos}


\begin{algorithm}
  \caption{Algoritmos de \tadNombre{RED}}
\begin{algorithmic}[1]
  \Procedure{iIniciarRed}{\ } $\to$ $res$ : \texttt{redEstr}
  \State vecinos $\gets$ Vacia() \Comment $\Ogr(1)$
  \State interfaces $\gets$ Vacia() \Comment $\Ogr(1)$
  \State $res \gets$ (vecinos, interfaces) \Comment $\Ogr(1)$
  \EndProcedure
  \underline{Complejidad:} $\Ogr(1)$
\end{algorithmic}
\end{algorithm}


\begin{algorithm}
\caption{Agregar Computadora}
\begin{algorithmic}[1]
  \Procedure{iAgregarComputadora}{\texttt{in/out} $r$ : \texttt{redEstr}, \texttt{in} $c$ : \texttt{compu}}
  \State definir(r.vecinos, id($c$), Vacia()) \Comment $\Ogr(L)$
  \State definir(r.interfaces, id($c$), $c$) \Comment $\Ogr(L)$
 \EndProcedure
 \underline{Complejidad:} $\Ogr(L)$
\end{algorithmic}
\end{algorithm}


\begin{algorithm}
\caption{Conectar}
\begin{algorithmic}[1]
  \Procedure{iConectar}{\texttt{in/out} $r$ : \texttt{redEstr}, \texttt{in} $c_1$ : \texttt{compu}, \texttt{in} $c_2$ :  \texttt{compu}, \texttt{in} $i_1$ : \texttt{interfaz}, \texttt{in} $i_2$ : \texttt{interfaz}}{}
  \State $x \gets$ obtener(r.vecinos, $c_1$) \Comment $\Ogr(L)$
  \State definir($x$, id($c_2$), $i_1$) \Comment $\Ogr(L)$
  \State $x \gets$ obtener(r.vecinos, $c_2$) \Comment $\Ogr(L)$
  \State definir($x$, id($c_1$), $i_2$) \Comment $\Ogr(L)$
 \EndProcedure
 \underline{Complejidad:} $\Ogr(L)$
\end{algorithmic}
\end{algorithm}



\begin{algorithm}
\caption{Computadoras}
\begin{algorithmic}[1]
  \Procedure{iComputadoras}{\texttt{in} $r$ : \texttt{redEstr}} $\to res$ : \texttt{conj(compu)} 
  \State $res \gets$ Claves($interfaces$) \Comment $\Ogr(n)$
 \EndProcedure
\underline{Complejidad:} $\Ogr(n)$
\end{algorithmic}
\end{algorithm}


\begin{algorithm}
\caption{Conectadas?}
\begin{algorithmic}[1]
  \Procedure{iConectadas?}{\texttt{in} $r$ : \texttt{redEstr}, \texttt{in} $c_1$ : \texttt{compu}, \texttt{in} $c_2$ : \texttt{compu}}  $\to res$ : \texttt{bool}
  \State $x \gets$ Obtener(id($c_1$), $r$.vecinos) \Comment $\Ogr(L)$
  \State $res \gets$ def?(id($c_2$), $x$) \Comment $\Ogr(L)$
 \EndProcedure
 \underline{Complejidad:} $\Ogr(L)$
\end{algorithmic}
\end{algorithm}


\begin{algorithm}
\caption{Interfaz Usada}
\begin{algorithmic}[1]
  \Procedure{iInterfazUsada}{\texttt{in} $r$ : \texttt{redEstr}, \texttt{in} $c_1$ : \texttt{compu}, \texttt{in} $c_2$ : \texttt{compu}} $\to res$ : \texttt{interfaz}
  \State $x \gets$ Obtener(id($c_1$), $r$.vecinos) \Comment $\Ogr(L)$
  \State $res \gets$ Obtener(id($c_2$), $x$) \Comment $\Ogr(L)$
 \EndProcedure
 \underline{Complejidad:} $\Ogr(L)$
\end{algorithmic}
\end{algorithm}


\begin{algorithm}
\caption{Vecinos}
\begin{algorithmic}[1]
  \Procedure{iVecinos}{\texttt{in} $r$ : \texttt{redEstr}, \texttt{in} $c$ :\texttt{compu}}  $\to res$ : \texttt{conj(compu)}
  \State $x \gets$ Obtener(id($c$), $r$.vecinos) \Comment $\Ogr(L)$
  \State $res \gets$ Claves($x$) \Comment $\Ogr(n)$
 \EndProcedure
 \underline{Complejidad:} $\Ogr(n+L)$
\end{algorithmic}
\end{algorithm}



\begin{algorithm}
\caption{Usa Interfaz?}
\begin{algorithmic}[1]
  \Procedure{iUsaInterfaz?}{\texttt{in} $r$ : \texttt{redEstr}, \texttt{in} $c$ : \texttt{compu},  \texttt{in} $i$ : interfaz} $\to res$ : \texttt{bool} 
  \State $x \gets$ Obtener(id($c$), $r$.vecinos) \Comment $\Ogr(L)$
  \State $i :\tadNombre{itdicc} \gets$ NuevoIterador($x$) \Comment $\Ogr(n)$
  \State $res \gets$ False \Comment $\Ogr(1)$
  \While{haySiguiente($i$)} \Comment Se repite $\Ogr(n)$ veces
    \If {Obtener(Siguiente($i$), $x$) = $i$}  \Comment $\Ogr(L)$
       \State $res \gets$ True  \Comment $\Ogr(1)$
    \EndIf
    \State Avanzar($i$) \Comment $\Ogr(1)$
  \EndWhile
 \EndProcedure
 \underline{Complejidad:} $\Ogr(nL)$
 \underline{Justificación:} $\Ogr(L) + \Ogr(n) + \Ogr(1) + \Ogr(n)(\Ogr(L) + \Ogr(1) = \Ogr(n)\Ogr(L) = \Ogr(nL)$
\end{algorithmic}
\end{algorithm}


\begin{algorithm}
\caption{Caminos Minimos}
\begin{algorithmic}[1]
  \Procedure{iCaminosMinimos}{\texttt{in} $r$ : \texttt{redEstr}, \textbf{in} $c_1$ : \texttt{compu},  \texttt{in} $c_2$ : \texttt{compu}} $\to res$ : \texttt{conj(secu(compu))}  
   \State $k \gets$ 1 \Comment $\Ogr(1)$
   \State $n \gets$ Cardinal(Computadoras($r$))  \Comment $\Ogr(n)$
   \While{$k<n$ $\land$ esVacio?(CaminosDeLargoN($r$, $c_1$, $c_2$, k))} \Comment Se repite $\Ogr(n)$ veces y cuesta  $\Ogr(n^2!)$
     \State k++ \Comment $\Ogr(1)$
   \EndWhile
   \State $res \gets$ CaminosDeLargoN($r$, $c_1$, $c_2$, k) \Comment $\Ogr(n^2!)$
\EndProcedure
\underline{Complejidad:} $\Ogr(n \cdot n^2!)$
\underline{Justificacion:} $\Ogr(1) + \Ogr(n) + \Ogr(n)\Ogr(n^2!) + \Ogr(n^2!) = \Ogr(n)\Ogr(n^2!) = \Ogr(n \cdot n^2!)$
\end{algorithmic}
\end{algorithm}


\begin{algorithm}
\caption{Hay Camino?}
\begin{algorithmic}[1]
  \Procedure{iHayCamino?}{\texttt{in} $r$ : \texttt{redEstr}, \texttt{in} $c_1$ : \texttt{compu},  \texttt{in} $c_2$ : \texttt{compu}} $\to res$ : \texttt{bool} 
   \State $res \gets$ $\neg$ esVacio?(CaminosMinimos($r$,$c_1$, $c_2$)) \Comment  $\Ogr(n \cdot n^2!)$
\EndProcedure
\underline{Complejidad:} $\Ogr(n \cdot n^2!)$
\end{algorithmic}
\end{algorithm}

\begin{algorithm}
\caption{Caminos De Largo N }
\begin{algorithmic}[1]
  \Procedure{iCaminosDeLargoN}{\texttt{in} $r$ : \texttt{redEstr}, \textbf{in} $c_1$ : \texttt{compu},  \texttt{in} $c_2$ : \texttt{compu}, \texttt{in} $n$ : \texttt{nat}} $\to res$ : \texttt{conj(secu(compu))} 
   \State conj(secu(compu)) $caminos \gets$ Vacio() \Comment $\Ogr(1)$
   \If{n = 0}
     \State secu(compus) $camino \gets$ Vacio \Comment $\Ogr(1)$
     \State AgregarAtras($camino$, $c_1$) \Comment $\Ogr(1)$
     \State Agregar($camino$, $caminos$) \Comment $\Ogr(1)$
   \Else
     \State $vec \gets$ vecinos($c_1$) \Comment $\Ogr(n+L)$
     \State $itVecinos \gets$ CrearIt($vec$) \Comment $\Ogr(1)$
     \While{haySiguiente?($itVecinos$)} \Comment Se va a repetir $\Ogr(n)$ veces
       \State $v \gets$ Siguiente($itVecinos$) \Comment $\Ogr(1)$
       \State $cams \gets$ CaminosDeLargoN($r$, $v$, $c_2$, $n-1$) \Comment $T(n-1)$
       \State $itCaminos \gets$ CreatIt($cams$) \Comment $\Ogr(1)$
       \While{haySiguiente?($itCaminos$)}  \Comment Se va a repetir $\Ogr(|caminos|) \subseteq \Ogr(n^2!)$ veces \footnotesize{Si el grafo de computadoras fuera completo (alcanzando asi la maxima cantidad de links posibles, que es $\frac{n(n-1)}{2})$ es claro que se puede elegir $\frac{n(n-1)}{2}!$ caminos distintos, por combinatoria} \normalsize
         \State $camino \gets$ Siguiente($itCaminos$) \Comment $\Ogr(1)$
         \If{Ultimo($camino$) = $c_2$} \Comment $\Ogr(1)$
           \State $camino_2 \gets$ camino \Comment $\Ogr(1)$
           \State AgregarAdelante($c_1$, $camino_2$) \Comment $\Ogr(1)$
           \State Agregar ($camino_2$, $caminos$) \Comment $\Ogr(1)$
         \EndIf
       \State Avanzar($itCaminos$) \Comment $\Ogr(1)$
       \EndWhile
    \State Avanzar($itVecinos$) \Comment $\Ogr(1)$
    \EndWhile    
   \EndIf
   
   \State $itCaminos \gets$ crearIt($caminos$) \Comment $\Ogr(1)$
   \While{haySiguiente?($itCaminos$)} \Comment Se va a repetir $\Ogr(n)$ veces
     \If{ultimo(Siguiente($itCaminos$)) $\neq$ $c_2$} \Comment $\Ogr(1)$
       \State EliminarSiguiente($itCaminos$) \Comment $\Ogr(1)$
      \EndIf
    \EndWhile
    \State $res \gets caminos$ \Comment $\Ogr(1)$

 \EndProcedure
 \underline{Complejidad:} $\Ogr(n^2!)$
 \underline{Justificación:} $T(0) = \Ogr(1), T(n) = \Ogr(n+L) + \Ogr(n) \cdot T(n-1) + \Ogr(n^2!) + \Ogr(n)$ \hspace{3cm}  Por lo tanto, $T(n) =  \Ogr(n)T(n-1) + \Ogr(n^2!) = \Ogr(n)(\Ogr(n-1)T(n-1) + \Ogr((n-1)^2!)) + \Ogr(n^2!)) = \Ogr(n)\Ogr(n-1) T(n-2) + \Ogr((n-1)^2!)\Ogr(n) + \Ogr(n^2!) = \Ogr(n)\Ogr(n-1) T(n-2) + \Ogr(n^2!) = ... = \Ogr(n)\cdot ... \cdot \Ogr(1) + \Ogr(n^2!) = \Ogr(n!) + \Ogr(n^2!) = \Ogr(n^2!)$
\end{algorithmic}
\end{algorithm}


\end{Algoritmos}








\clearpage

\section{Módulo DiccTrie}
\begin{Interfaz}
  \textbf{parámetros formales}\hangindent=2\parindent\\
  \parbox{1.7cm}{\textbf{géneros}} $\sigma$\\
  \parbox[t]{1.7cm}{\textbf{función}}\parbox[t]{\textwidth-2\parindent-1.7cm}{    	
    \InterfazFuncion{Copiar}{\In{s}{$\sigma$}}{$\sigma$}
    {$res \igobs s$}
    [$\Theta(copy(s))$]
    [función de copia de $\sigma$]
    }
      
  \textbf{se explica con}: \tadNombre{diccionario($string, \sigma$)}.

  \textbf{géneros}: \TipoVariable{diccTrie($\sigma$), itClavesDiccTrie($\sigma$)}.

  \section*{Operaciones básicas DiccTrie($\sigma$)}

  \InterfazFuncion{Vacío}{}{diccTrie($\sigma$)}
  {$res$ \igobs vacío()}
  [$\Theta(1)$]
  [genera un diccionario vacío.]

  \InterfazFuncion{Definir}{\Inout{d}{diccTrie($\sigma$)}, \In{k}{$string$}, \In{s}{$\sigma$}}{}
  [$d \igobs d_0$]  
  {$d$ \igobs definir($k, s, d_0$)}
  [$\Theta(|k| + copy(s))$]
  [define la clave $k$ con el sinificado $s$.]
  [$k$ y $s$ se definen por copia.]
    
  \InterfazFuncion{Definido?}{\In{d}{diccTrie($\sigma$)}, \In{k}{$string$}}{bool}
  {$res$ \igobs def?($k$, $d$) }
  [$\Theta(|k|)$]
  [devuelve $true$ si la clave $k$ está definida en el diccionario.]
  
  \InterfazFuncion{Obtener}{\In{d}{diccTrie($\sigma$)}, \In{k}{$string$}}{$\sigma$}
  [def?($k$, $d$)]  
  {alias($res$ \igobs obtener($k$, $d$))}
  [$\Theta(|k|)$,]
  [devuelve el significado de la clave $k$ en $d$.]
  [$res$ es modificable si y sólo si $d$ lo es]
  
  \InterfazFuncion{Claves}{\In{d}{diccTrie($\sigma$)}}{conj(string)}
  {$res$ \igobs claves($d$)}
  [$\Theta()$]
  [devuelve un conjunto con todas las claves del diccionario.] 
  
  \InterfazFuncion{Borrar}{\In{d}{diccTrie($\sigma$)}, \In{k}{$string$}}{}
  [$d$ \igobs $d_0 \land$ def?($k$, $d_0$) ]
  {$d \igobs$ borrar($k$, $d_0$)}
  [$\Theta(|k|)$]
  [elimina la entrada $k$ del diccionario.] 
  
  \section*{Operaciones del iterador}

  \InterfazFuncion{CrearIt}{\In{d}{diccTrie($\sigma$)}}{itDiccTrie($\sigma$)}
  {$res$ $\igobs$ crearItBi(\secuencia{}, $l$) $\land$ alias(SecuSuby($it$) $=$ $l$)}
  [$\Theta(1)$]
  [crea un iterador bidireccional de la lista, de forma tal que al pedir \NombreFuncion{Siguiente} se obtenga el primer elemento de $l$.]
  [el iterador se invalida si y sólo si se elimina el elemento siguiente del iterador sin utilizar la función \NombreFuncion{EliminarSiguiente}.]
  
  \InterfazFuncion{CrearItUlt}{\In{l}{lista($\alpha$)}}{itLista($\alpha$)}
  {$res$ $\igobs$ crearItBi($l$, \secuencia{}) $\land$ alias(SecuSuby($it$) $=$ $l$)}
  [$\Theta(1)$]
  [crea un iterador bidireccional de la lista, de forma tal que al pedir \NombreFuncion{Anterior} se obtenga el último elemento de $l$.]  
  [el iterador se invalida si y sólo si se elimina el elemento siguiente del iterador sin utilizar la función \NombreFuncion{EliminarSiguiente}.]

\end{Interfaz}

\begin{Representacion}
  
  \section*{Representación de DiccTrie($\sigma$)}

  \begin{Estructura}{diccTrie($\sigma$)}[estr]
    \begin{Tupla}[estr]
      \tupItem{primero}{puntero(nodo)}
      \tupItem{claves}{conj(tupla(string, puntero($\sigma$)))}
    \end{Tupla}

    \begin{Tupla}[nodo]
      \tupItem{caracteres}{arreglo\_estático[256] de puntero(nodo)}
      \tupItem{significado}{puntero($\sigma$)}
    \end{Tupla}
  \end{Estructura}

  \Rep[estr][e]{}\mbox{}


  \tadOperacion{Nodo}{lst/l,nat}{puntero(nodo)}{$l$.primero $\neq$ NULL}
  \tadAxioma{Nodo($l$,$i$)}{\IF $i = 0$ THEN $l$.primero ELSE Nodo(FinLst($l$), $i-1$) FI}

  \AbsFc[estr]{dicc($string, \sigma$)}[e]{}

  \section*{Representación del iterador}

  \begin{Estructura}{itClavesDiccTrie($\sigma$)}[iter]
    \begin{Tupla}[iter]
      \tupItem{iterador}{itConj(tupla(string, puntero($\sigma$)))}
    \end{Tupla}
  \end{Estructura}

  \Rep[iter][it]{Rep($\ast$($it$.lista)) $\yluego$ ($it$.siguiente $=$ NULL $\oluego$ ($\exists i$: nat)(Nodo($\ast it$.lista, $i$) $=$ $it$.siguiente)}

  ~

  \Abs[iter]{itBi($\alpha$)}[it]{b}{Siguientes($b$) $=$ Abs(Sig($it$.lista, $it$.siguiente)) $\land$\\
    Anteriores($b$) $=$ Abs(Ant($it$.lista, $it$.siguiente))}

  ~

  \tadOperacion{Sig}{puntero(lst)/l,puntero(nodo)/p}{lst}{Rep($\langle l, p\rangle$)}
  \tadAxioma{Sig($i, p$)}{Lst($p$, $l$\DRef longitud $-$ Pos($\ast l$, $p$))}

  ~

  \tadOperacion{Ant}{puntero(lst)/l,puntero(nodo)/p}{lst}{Rep($\langle l, p\rangle$)}
  \tadAxioma{Ant($i, p$)}{Lst(\IF $p$ $=$ $l$\DRef primero THEN NULL ELSE $l$\DRef primero FI, Pos($\ast l$, $p$))}

  ~

  {\small Nota: cuando $p$ $=$ NULL, Pos devuelve la longitud de la lista, lo cual está bien, porque significa que el iterador no tiene siguiente.}
  \tadOperacion{Pos}{lst/l,puntero(nodo)/p}{puntero(nodo)}{Rep($\langle l, p\rangle$)}
  \tadAxioma{Pos($l$,$p$)}{\IF $l$.primero $=$ $p$ $\lor$ $l$.longitud $=$ $0$ THEN $0$ ELSE $1$ $+$ Pos(FinLst($l$), $p$) FI}


\end{Representacion}
\clearpage

\section{Módulo DiccAVL($\kappa, \sigma$)}
\begin{Interfaz}
  \textbf{parámetros formales}\hangindent=2\parindent\\
  \parbox{1.7cm}{\textbf{géneros}} $\kappa, \sigma$\\
  \parbox[t]{1.7cm}{\textbf{función}}\parbox[t]{\textwidth-2\parindent-1.7cm}{    	
    \InterfazFuncion{$\bullet <_{\kappa} \bullet$}{\In{k}{$\kappa$}, \In{k'}{$\kappa$}}{$bool$}
    {$res \igobs (k <_{\kappa} k')$}
    [$\Ogr(comp_{\kappa}(k, k'))$]
    [relación de orden total de $\kappa$]
    }
        
   \parbox[t]{1.7cm}{\textbf{función}}\parbox[t]{\textwidth-2\parindent-1.7cm}{    	
    \InterfazFuncion{$\bullet =_{\kappa} \bullet$}{\In{k}{$\kappa$}, \In{k'}{$\kappa$}}{$bool$}
    {$res \igobs (k =_{\kappa} k')$}
    [$\Ogr(equal_{\kappa}(k, k'))$]
    [igualdad de $\kappa$]
    }
    
  \textbf{se explica con}: \tadNombre{diccionario($\kappa, \sigma$)}.

  \textbf{géneros}: \TipoVariable{diccAVL($\kappa, \sigma$)}.

  \section*{Operaciones básicas DiccAVL($\kappa, \sigma$)}

  \InterfazFuncion{Vacío}{}{diccAVL($\kappa, \sigma$)}
  {$res$ \igobs vacío()}
  [$\Ogr(1)$]
  [genera un diccionario vacío.]

  \InterfazFuncion{Definir}{\Inout{d}{diccAVL($\kappa, \sigma$)}, \In{k}{$\kappa$}, \In{s}{$\sigma$}}{}
  [$d \igobs d_0$]  
  {$d$ \igobs definir($k, s, d_0$)}
  [$\Ogr(log(n) \times comp_{\kappa}(k, k'))$\footnote{Dentro de este módulo, $n$ es la cantidad de entradas del diccionario.}]
  [define la clave $k$ con el sinificado $s$.]
  [$k$ y $s$ se definen por referencia.]
    
  \InterfazFuncion{Definido?}{\In{d}{diccAVL($\kappa, \sigma$)}, \In{k}{$\kappa$}}{bool}
  {$res$ \igobs def?($k$, $d$) }
  [$\Ogr(log(n) \times comp_{\kappa}(k, k'))$]
  [devuelve $true$ si la clave $k$ está definida en el diccionario.]
  
  \InterfazFuncion{Obtener}{\In{d}{diccAVL($\kappa, \sigma$)}, \In{k}{$\kappa$}}{$\sigma$}
  [def?($k$, $d$)]  
  {alias($res$ \igobs obtener($k$, $d$))}
  [$\Ogr(log(n) \times (equal_{\kappa}(k, k') + comp_{\kappa}(k, k')) + equal_{\kappa}(k, k'))$]
  [devuelve el significado de la clave $k$ en $d$.]
  [$res$ es modificable si y sólo si $d$ lo es]
  
  \InterfazFuncion{Borrar}{\In{d}{diccAVL($\kappa, \sigma$)}, \In{k}{$\kappa$}}{}
  [$d$ \igobs $d_0 \land$ def?($k$, $d_0$) ]
  {$d \igobs$ borrar($k$, $d_0$)}
  [$\Ogr(log(n) \times comp_{\kappa}(k, k'))$]
  [elimina la entrada $k$ del diccionario.] 
  
  \underline{Nota:} en el caso en que $equal_{\kappa}(k, k')$ y $comp_{\kappa}(k, k')$ sean $\Ogr(1)$, todas las funciones del diccionario, salvo Vacio, son $\Ogr(log(n))$
  \clearpage
\end{Interfaz}

\begin{Representacion}
    
  \section*{Representación de DiccAVL($\kappa, \sigma$)}

  \begin{Estructura}{diccAVL($\kappa, \sigma$)}[puntero(puntero(nodo))]
    \begin{Tupla}[nodo]
      \tupItem{clave}{$\kappa$}
      \tupItem{\\significado}{$\sigma$}
      \tupItem{\\izq}{puntero(nodo)}
      \tupItem{\\der}{puntero(nodo)}
      \tupItem{\\altura}{nat}
    \end{Tupla}

  \end{Estructura}

  \Rep[puntero(puntero(nodo))][a]{
  					   $a \neq $NULL \yluego \\
  					   \hspace*{4em}$*a$ = NULL \oluego \\
  					   \hspace*{4em}($\exists n:$nat) finaliza($*a$, $n$) \yluego \\
  					   \hspace*{4em}$(*a)\rightarrow altura$ = alto($*a$) $\land$ \\
  					   \hspace*{4em}diferenciaDeAlturas($(*a)\rightarrow izq$, $(*a)\rightarrow der$)$\leq$ 1 $\land$ \\
  					   \hspace*{4em}($\forall c:\kappa$) ($c\in$ ClavesDelAvl($(*a)\rightarrow izq$) $\Rightarrow (*a)\rightarrow clave < c)$ $\land $\\
  					   \hspace*{4em}($\forall c:\kappa$) ($c\in$ ClavesDelAvl($(*a)\rightarrow der$) $\Rightarrow (*a)\rightarrow clave > c)$ $\land $\\
  					   \hspace*{4em}Rep($(*a) \rightarrow izq$) $\land$ Rep($(*a) \rightarrow der$)}
  					   
  
  \AbsFc[puntero(puntero(nodo))]{diccAVL($\kappa, \sigma$)}[a]{
	$d$:dicc($\kappa, \sigma$) |
	($\forall c:\kappa$) (def?($c, d$) $\Leftrightarrow$ $c \in$ clavesDelAvl($*a$)) \yluego\\
	\hspace*{5.9em}($\forall c:\kappa$) (def?($c, d$) \impluego obtener($c, d$) = darSignificado($*a, c$))
  }  
  
  \tadOperacion{finaliza}{puntero(nodo), nat}{bool}{}
	\tadAxioma{finaliza($p,n$)}{$n > 0$ \yluego ($p$ = NULL \oluego (finaliza($p\rightarrow izq$, $n-1$) $\land$ finaliza($p\rightarrow der$, $n-1$)))}

  \tadOperacion{alto}{puntero(nodo)}{nat}{}
	\tadAxioma{alto($p$)}{\IF $p = $NULL THEN 0 ELSE $1 + $max(alto($p\rightarrow izq$), alto($p\rightarrow der$))FI}
 
  \tadOperacion{diferenciaDeAlturas}{puntero(nodo) , puntero(nodo)}{nat}{}
	\tadAxioma{diferenciaDeAlturas($p, p'$)}{\IF $p\rightarrow altura \leq p'\rightarrow altura$
											THEN $(p'\rightarrow altura) - (p\rightarrow altura)$
											ELSE $(p\rightarrow altura) - (p'\rightarrow altura)$
											FI} 
 
  \tadOperacion{clavesDelAvl}{puntero(nodo)}{conj($\kappa$)}{}
	\tadAxioma{clavesDelAvl($p$)}{\IF $p = $NULL THEN $\emptyset$ ELSE Ag($p\rightarrow clave$, clavesDelAvl($p\rightarrow izq$) $\cup$ clavesDelAvl($p\rightarrow der$))FI} 
    
   
  \tadOperacion{darSignificado}{puntero(nodo)/p , $\kappa$/c}{$\sigma$}{$c \in$ clavesDelAvl($p$)}
	\tadAxioma{darSignificado($p, c$)}{\IF $p\rightarrow clave$ = $c$ THEN $p\rightarrow significado$ ELSE 
											{\IF $c\in$ clavesDelAvl($p\rightarrow izq$)
											THEN darSignificado($p\rightarrow izq$, $c$)
											ELSE darSignificado($p\rightarrow der$, $c$)
											FI}
										FI} 

\end{Representacion}

\begin{Algoritmos}

\begin{algorithm}
\caption{actualizarAltura}
\begin{algorithmic}[1]
\Procedure{actualizarAltura}{\texttt{in/out} $p$ : \texttt{puntero(nodo)}}
	\If{$p\neq$ NULL}
	\Comment $\Ogr(1)$
    \State $(p\to altura)\gets 1 + $max($(p\to izq)\to altura, (p\to der)\to altura)$
    \Comment $\Ogr(1)$
  \EndIf
\EndProcedure
\end{algorithmic}
\underline{Complejidad:} $\Ogr(1)$ \\
\underline{Justificación:} $\Ogr(1) + \Ogr(1) = \Ogr(1)$
\end{algorithm}

\begin{algorithm}
\caption{rotarSimple}
\begin{algorithmic}[1]
\Procedure{rotarSimple}{\texttt{in/out} $a$ : \texttt{puntero(puntero(nodo))}, \texttt{in} $rota\_izq$ : \texttt{bool}}
	\State puntero(nodo) $a_1$
	\Comment $\Ogr(1)$
  \If{$rota\_izq$}
  \Comment $\Ogr(1)$
    \State $a_1 \gets (*a\to izq)$
    \Comment $\Ogr(1)$
    \State $(*a\to izq) \gets (a_1\to der)$
    \Comment $\Ogr(1)$
    \State $(a_1\to der)\gets *a$
    \Comment $\Ogr(1)$
  \Else
    \State $a_1 \gets (*a\to der)$
    \Comment $\Ogr(1)$
    \State $(*a\to der) \gets (a_1\to izq)$
    \Comment $\Ogr(1)$
    \State $(a_1\to izq)\gets *a$
    \Comment $\Ogr(1)$
  \EndIf

  \State actualizarAltura($*a$)
  \Comment $\Ogr(1)$
  \State actualizarAltura($a_1$)
  \Comment $\Ogr(1)$
  
  \State $*a \gets a_1$
  \Comment $\Ogr(1)$
\EndProcedure
\end{algorithmic}
\underline{Complejidad:} $\Ogr(1)$ \\
\underline{Justificación:} Siempre se llevan a cabo 8 operaciones con costo $\Ogr(1)$. Luego, el orden total es $\Ogr(1)$.
\end{algorithm}

\begin{algorithm}
\caption{rotarDoble}
\begin{algorithmic}[1]
\Procedure{rotarDoble}{\texttt{in/out} $a$ : \texttt{puntero(puntero(nodo))}, \texttt{in} $rota\_izq$ : \texttt{bool}}
	\If{$rota\_izq$}
	\Comment $\Ogr(1)$
    \State rotarSimple($\&(*a\to izq), false$)
    \Comment $\Ogr(1)$
    \State rotarSimple($a, true$)
    \Comment $\Ogr(1)$
  \Else
		\State rotarSimple($\&(*a\to der), true$)
		\Comment $\Ogr(1)$
    \State rotarSimple($a, false$)
    \Comment $\Ogr(1)$
  \EndIf
\EndProcedure
\end{algorithmic}
\underline{Complejidad:} $\Ogr(1)$ \\
\underline{Justificación:} Siempre se llevan a cabo 3 operaciones con costo $\Ogr(1)$
\end{algorithm}

\begin{algorithm}
\caption{balancear}
\begin{algorithmic}[1]
\Procedure{balancear}{\texttt{in/out} $a$}
	\If{$*a \neq$ NULL}
	\Comment $\Ogr(1)$
 		\If{$(*a\to izq)\to altura \geq (*a\to der)\to altura$}		
 		\Comment $\Ogr(1)$
 			\If{$(*a\to izq)\to altura - (*a\to der)\to altura = 2$}
 				\Comment desequilibrio hacia la izquierda $\Ogr(1)$
 				\If{$((*a\to izq)\to izq)\to altura \geq ((*a\to izq)\to der)\to altura $}
 		   		\Comment desequilibrio simple $\Ogr(1)$
 		   		\State rotarSimple($a, true$)
 		   		\Comment $\Ogr(1)$
 		   	\Else \Comment desequilibrio doble
 		   		\State rotarDoble($a, true$)
 		   		\Comment $\Ogr(1)$
 		   	\EndIf			
			\EndIf	
			\Else
			\If{$(*a\to der)\to altura - (*a\to izq)\to altura = 2$}
				\Comment desequilibrio hacia la derecha $\Ogr(1)$
				\If{$((*a\to der)\to der)\to altura \geq ((*a\to der)\to der)\to altura $}
 		   		\Comment desequilibrio simple $\Ogr(1)$
 		   		\State rotarSimple($a, false$)
 		   		\Comment $\Ogr(1)$
 		   	\Else \Comment desequilibrio doble
 		   		\State rotarDoble($a, false$)
 		   		\Comment $\Ogr(1)$
 		   	\EndIf
			\EndIf
		\EndIf
  \EndIf
\EndProcedure
\end{algorithmic}
\underline{Complejidad:} $\Ogr(1)$ \\
\underline{Justificación:} Siempre se llevan a cabo 5 operaciones con costo $\Ogr(1)$
\end{algorithm}

\begin{algorithm}
\caption{Vacio}
\begin{algorithmic}[1]
\Procedure{iVacio}{}$\to res$ : \texttt{puntero(puntero(nodo))}
	\State $*res \gets$ NULL
	\Comment $\Ogr(1)$
\EndProcedure
\end{algorithmic}
\underline{Complejidad:} $\Ogr(1)$
\end{algorithm}

\begin{algorithm}
\caption{Definir}
\begin{algorithmic}[1]
\Procedure{iDefinir}{\texttt{in/out} $d$ : \texttt{puntero(puntero(nodo))}, \texttt{in} $k$ : \texttt{$\kappa$}, \texttt{in} $s$ : \texttt{$\sigma$}}
	\If{$*d =$ NULL}
	\Comment $\Ogr(1)$
    \State nodo $nuevo \gets$ $(k, s,$ NULL, NULL, $1)$
    \Comment $\Ogr(1)$
    \State $*d \gets \&nuevo$
    \Comment $\Ogr(1)$
  \Else
		\If{$k < *d\to clave$}
		\Comment $\Ogr(comp_{\kappa}(k, k'))$
			\State Definir($\&(*d\to izq), k, s$)
			\Comment $\Ogr(T(n/2))$
		\Else
			\State Definir($\&(*d\to der), k, s$)
			\Comment $\Ogr(T(n/2))$
		\EndIf
  \EndIf
  \State balancear($d$)
  \Comment $\Ogr(1)$
  \State actualizarAltura($*d$)
  \Comment $\Ogr(1)$
\EndProcedure
\end{algorithmic}
\underline{Complejidad:} $\Ogr(log(n) \times comp_{\kappa}(k, k'))$ \\
\underline{Justificación:} Sea T($m$) la función costo de ejecutar Definir, con $m = n + 1$, donde $n$ cantidad de nodos del árbol que se pasa como parámetro (el +1 es por cuestiones prácticas a la hora de definir el caso base), definida de la siguiente forma:

T(1) = 4, T($m$) = T($m/2$) + $comp_{\kappa}(k, k') + 2$ si $m > 1$

Notar que como el AVL está casi perfectamente balanceado, no está mal suponer que cada recursión toma como parámetro aproximadamente la mitad de la cantidad de los elementos del árbol.

Si el árbol es de altura $h$, el peor caso es que tenga $n = 2^{h}$ nodos, es decir que tenga todos sus niveles completos, pues habrá que recorrer una rama de largo $h = log_2(n)$ para agregar el nuevo elemento. Luego, T se puede escribir como T($m$) = $log_2(n + 1) \times (comp_{\kappa}(k, k') + 2) + 4$. Tomando $\Ogr$, nos queda $\Ogr(T) = \Ogr(max(log_2(n + 1) \times (comp_{\kappa}(k, k') + 2), 1)) = \Ogr(log_2(n + 1) \times (comp_{\kappa}(k, k') + 2)) = \Ogr(log_2(n + 1) \times comp_{\kappa}(k, k')) = \Ogr(log(n + 1) \times comp_{\kappa}(k, k'))$, por álgebra de órdenes. Y como el análisis es asintótico, $\Ogr(log(n + 1) \times comp_{\kappa}(k, k')) = \Ogr(log(n) \times comp_{\kappa}(k, k'))$.

\end{algorithm}

\begin{algorithm}
\caption{Definido?}
\begin{algorithmic}[1]
\Procedure{iDefinido?}{\texttt{in} $d$ : \texttt{puntero(puntero(nodo))}, \texttt{in} $k$ : \texttt{$\kappa$}}$\to res$ : \texttt{bool}
	\If{$*d = NULL$}
	\Comment $\Ogr(1)$
	 \State $res\gets false$
	 \Comment $\Ogr(1)$
	 \Else	
			\If{$k < *d\to clave$}	
			\Comment $\Ogr(comp_{\kappa}(k, k'))$
				\State Definido?($\&(*d\to izq), k$)	
				\Comment $T(n/2)$	
			\Else
				\If{$*d\to clave < k$}
				\Comment $\Ogr(comp_{\kappa}(k, k'))$
					\State Definido?($\&(*d\to der), k$)
					\Comment $T(n/2)$
				\Else
					\State $res\gets true$
					\Comment $\Ogr(1)$	
				\EndIf
			\EndIf
	\EndIf
\EndProcedure
\end{algorithmic}
\underline{Complejidad:} $\Ogr(log(n)  \times comp_{\kappa}(k, k'))$ \\
\underline{Justificación:} Dado que los que nos importa es la complejidad en el peor caso podemos asumir que el elemento buscado no está. Sea T($m$) la función de costo de ejecutar Definido?, con $m = n + 1$, donde $n$ cantidad de nodos del árbol que se pasa como parámetro, dada por: T(1) = 2, T($m$) = T($m/2$) + 2 $\times comp_{\kappa}(k, k') + 1$ si $m > 1$.

Por un argumento análogo al usado en Definir podemos llegar a que, en el peor caso, T($m$) = $log_2(n + 1) \times (2 \times comp_{\kappa}(k, k') + 1) + 2$. Álgebra de órdenes mediante, llegamos a que $\Ogr(T) = \Ogr(log(n) \times comp_{\kappa}(k, k'))$
\end{algorithm}

\begin{algorithm}
\caption{Obtener}
\begin{algorithmic}[1]
\Procedure{iObtener}{\texttt{in} $d$ : \texttt{puntero(puntero(nodo))}, \texttt{in} $k$ : \texttt{$\kappa$}}$\to res$ : \texttt{$\sigma$}	
	\If{$k = *d\to clave$}
	\Comment $\Ogr(equal_{\kappa}(k, k'))$
		\State $res\gets (*d\to significado)$
		\Comment $\Ogr(1)$
	\Else
		\If{$k < *d\to clave$}	
		\Comment $\Ogr(comp_{\kappa}(k, k'))$
			\State Obtener($\&(*d\to izq), k$)	
			\Comment $T(n/2)$	
		\Else
			\State Obtener($\&(*d\to der), k$)
			\Comment $T(n/2)$
		\EndIf
	\EndIf

\EndProcedure
\end{algorithmic}
\underline{Complejidad:} $\Ogr(log(n) \times (equal_{\kappa}(k, k') + comp_{\kappa}(k, k')) + equal_{\kappa}(k, k'))$ \\
\underline{Justificación:} Sea T($n$) la función costo de ejecutar Obtener para un árbol de tamaño $n$, dada por: T(1) = $equal_{\kappa}(k, k') + 1$ , T($n$) = T($n/2) + equal_{\kappa}(k, k') + comp_{\kappa}(k, k')$. Como nos importa el peor caso, vamos a asumir que el nodo buscado es una hoja y que el árbol está balanceado por lo que la hoja está a distancia $log_2(n)$ de la raíz. Luego, en este caso, T($n$) = $log_2(n) \times (equal_{\kappa}(k, k') + comp_{\kappa}(k, k')) + equal_{\kappa}(k, k') + 1$. 

Aplicando $\Ogr$, $\Ogr(T) = \Ogr(log(n) \times (equal_{\kappa}(k, k') + comp_{\kappa}(k, k')) + equal_{\kappa}(k, k'))$.
\end{algorithm}

\begin{algorithm}
\caption{Borrar}
\begin{algorithmic}[1]
\Procedure{iBorrar}{\texttt{in/out} $d$ : \texttt{puntero(puntero(nodo))}, \texttt{in} $k$ : \texttt{$\kappa$}}
	\State puntero(nodo) $aux$
	\Comment $\Ogr(1)$
	\If{$k < (*d\to clave)$}
	\Comment $\Ogr(comp_{\kappa}(k, k'))$
		\State Borrar($\&(*d\to izq), k$)
		\Comment $T(n/2)$
	\Else 
		\If{$*d\to clave < k$}
		\Comment $\Ogr(comp_{\kappa}(k, k'))$
			\State Borrar($\&(*d\to der), k$)
			\Comment $T(n/2)$
		\Else 
			\If{$*d\to izq =$ NULL $\land *d\to der =$ NULL}
				\Comment Es una hoja $\Ogr(1)$
				\State delete($*d$)
				\Comment $\Ogr(1)$
				\State $*d\gets$ NULL
				\Comment $\Ogr(1)$
			\Else
				\If{$*d\to izq =$ NULL}
					\Comment subárbol izquierdo vacío $\Ogr(1)$
					\State $aux \gets *d$
					\Comment $\Ogr(1)$
					\State $*d\gets (*d\to der)$
					\Comment $\Ogr(1)$
					\State delete($aux$)
					\Comment $\Ogr(1)$
					\State $aux \gets$ NULL
					\Comment $\Ogr(1)$
				\Else
					\If{$*d\to der =$ NULL} 
						\Comment subárbol derecho vacío $\Ogr(1)$
						\State $aux \gets *d$
						\Comment $\Ogr(1)$
						\State $*d\gets (*d\to izq)$
						\Comment $\Ogr(1)$
						\State delete($aux$)
						\Comment $\Ogr(1)$
						\State $aux \gets$ NULL
						\Comment $\Ogr(1)$
					\Else
						\Comment el árbol tiene dos hijos
						\State tupla($\kappa, \sigma)$ $min$ $\gets$ borrarMin($\&(*d\to der)$)
						\Comment $\Ogr(log(n))$
						\State($*d\to clave) \gets \pi_1(min)$
						\Comment $\Ogr(1)$
						\State($*d\to significado) \gets \pi_2(min)$
						\Comment $\Ogr(1)$
					\EndIf
				\EndIf
			\EndIf 
		\EndIf
	\EndIf
	\State balancear($d$)
	\Comment $\Ogr(1)$
	\State actualizarAltura($*d$)
	\Comment $\Ogr(1)$
\EndProcedure
\end{algorithmic}
\underline{Complejidad:} $\Ogr(log(n) \times comp_{\kappa}(k, k'))$ \\
\underline{Justificación:} Podemos separar a la función borrar en tres partes: búsqueda del elemento, eliminación y rebalanceo.

La búsqueda en el peor de los casos será $\Ogr(log(n) \times comp_{\kappa}(k, k'))$.

La eliminación tiene tres casos posibles de los cuáles el peor claramente es en el que el nodo a eliminar tiene dos hijos, pues es el único en el que se hace uso de una función no constante. El orden de la eliminación en este caso es $\Ogr(log(n))$.

Finalmente el rebalanceo (y la actualización de la altura) se realizará para cada recursión que haya ocurrido hasta encontrar el elemento. Esto es en el peor caso $\Ogr(log(n))$.

Luego, $\Ogr(log(n)) \times comp_{\kappa}(k, k') + \Ogr(log(n)) + \Ogr(log(n)) = \Ogr(log(n)) \times comp_{\kappa}(k, k')$  
\end{algorithm}

\begin{algorithm}
\caption{borrarMin}
\begin{algorithmic}[1]
\Procedure{borrarMin}{\texttt{in/out} $d$ : \texttt{puntero(puntero(nodo))}}$\to res$ : \texttt{tupla($\kappa, \sigma$)}
	\State \Comment $Pre: d \neq NULL \yluego *d \neq NULL$
	\If{$*d\to izq \neq$ NULL}
	\Comment $\Ogr(1)$
		\State tupla($\kappa, \sigma$) $x\gets$ borrarMin($\&(*d\to izq)$)
		\Comment $\Ogr(T(n/2))$
		\State balancear($d$)
		\Comment $\Ogr(1)$
		\State actualizarAltura($*d$)
		\Comment $\Ogr(1)$
		\State $res \gets x$
		\Comment $\Ogr(1)$
	\Else
		\State puntero(nodo) $aux \gets *d$
		\Comment $\Ogr(1)$
		\State tupla($\kappa, \sigma) x \gets (aux\to clave, aux\to significado$)
		\Comment $\Ogr(1)$
		\State $*d\gets (*d\to der)$
		\Comment $\Ogr(1)$
		\State delete($aux$)
		\Comment $\Ogr(1)$
		\State $aux \gets$ NULL
		\Comment $\Ogr(1)$
		\State balancear($d$)
		\Comment $\Ogr(1)$
		\State actualizarAltura($*d$)
		\Comment $\Ogr(1)$
		\State $res \gets x$
		\Comment $\Ogr(1)$
	\EndIf
\EndProcedure
\end{algorithmic}
	\underline{Complejidad:} $\Ogr(log(n))$ \\
	\underline{Justificación:} La función de costo en este caso es T(1) = 9, T($m$) = T($m/2$) + 4. Luego, en el peor caso (análogamente a los algoritmos anteriores), T($m$) = $log_2(n + 1) \times 4 + 9$. Es inmediato ver entonces que $\Ogr(T) = \Ogr(log(n))$
\end{algorithm}

\end{Algoritmos}

\clearpage

\section{Módulo ConjAVL($\alpha$)}
\begin{Interfaz}

  \textbf{se explica con}: \tadNombre{conj($\alpha$)}.

  \textbf{géneros}: \TipoVariable{conjAVL($\alpha$)}.

  \section*{Operaciones básicas ConjAVL($\alpha$)}

  \InterfazFuncion{Vacio}{}{conjAVL($\alpha$)}
  {$res \igobs \emptyset$}
  [$\Theta(1)$]
  [genera un conjunto vacío.]

  \InterfazFuncion{Agregar}{\Inout{c}{conjAVL($\alpha$)}, \In{a}{$\alpha$}}{}
  [$c \igobs c_0$]  
  {$c$ \igobs Ag($a,c_0$)}
  [$\Theta(log(n))$\footnote{Dentro de este módulo, n es la cantidad de elementos del conjunto.}]
  [agrega el elemento $a$ al conjunto $c$.]
  [$a$ se agrega por referencia.]
    
  \InterfazFuncion{Pertenece?}{\In{c}{conjAVL($\alpha$)}, \In{a}{$\alpha$}}{bool}
  {$res$ \igobs ($a\in c$) }
  [$\Theta(log(n))$]
  [devuelve $true$ si y sólo si el elemento $a$ pertenece al conjunto.]
  
  \InterfazFuncion{Eliminar}{\In{c}{conjAVL($\alpha$)}, \In{a}{$\alpha$}}{}
  [$c$ \igobs $c_0 \land a \in c$ ]
  {$c \igobs (c_0 - \{a\})$}
  [$\Theta(log(n))$]
  [elimina el elemento $a$ del conjunto.] 
  
\end{Interfaz}

\begin{Representacion}
    
  \section*{Representación de ConjAVL($\alpha$)}

  \begin{Estructura}{conjAVL($\alpha$)}[diccAVL($\alpha$, nat)]
  \end{Estructura}
  
  \Rep[diccAVL($\alpha${,} nat)][d]{true}
  
  \AbsFc[diccAVL($\alpha${,} nat)]{conjAVL($\alpha$)}[d]{
	$c$ : conj($\alpha$) | ($\forall a: \alpha$) ($a\in c \Leftrightarrow a\in claves(d)$)}
\end{Representacion}


\begin{Algoritmos}

\begin{algorithm}
\caption{Vacio}
\begin{algorithmic}[1]
\Procedure{iVacio}{}$\to res$ : \texttt{diccAVL($\alpha$, nat)}
	\State $res \gets$ Vacio()
\EndProcedure
\end{algorithmic}
\end{algorithm}

\begin{algorithm}
\caption{Agregar}
\begin{algorithmic}[1]
\Procedure{iAgregar}{\texttt{in/out} $c$ : \texttt{diccAVL($\alpha$, nat)}, \texttt{in} $a$ : \texttt{$\alpha$}}
	\State Definir($c, a, 0$)
\EndProcedure
\end{algorithmic}
\end{algorithm}

\begin{algorithm}
\caption{Pertenece?}
\begin{algorithmic}[1]
\Procedure{iPertenece?}{\texttt{in/out} $c$ : \texttt{diccAVL($\alpha$, nat)}, \texttt{in} $a$ : \texttt{$\alpha$}}$\to res$ : \texttt{bool}
	\State $res \gets$ Definido?($c, a$)
\EndProcedure
\end{algorithmic}
\end{algorithm}

\begin{algorithm}
\caption{Eliminar}
\begin{algorithmic}[1]
\Procedure{iEliminar}{\texttt{in/out} $c$ : \texttt{diccAVL($\alpha$, nat)}, \texttt{in} $a$ : \texttt{$\alpha$}}
	\State Borrar($c, a$)
\EndProcedure
\end{algorithmic}
\end{algorithm}
\end{Algoritmos}
\clearpage


\section{Módulo Cola de Prioridad($\alpha$)}



\section{Módulo Cola de Prioridad($\alpha$)}

\begin{Interfaz}

  \textbf{se explica con}: Cola de Prioridad($\alpha$), Iterador Unidireccional Modificable($\alpha$)

  \textbf{usa}: Nat, bool
  
  \textbf{genero}: colaPrior($\alpha$), itColaPrior($\alpha$)
  
\subsubsection{Operaciones de Cola de Prioridad}

  \InterfazFuncion{Vacia}{}{colaPrior($\alpha$)}
  [true]
  {$res$ $\igobs$ vacia}
  [O(1)]
  [Crea una cola de prioridad]\\ 
  
  \InterfazFuncion{Vacia?}{\In{c}{colaPrior($\alpha$)}}{bool}
  [true]
  {$res$ $\igobs$ vacia?(c)}
  [O(1)]
  [Dice si la cola no tiene ningun elemento]\\ 

  \InterfazFuncion{Desencolar}{\Inout{c}{colaPrior($\alpha$)}}{$\alpha$}
  [$\neg$vacia?($c$) $\land$ $c$ $\igobs$ $c_0$]
  {$res$ $\igobs$ proximo($c_0$) $\land$ $c$ $\igobs$ desencolar($c_0$)}
  [O(log(tamano(c)))]
  [Quita el elemento mas prioritario]\\   
  
  \InterfazFuncion{Encolar}{\Inout{c}{colaPrior($\alpha$)}, \In{a}{$\alpha$}}{}
  [$c$ $\igobs$ $c_0$ $\land$ $\neg$esta($a$, $c$)] %agregar el esta
  {$c$ $\igobs$ encolar(a,$c_0$)}
  [O(log(tamano(c)))]
  [Agrega al elemento a a la cola de prioridad]
  [El iterador se invalida si, y solo si se elimina el elemento siguiente del iterador sin llamar a la funcion Eliminar del mismo]\\ 
\end{Interfaz}


\subsection{Representación de la cola de prioridad}
\begin{Representacion}
  
  \begin{Estructura}{colaPrior($\alpha$)}[estr]

  \begin{Tupla}[estr]
    \tupItem{tam}{nat}%
    \tupItem{\\ cabeza}{puntero(nodo)}%
    \tupItem{\\ ultimo}{puntero(nodo)}%
  \end{Tupla}

  ~

  \begin{Tupla}[nodo]
    \tupItem{padre}{puntero(nodo)}%
    \tupItem{\\ izq}{puntero(nodo)}%
    \tupItem{\\ der}{puntero(nodo)}%
    \tupItem{\\ dato}{puntero($\alpha$)}%
  \end{Tupla}

  \end{Estructura}


  
FALTA INVREP Y ABS



\end{Representacion}






\begin{Algoritmos}





\end{Algoritmos}












\clearpage

\section{Módulo DCNet}
\begin{Interfaz}
  
  \textbf{se explica con}: \tadNombre{dcnet}.

  \textbf{géneros}: \TipoVariable{dcnet}.

  \section*{Operaciones básicas de DCNet}

  \InterfazFuncion{Iniciar}{\In{r}{red}}{dcnet}
  {$res \igobs iniciarDCNet(r)$}
  [$\Theta(1)$]
  [genera una DCNet con las computadoras y conexiones de la red pasada como parámetro.]

  \InterfazFuncion{CrearPaquete}{\Inout{d}{dcnet}, \In{p}{paquete}}{}
  [$d \igobs d_0 \land 
   \neg((\exists p':paquete) (paqueteEnTransito?(d_0,p') \land id(p') \igobs id(p)) \land\\ 
   \hspace*{3em} origen(p) \in computadoras(red(d_0)) \yluego
   destino(p) \in computadoras(red(d_0)) \yluego\\
   \hspace*{3em} hayCamino?(red(d), origen(p), destino(p))$]
  {$d \igobs crearPaquete(d_0, p)$}
  [$\Theta()$]
  [agrega el paquete $p$ a la cola de la computadora $p.origen$.]
  
  \InterfazFuncion{AvanzarSegundo}{\Inout{d}{dcnet}}{dcnet}
  [$d \igobs d_0$]  
  {$res \igobs avanzarSegundo(d_0)$}
  [$\Theta()$]
  [avanza un segundo; todas las computadoras que tengan paquetes por enviar los envían.]
  
  \section*{Operaciones del iterador}

  \InterfazFuncion{CrearIt}{\In{l}{lista($\alpha$)}}{itLista($\alpha$)}
  {$res$ $\igobs$ crearItBi(\secuencia{}, $l$) $\land$ alias(SecuSuby($it$) $=$ $l$)}
  [$\Theta(1)$]
  [crea un iterador bidireccional de la lista, de forma tal que al pedir \NombreFuncion{Siguiente} se obtenga el primer elemento de $l$.]
  [el iterador se invalida si y sólo si se elimina el elemento siguiente del iterador sin utilizar la función \NombreFuncion{EliminarSiguiente}.]
  
  \InterfazFuncion{CrearItUlt}{\In{l}{lista($\alpha$)}}{itLista($\alpha$)}
  {$res$ $\igobs$ crearItBi($l$, \secuencia{}) $\land$ alias(SecuSuby($it$) $=$ $l$)}
  [$\Theta(1)$]
  [crea un iterador bidireccional de la lista, de forma tal que al pedir \NombreFuncion{Anterior} se obtenga el último elemento de $l$.]  
  [el iterador se invalida si y sólo si se elimina el elemento siguiente del iterador sin utilizar la función \NombreFuncion{EliminarSiguiente}.]

\end{Interfaz}

\begin{Representacion}
  
  \section*{Representación de DCNet}

  \begin{Estructura}{lista$(\alpha)$}[lst]
    \begin{Tupla}[lst]
      \tupItem{primero}{puntero(nodo)}%
      \tupItem{longitud}{nat}%
    \end{Tupla}

    \begin{Tupla}[nodo]
      \tupItem{dato}{$\alpha$}%
      \tupItem{anterior}{puntero(nodo)}%
      \tupItem{siguiente}{puntero(nodo)}%
    \end{Tupla}
  \end{Estructura}

  \Rep[lst][l]{($l$.primero $=$ NULL) $=$ ($l$.longitud $=$ $0$) $\yluego$ ($l$.longitud $\neq$ $0$ \impluego \\
    Nodo($l$, $l$.longitud) $=$ $l$.primero $\land$ \\
    ($\forall i$: nat)(Nodo($l$,$i$)\DRef siguiente $=$ Nodo($l$,$i+1$)\DRef anterior) $\land$ \\
    ($\forall i$: nat)($1 \leq i <$ $l$.longitud $\implies$ Nodo($l$,$i$) $\neq$ $l$.primero)}\mbox{}

  ~      

  \tadOperacion{Nodo}{lst/l,nat}{puntero(nodo)}{$l$.primero $\neq$ NULL}
  \tadAxioma{Nodo($l$,$i$)}{\IF $i = 0$ THEN $l$.primero ELSE Nodo(FinLst($l$), $i-1$) FI}

  ~

  \tadOperacion{FinLst}{lst}{lst}{}
  \tadAxioma{FinLst($l$)}{Lst($l$.primero\DRef siguiente, $l$.longitud $-$ $\min$\{$l$.longitud, $1$\})}

  ~

  \tadOperacion{Lst}{puntero(nodo),nat}{lst}{}
  \tadAxioma{Lst($p,n$)}{$\langle p, n\rangle$}

  ~
 
  \AbsFc[lst]{secu($\alpha$)}[l]{\IF $l$.longitud $=$ $0$ THEN \secuencia{} ELSE \secuencia{$l$.primero\DRef dato}[Abs(FinLst($l$))] FI}

  \section*{Representación del iterador}

  \begin{Estructura}{itLista($\alpha$)}[iter]
    \begin{Tupla}[iter]
      \tupItem{siguiente}{puntero(nodo)}%
      \tupItem{lista}{puntero(lst)}%
    \end{Tupla}
  \end{Estructura}

  \Rep[iter][it]{Rep($\ast$($it$.lista)) $\yluego$ ($it$.siguiente $=$ NULL $\oluego$ ($\exists i$: nat)(Nodo($\ast it$.lista, $i$) $=$ $it$.siguiente)}

  ~

  \Abs[iter]{itBi($\alpha$)}[it]{b}{Siguientes($b$) $=$ Abs(Sig($it$.lista, $it$.siguiente)) $\land$\\
    Anteriores($b$) $=$ Abs(Ant($it$.lista, $it$.siguiente))}

  ~

  \tadOperacion{Sig}{puntero(lst)/l,puntero(nodo)/p}{lst}{Rep($\langle l, p\rangle$)}
  \tadAxioma{Sig($i, p$)}{Lst($p$, $l$\DRef longitud $-$ Pos($\ast l$, $p$))}

  ~

  \tadOperacion{Ant}{puntero(lst)/l,puntero(nodo)/p}{lst}{Rep($\langle l, p\rangle$)}
  \tadAxioma{Ant($i, p$)}{Lst(\IF $p$ $=$ $l$\DRef primero THEN NULL ELSE $l$\DRef primero FI, Pos($\ast l$, $p$))}

  ~

  {\small Nota: cuando $p$ $=$ NULL, Pos devuelve la longitud de la lista, lo cual está bien, porque significa que el iterador no tiene siguiente.}
  \tadOperacion{Pos}{lst/l,puntero(nodo)/p}{puntero(nodo)}{Rep($\langle l, p\rangle$)}
  \tadAxioma{Pos($l$,$p$)}{\IF $l$.primero $=$ $p$ $\lor$ $l$.longitud $=$ $0$ THEN $0$ ELSE $1$ $+$ Pos(FinLst($l$), $p$) FI}


\end{Representacion}


\clearpage

\end{document}
