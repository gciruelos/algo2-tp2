\documentclass[a4paper,10pt, nofootinbib]{article}
\usepackage[width=15.5cm, left=3cm, top=2.5cm, right=1cm, left=2cm, height= 24.5cm]{geometry}
\usepackage[spanish]{babel}
\usepackage[utf8]{inputenc}
\usepackage[T1]{fontenc}
\usepackage{xspace}
\usepackage{xargs}
\usepackage{ifthen}
\usepackage{caratula}
\usepackage{fancyhdr}
\usepackage{aed2-tad,aed2-symb,aed2-itef}
\usepackage[bottom]{footmisc}
\usepackage{modulos}

\newcommand{\moduloNombre}[1]{\textbf{#1}}


\let\NombreFuncion=\textsc
\let\TipoVariable=\texttt
\let\ModificadorArgumento=\textbf
\newcommand{\res}{$res$\xspace}
\newcommand{\tab}{\hspace*{7mm}}

\newcommand{\footnoteMio}[2]{\textsuperscript{#1} \lfoot{\footnotesize \parbox{16cm}{\textsuperscript{#1}#2}}}

\newcommandx{\TipoFuncion}[3]{%
  \NombreFuncion{#1}(#2) \ifx#3\empty\else $\to$ \res\,: \TipoVariable{#3}\fi%
}
\newcommand{\In}[2]{\ModificadorArgumento{in} \ensuremath{#1}\,: \TipoVariable{#2}\xspace}
\newcommand{\Out}[2]{\ModificadorArgumento{out} \ensuremath{#1}\,: \TipoVariable{#2}\xspace}
\newcommand{\Inout}[2]{\ModificadorArgumento{in/out} \ensuremath{#1}\,: \TipoVariable{#2}\xspace}
\newcommand{\Aplicar}[2]{\NombreFuncion{#1}(#2)}

\newcommand{\Titulo}[1]{
  \vspace*{1ex}\par\noindent\textbf{\large #1}\par
}
\newcommand{\DRef}{\ensuremath{\rightarrow}}

%%Información para la carátula
\materia{Algoritmos y Estructuras de Datos II}

\titulo{\LARGE Trabajo Práctico Nº2}

\grupo{Grupo 18}

\integrante{Ciruelos Rodríguez, Gonzalo}{063/14}{gonzalo.ciruelos@gmail.com}
\integrante{Costa, Manuel José Joaquín}{035/14}{manuc94@hotmail.com}
\integrante{Gatti, Mathias Nicolás}{477/14}{mathigatti@gmail.com}
\integrante{Ginsberg, Mario Ezequiel}{145/14}{ezequielginsberg@gmail.com}

\def\cuatrimestre{1}

%%fancyhdr
\pagestyle{fancy}
\thispagestyle{fancy}
\addtolength{\headheight}{1pt}
\lhead{Algoritmos y Estructuras de Datos II: TP1}
\rhead{$1^{\mathrm{er}}$ cuatrimestre de 2015}
\cfoot{\thepage\ / 11}
\renewcommand{\footrulewidth}{0.4pt}

\setlength{\parskip}{0.8em}


\begin{document}
\maketitle
\thispagestyle{empty}

\section{Módulo Red}
\begin{Interfaz}
  
  \textbf{se explica con}: \tadNombre{Red}.

  
  \section*{Operaciones básicas de \tadNombre{Red}}

  \InterfazFuncion{IniciarRed}{}{red}
  {$res \igobs iniciarRed()$}
  [$\Theta()$]
  [genera la red vacía]

  \InterfazFuncion{AgregarComputadora}{\Inout{r}{\tadNombre{Red}}, \In{c}{\tadNombre{Compu}}}{}
  [$r \igobs r_0 \land (\forall c' : compu) (c' \in computadoras(r) \impluego ip(c) \neq ip(c'))$] 
  {$computadoras(r) = Ag(c, computadoras(r_0)) \yluego \\
     (\forall c_1, c_2 : compu) (c_1 \in computadoras(r_0) \land c_2 \in computadoras(r_0) \impluego \\ 
   conectadas(r_0, c_1, c_2) = conectadas(r, c_1, c_2) \land interfazUsada(r_0, c_1, c_2) = interfazUsada(r, c_1, c_2)$)}
  [$\Theta()$]
  [agrega el la computadora $c$ a la red $r$.]
  [el elemento $c$ agrega por copia.]
  
  
  \InterfazFuncion{conectar}{\Inout{r}{\tadNombre{Red}}, \In{c_1}{\tadNombre{Compu}},  \In{i_1}{\tadNombre{Interfaz}}, \In{c_2}{\tadNombre{Interfaz}}, \In{i_2}{\tadNombre{Interfaz}}}{}
 [$r \igobs r_0 \land c_1 \in computadoras(r) \land c_2 \in computadoras(r) \land ip(c_1) \neq ip(c_2) \land \neg conectadas?(r, c_1, c_2) \land \neg usaInterfaz?(r,c_1, i_1) \land \neg usaInterfaz?(r, c_2, i_2)$]
  {$computadoras(r) = computadoras(r_0) \yluego \\
    (\forall c'_1, c'_2 : compu) (c'_1 \in computadoras(r_0) \land c'_2 \in computadoras(r_0) \land \{c_1, c_2\} \neq \{c'_1, c'_2\} \impluego \\ 
   conectadas(r_0, c'_1, c'_2) = conectadas(r, c'_1, c'_2) \land interfazUsada(r_0, c'_1, c'_2) = interfazUsada(r, c'_1, c'_2)) \land \\
   conectadas(r_0, c_1, c_2) \land interfazUsada(r, c_1, c_2) = i_1 \land interfazUsada(r, c_2, c_1) = i_2$}
  [$\Theta()$]
  [conecta las computadoras $c_1$ y $c_2$ mediante las interfaces $i_1$ e $i_2$ respectivamente en la red $r$.]


  \InterfazFuncion{computadoras}{\In{r}{\tadNombre{Red}}}{}
  {$res = computadoras(r)$}
  [$\Theta()$]
  [devuelve un conjunto con todas las computadoras de la red $r$.]
  [?]


  \InterfazFuncion{conectadas?}{\In{r}{\tadNombre{Red}}, \In{c_1}{\tadNombre{Compu}}, \In{c_2}{\tadNombre{Compu}}}{\tadNombre{bool}}
  [$c_1 \in computadoras(r) \land c_2 \in computadoras(r)$]
  {$res = conectadas?(r,c_1,c_2)$}
  [$\Theta()$]
  [dice si las computadoras $c_1$ y $c_2$ están conectadas en la red $r$]


  \InterfazFuncion{interfazUsada}{\In{r}{\tadNombre{Red}}, \In{c_1}{\tadNombre{Compu}}, \In{c_2}{\tadNombre{Compu}}}{\tadNombre{bool}}
  [$conectadas?(r,c_1,c_2)$]
  {$res = interfazUsada(r,c_1,c_2)$}
  [$\Theta()$]
  [devuelve la interfaz mediante la cual $c_1$ se conecta a $c_2$  en la red $r$.]


  \InterfazFuncion{vecinos}{\In{r}{\tadNombre{Red}}, \In{c}{\tadNombre{Compu}}}{\tadNombre{conj(compu)}}
  [$c \in computadoras(r)$]
  {$res = vecinos(r,c)$}
  [$\Theta()$]
  [devuelve las computadoras conectadas a $c$ en la red $r$.]
  [?]


  \InterfazFuncion{usaInterfaz?}{\In{r}{\tadNombre{Red}}, \In{c}{\tadNombre{Compu}}, \In{i}{\tadNombre{Interfaz}}}{\tadNombre{bool}}
  [$c \in computadoras(r)$]
  {$res = usaInterfaz?(r,c,i)$}
  [$\Theta()$]
  [devuelve true si y solo si $c$ está conectada a otra pc mediante la interfaz $i$ en la red $r$.]


  \InterfazFuncion{caminosMinimos}{\In{r}{\tadNombre{Red}}, \In{c_1}{\tadNombre{Compu}}, \In{c_2}{\tadNombre{Compu}}}{\tadNombre{conj(secu(compu))}}
  [$c_1 \in computadoras(r) \land c_2 \in computadoras(r)$]
  {$res = caminosMinimos(r,c_1,c_2)$}
  [$\Theta()$]
  [devuelve un conjunto con los caminos más cortos entre las computadoras $c_1$ y $c_2$ en la red $r$.]

%%

  \InterfazFuncion{hayCamino?}{\In{r}{\tadNombre{Red}}, \In{c_1}{\tadNombre{Compu}}, \In{c_2}{\tadNombre{Compu}}}{\tadNombre{bool}}
  [$c_1 \in computadoras(r) \land c_2 \in computadoras(r)$]
  {$res = hayCamino?(r,c_1,c_2)$}
  [$\Theta()$]
  [devuelve true si y solo si hay un camino entre las computadoras $c_1$ y $c_2$ en la red $r$.]



\end{Interfaz}

\begin{Representacion}
  
  
  
  \section*{Representación de la red}

  \begin{Estructura}{red}[estr]
    \begin{Tupla}[estr]
      \tupItem{vecinos}{dicc(id, dicc(id, interfaz))}\\
      \tupItem{interfaces}{dicc(id, conj(interfaz))}%
    \end{Tupla}

  \end{Estructura}

  \Rep[estr][e]{claves(e.vecinos) $=$ claves(e.interfaces) $\yluego$ \\
  ($\forall i_1, i_2$ : id) def?($i_2$, (obtener($i_1$, e.vecinos))) $\impluego$ def?($i_2$, e.vecinos) $\yluego$ def?($i_1$, obtener($i_2$, e.vecinos)) $\yluego$ \\
obtener($i_2$, obtener($i_1$, e.vecinos)) $\in$ obtener($i_1$, e.interfaces)
$\land$\\ 
($\forall i, i_1, i_2$ : id) def?($i$, e.vecinos) $\impluego$ \\
(def?($i_1$, obtener($i$, e.vecinos)) $\land$ def?($i_2$, obtener($i$, e.vecinos)) $\land$ $i_1 \neq i_2$ $\impluego$ obtener($i_1$, obtener($i$, e.vecinos)) $\neq$ obtener($i_2$, obtener($i$, e.vecinos)))

}\mbox{}
  
   
   \AbsFc[estr]{red}[e]{
     r : red / computadoras(r) = claves(e.vecinos) $\land$ \\ 
     ($\forall c_1, c_2$ : computadora) conectadas($r, c_1, c_2$) = def?(id($c_1$), obtener(id($c_2$), e.vecinos)) $\land$ \\
     ($\forall c_1, c_2$: computadora) ($\forall i$ : interfaz) conectadas($r,c_1,c_2$) $\impluego$ interfazUsada($r, c_1, c_2$) = obtener(id($c_1$), obtener(id($c_2$), e.vecinos))
  }

\end{Representacion}








\newpage
\section{Módulo DiccTrie($\sigma)$}
\begin{Interfaz}
  \textbf{parámetros formales}\hangindent=2\parindent\\
  \parbox{1.7cm}{\textbf{géneros}} $\sigma$\\
  \parbox[t]{1.7cm}{\textbf{función}}\parbox[t]{\textwidth-2\parindent-1.7cm}{    	
    \InterfazFuncion{Copiar}{\In{s}{$\sigma$}}{$\sigma$}
    {$res \igobs s$}
    [$\Theta(copy(s))$]
    [función de copia de $\sigma$]
    }
      
  \textbf{se explica con}: \tadNombre{diccionario($string, \sigma$)}.

  \textbf{géneros}: \TipoVariable{diccTrie($\sigma$), itClavesDiccTrie($\sigma$)}.

  \section*{Operaciones básicas DiccTrie($\sigma$)}

  \InterfazFuncion{Vacío}{}{diccTrie($\sigma$)}
  {$res$ \igobs vacío()}
  [$\Ogr(1)$]
  [genera un diccionario vacío.]

  \InterfazFuncion{Definir}{\Inout{d}{diccTrie($\sigma$)}, \In{k}{$string$}, \In{s}{$\sigma$}}{}
  [$d \igobs d_0$]  
  {$d$ \igobs definir($k, s, d_0$)}
  [$\Ogr(|k|)$]
  [define la clave $k$ con el sinificado $s$.]
  [$s$ se define por referencia.]
    
  \InterfazFuncion{Definido?}{\In{d}{diccTrie($\sigma$)}, \In{k}{$string$}}{bool}
  {$res$ \igobs def?($k$, $d$) }
  [$\Ogr(|k|)$]
  [devuelve $true$ si la clave $k$ está definida en el diccionario.]
  
  \InterfazFuncion{Obtener}{\In{d}{diccTrie($\sigma$)}, \In{k}{$string$}}{$\sigma$}
  [def?($k$, $d$)]  
  {alias($res$ \igobs obtener($k$, $d$))}
  [$\Ogr(|k|)$,]
  [devuelve el significado de la clave $k$ en $d$.]
  [$res$ es modificable si y sólo si $d$ lo es]
  
  \InterfazFuncion{Borrar}{\In{d}{diccTrie($\sigma$)}, \In{k}{$string$}}{}
  [$d$ \igobs $d_0 \land$ def?($k$, $d_0$) ]
  {$d \igobs$ borrar($k$, $d_0$)}
  [$\Ogr(|k|)$]
  [elimina la entrada $k$ del diccionario.] 
  
  \section*{Operaciones del iterador}

  \InterfazFuncion{CrearItClaves}{\In{d}{diccTrie($\sigma$)}}{itDiccTrie($\sigma$)}
  {alias(esPermutación?(SecuSuby($res$), clavesExtendidas($d$))) $\land$ vacia?(Anteriores($res$))}
  [$\Ogr(n)$]
  [crea un iterador del diccionario, de forma que pueda ser recorrido completamente aplicando iterativamente Siguiente.]
  
  \InterfazFuncion{HaySiguiente?}{\In{it}{itDiccTrie($\sigma$)}}{bool}
  {$res$ $\igobs$ haySiguiente($it$)}
  [$\Ogr(1)$]
  [devuelve $true$ si y sólo si quedan elementos para iterar.]  

  \InterfazFuncion{SiguienteClave}{\In{it}{itDiccTrie($\sigma$)}}{string}
  [haySiguiente?($it$)]  
  {alias($res$ $\igobs$ siguiente($it$).clave)}
  [$\Ogr(1)$]
  [devuelve la clave de la entrada del diccionario apuntada por el iterador.]  
  [$res$ no es modificable.]
  
  \InterfazFuncion{SiguienteSignificado}{\In{it}{itDiccTrie($\sigma$)}}{$\sigma$}
  [haySiguiente?($it$)]  
  {alias($res$ $\igobs$ siguiente($it$).significado)}
  [$\Ogr(1)$]
  [devuelve el significado de la entrada del diccionario apuntada por el iterador.]  
  [$res$ no es modificable.]  
  
  \InterfazFuncion{Avanzar}{\Inout{it}{itDiccTrie($\sigma$)}}{}
  [$it \igobs it_0 \land$ haySiguiente?($it_0$)]  
  {$it$ $\igobs$ avanzar($it_0$)}
  [$\Ogr(1)$]
  [avanza a la posición siguiente del iterador.] 
  
\end{Interfaz}

\newpage
\section{Módulo DCNet}
\begin{Interfaz}
  
  \textbf{se explica con}: \tadNombre{dcnet}.

  \textbf{géneros}: \TipoVariable{dcnet}.

  \section*{Operaciones básicas de DCNet}

  \InterfazFuncion{IniciarDCNet}{\In{r}{\tadNombre{red}}}{\tadNombre{dcnet}}
  [ $True$ ]
  {$ res \igobs IniciarDCNet(r)  $ }
  [$\Theta(1)$]
  [Genera una DCNet con las computadoras y conexiones de la red pasada como parámetro.]

  \InterfazFuncion{CrearPaquete}{\Inout{s}{\tadNombre{dcnet}}, \In{p}{\tadNombre{paquete}}}{}
  [$s \igobs s_0 \land 
   \neg((\exists p':paquete) (paqueteEnTransito?(s_0,p') \land id(p') \igobs id(p)) \land\\ 
   \hspace*{3em} origen(p) \in computadoras(red(s_0)) \yluego
   destino(p) \in computadoras(red(s_0)) \yluego\\
   \hspace*{3em} hayCamino?(red(s), origen(p), destino(p))$]
  {$s \igobs crearPaquete(s_0, p)$}
  [$\Theta()$]
  [Agrega el paquete $p$ a la cola de la computadora $p.origen$.]
  
  \InterfazFuncion{AvanzarSegundo}{\Inout{s}{\tadNombre{dcnet}}}{}
  [$s \igobs s_0$]  
  {$res \igobs avanzarSegundo(s_0)$}
  [$\Theta()$]
  [Avanza un segundo; todas las computadoras que tengan paquetes por enviar los envían.]
  
  
  \InterfazFuncion{red}{\In{s}{\tadNombre{dcnet}}}{\tadNombre{red}}
  [$ True $]  
  {$res \igobs red(s_0)$}
  [$\Theta(1)$]
  [Expresa la red que esta contenida en la DCNet.]

  
  \InterfazFuncion{caminoRecorrido}{\In{s}{\tadNombre{dcnet}}, \In{p}{\tadNombre{paquete}}}{\tadNombre{secu(compu)}}
  [$ paqueteEnTransito?(s,p) $]
  {$res \igobs caminoRecorrido(s, p)$}
  [$\Theta()$]
  [Expresa el camino de computadoras recorrido por el paquete $p$ desde su inicio hasta posicion actual.]


  \InterfazFuncion{cantidadEnviados}{\In{s}{\tadNombre{dcnet}}, \In{c}{\tadNombre{compu}}}{\tadNombre{nat}}
  [$ c \in computadoras(red(s)) $]  
  {$res \igobs cantidadEnviados(s, c)$}
  [$\Theta()$]
  [Expresa la cantidad de mensajes enviados por la compu $c$.]

  
  \InterfazFuncion{enEspera}{\In{s}{\tadNombre{dcnet}}, \In{c}{\tadNombre{compu}}}{\tadNombre{conj(paquete)}}
  [$ c \in computadoras(red(s)) $]  
  {$res \igobs enEspera(s, c)$}
  [$\Theta()$]
  [Expresa el camino de computadoras recorrido por el paquete $p$ desde su inicio hasta posicion actual.]


  \InterfazFuncion{PaqueteEnTransito}{\In{s}{\tadNombre{dcnet}}, \In{p}{\tadNombre{paquete}}}{\tadNombre{bool}}
  [$ True $]  
  {$res \igobs paqueteEnTransito?(s, p)$}
  [$\Theta()$]
  [Expresa si el paquete $p$ esta en alguna computadora.]
  
  
  \InterfazFuncion{LaQueMasEnvio}{\In{s}{\tadNombre{dcnet}}}{\tadNombre{compu}}
  [$ True $]
  {$res \igobs laQuemasEnvio(s)$}
  [$\Theta(1)$]
  [Devuelve la computadora que mas paquetes envio.]
  

\end{Interfaz}

\begin{Representacion}
  
  \section*{Representación de DCNet}

  \begin{Estructura}{dcnet}[net]
    \begin{Tupla}[net]
      \tupItem{proximaEnCamino}{dicc(compu, dicc(compu, compu))}%
      \tupItem{\\ paquetes}{diccTrie(compu,infoPaquetes)}%
      \tupItem{\\ caminosRecorridos}{lista(lista(compu))}
      \tupItem{\\ laQueMasEnvio}{tupla($cuantosEnvio$ : nat, $cualCompu$ : compu)}
      \tupItem{\\ red}{red}
    \end{Tupla}
  
  
  \begin{Tupla}[infoPaquetes]
      \tupItem{colas}{colaPrior($p$ : paq)}
      \tupItem{\\ diccPaqCamino}{diccAVL($p$ : paq, $camino$ : itLista )}
      \tupItem{\\ conjPaquetes}{conj(paq)}
      \tupItem{\\ cantEnviados}{nat}                       
    \end{Tupla}
  \end{Estructura}


  \Rep[net][n]{computadoras(n.red) = claves(n.paquetes) $\yluego$ ($\forall$ c : compu) [((c $\in$ claves(n.paquetes) $\rightarrow$ (($\forall$ p : paquete) p $\in$ definicion(n.paquetes, c).conjPaquetes $\rightarrow$ p $\in$ definicion(n.paquetes,c).colas $\land$ p $\in$ claves(definicion(paquetes, c).diccPaqCamino) $\land$  ($\forall$ p : paquete) p $\in$ claves(definicion(paquetes, c).diccPaqCamino) $\rightarrow$ p $\in$ definicion(n.paquetes, c).conjPaquetes  $\land$  (($\forall$ p : paquete) p $\in$ definicion(n.paquetes, c).colas $\rightarrow $ p $\in$ definiciones(paquetes,c).colas $yluego$ definicion(definicion(n.paquetes,c).diccPaqCamino,p).camino $\in$ caminosMinimos(n.red, p.origen, c)] $\land$ ($\forall$ c1, c2 : compu) estanConectadas(n.red, c1,c2) $\rightarrow$ estaDefinida?(n.proximaEnCamino, c1) $\yluego$ estaDefinida?(definicion(n.proximaEnCamino, c1),c2) $\yluego$ ($\exists$ sc : secu(compu)) sc $\in$ caminosMinimos(n.red, c1, c2) $\land$ definicion(definicion(n.proximaEnCamino,c1),c2) = primero(fin(sc)) $\land$ (n.laQueMasEnvio).cualCompu $\in$ claves(paquetes) $\yluego$ definicion(n.paquetes, (n.laQueMasEnvio).cualCompu).cantEnviados = (n.laQueMasEnvio).cuantosEnvio $\land$ ($\forall$ c : compu) c $\in$ claves(n.paquetes) $\rightarrow$ (n.laQueMasEnvio).cuantosEnvio  $\geq$ definicion(n.paquetes, c).cantEnviados $\yluego$ sinrepetidos(definicion(paquetes, c).colas)         } \mbox{}

 
  \AbsFc[net]{dcnet}[n]{dnt: dcnet | red(dnt) = n.red  $\land$ (($\forall$ p : paquete) (paqueteEnTransito?(dnt,p) $\Rightarrow$ (($\exists$ c : compu) p $\in$ $\pi_2$(definicion(n, c))) $\yluego$
 ($\forall$ p : paquete) ($\exists$ c : compu) c in computadoras(dnt) $\Rightarrow$ siguiente(definicion($\pi_2$(definicion(n.paquetes,c)),p) = caminorecorrido(c,p)  $\yluego$
($\forall$ c : compu) (c $\in$ computadoras(red(dcn)) $\Rightarrow$ (cantidadEnviados(dcn, c) = $\pi_3$(definicion(n.paquetes, c) $\yluego$ 
($\forall$ c : compu) c $\in$ computadoras(dnt) $\Rightarrow$ enEspera(dnt, c) = $\pi_2$(definicion(n.paquetes,c))}

\end{Representacion}



\begin{Algoritmos}


\begin{algorithm}
\caption{Iniciar DCNet}
\begin{algorithmic}[1]
  \Procedure{IniciarDCNet(\textbf{in} $r$ : red) $\to res$ : dcnet}{}
  \State $red \gets r$ \Comment $\Ogr(1)$
  \State $caminosRecorridos \gets$ Vacia() \Comment $\Ogr(1)$ 
  \State $laQueMasEnvio \gets$ (0, NULL) \Comment $\Ogr(1)$
  \State conj(compu) $compus \gets$ Computadoras($red$) \Comment $\Ogr(1)$
   \State itConj $it \gets$ CrearIt($compus$) \Comment $\Ogr(1)$
  \State $proximaEnCamino \gets$ Vacio() \Comment $\Ogr(1)$
  \While{HaySiguiente?(it)} \Comment $\Ogr(1)$
    \State \tadNombre{itConj} $it2 \gets$ CrearIt($compus$) \Comment $\Ogr(1)$
    \State diccTrie(compu, puntero(compu)) $diccActual \gets$ Vacio() \Comment $\Ogr(1)$
	\State definir(paquetes, siguiente(it), tupla(vacio(), vacio(), vacio(), 0)) \Comment $\Ogr(log \ n)$
    \While{HaySiguiente?(it2)} \Comment $\Ogr(n)$
      \State conj(lista(compu)) $camMinimos \gets$ CaminosMinimos(red, SiguienteClave(it), SiguienteClave(it2)) \Comment $\Ogr(1)$
      \State itConj $it3 \gets$ CrearIt(camMinimos) \Comment $\Ogr(1)$
      \State $caminoMinimo \gets$ siguiente(it3)
      \State Fin(caminoMinimo)
      \State puntero(compu) $siguiente \gets$ \& Primero($caminoMinimo$)
      \State Definir($diccActual$, SiguienteClave(it2), id($siguiente$))
      \State Avanzar(it2)            
    \EndWhile
    \State definir($proximaEnCamino$, id(SiguienteClave($it$), $diccActual$)))
    \State Avanzar($it$)
  \EndWhile
   
  \EndProcedure
\end{algorithmic}
\end{algorithm}


\begin{algorithm}
\caption{Crear Paquete}
\begin{algorithmic}[1]
  \Procedure{CrearPaquete(\textbf{in/out} $s$ : dcnet, \textbf{in} $p$ : paquete)}{}
   \If {p.origen != p.destino}
   \State lista(compu) $nuevoCaminoRecorrido \gets$ Vacio()
   \State AgregarAtras(nuevoCaminoRecorrido, origen(p))
   \State itLista(lista(compu)) $it \gets$ AgregarAtras(caminosRecorridos, nuevoCaminoRecorrido)
   \State $losPaquetes \gets $ Obtener($paquetes$, p.origen)
   \State Encolar(losPaquetes.cola, $p$)
   \State Agregar(losPaquetes.conjPaquetes, $p$)
   \State Definir(diccPaqCamino, $p$, it)
   \EndIf
   \EndProcedure
\end{algorithmic}
\end{algorithm}


\begin{algorithm}
\caption{Avanzar Segundo}
\begin{algorithmic}[1]
  \Procedure{AvanzarSegundo(\textbf{in/out} $s$ : dcnet)}{}
  
  \State lista(tupla(paq, itLista(lista(id)))) $paquetesAEnviar \gets$ Vacio()
  
  \State itDiccTrie $it \gets$ CrearIt($paquetes$)
  \While{HaySiguientes?(it)} \Comment $\Ogr(n)$
  	\State infoCompu $s \gets$ SiguienteSignificado($it$) 
  	\If{$\neg$Vacía?($s$.$cola$)}
  		\State paquete $estePaquete \gets$ Desencolar($s$.$cola$)
  		\State Eliminar($s$.conjPaquetes, $estepaquete$)
		\State AgregarAtras(paquetesAEnviar(tupla($estePaquete$, definicion(s.diccPaqCamino, $estePaquete$))))
		\State Eliminar(s.diccPaqCamino, $estePaquete$) 
		\State cantEnviados++
		\If{cantEnviados > $s$.laQueMasEnvio.cuantosEnvio}
		   \State $laQueMasEnvio.cuantosEnvio \gets$ $s$.cantEnviados
		   \State $laQueMasEnvio.cualCompu \gets$ ultimo(Siguiente($pi_1$($estePaquete$)))
		\EndIf
	\EndIf
	\State Avanzar($it$)
  \EndWhile 
   
  \State itLista(tupla(paq, itLista(lista(compu)))) $it_2 \gets$ CrearIt($paquetesAEnviar$)
  \While{HaySiguientes?($it_2$)}
      \State tupla(paq, itLista(lista(compu)) $p \gets$ Siguiente($it_2$)
      \State compu $proximaCompu \gets$ Obtener(Obtener($proximaEnCamino$, $Ultimo(Siguiente(\pi_1(p)))$), destino($p$))
      \If {proximaCompu != $\pi_0(p)$}
	\State $paquetesDeProximaCompu \gets$ Obtener($paquetes$, proximaCompu)
      	\State AgregarAtras(Siguiente($\pi_1(p)$), proximaCompu)
      	\State Encolar($paquetesDeProximaCompu$.$cola$, $\pi_0(p)$)
      	\State Agregar($paquetesDeProximacompu$.$conjPaquetes$, $\pi_0$(p))
	\State Definir($paquetesDeProximacompu.diccPaqCamino$,$\pi_0$(p),$\pi_1$(p))      
      \EndIf
      \State Avanzar($it_2$)
  \EndWhile
 
 \EndProcedure
\end{algorithmic}
\end{algorithm}



\begin{algorithm}
\caption{Red}
\begin{algorithmic}[1]
 \Procedure{red(\textbf{in} $s$ : dcnet)}{}
 \State res $\gets$ red
 \EndProcedure
\end{algorithmic}
\end{algorithm}

\begin{algorithm}
\caption{Camino Recorrido}
\begin{algorithmic}[1]
  \Procedure{caminoRecorrido(\textbf{in} $s$ : dcnet, \textbf{in} $p$ : paquete) $\to res$ : secu(compu)}{}
   \State itDiccTrie $it$ $\gets$ CrearIt(paquetes)
   \While{$\neg$($p \in$ enEspera(siguiente(it)))} 
   \State avanzar(it)
   \EndWhile
   \State res $\gets$ siguiente(Obtener($\pi_1$(Obtener(paquetes,siguiente(it)),p))
  \EndProcedure
\end{algorithmic}
\end{algorithm}


\begin{algorithm}
\caption{Cantidad Enviados}
\begin{algorithmic}[1]
  \Procedure{cantidadEnviados(\textbf{in} $s$ : dcnet, \textbf{in} $c$ : compu)  $\to res$ : nat}{}
   \State res $\gets$ $\pi_3$(Obtener(paquetes, c))
  \EndProcedure
\end{algorithmic}
\end{algorithm}


\begin{algorithm}
\caption{Paquete En Transito}
\begin{algorithmic}[1]
  \Procedure{paqueteEnTransito(\textbf{in} $s$ : dcnet, \textbf{in} $p$ : paquete)  $\to res$ : bool}{}
   \State itDiccTrie $it$ $\gets$ CrearIt(paquetes)
   \While{HaySiguiente?(it) $\yluego$ $\neg$($p \in$ enEspera(siguiente(it)))} 
   \State avanzar(it)
	\EndWhile
	\If {HaySiguiente?(it)}
	\State res $\gets$   True
	\Else 
	\State res $\gets$ False
	\EndIf 
  \EndProcedure
\end{algorithmic}
\end{algorithm}




\begin{algorithm}
\caption{En Espera}
\begin{algorithmic}[1]
  \Procedure{enEspera(\textbf{in} $s$ : dcnet, \textbf{in} $c$ : compu)  $\to res$ : conj(compu)}{}
    \State res $\gets$ $\pi_2(Obtener(\pi_1(net),c))$ \Comment $\Ogr(1)+\Ogr(1)+\Ogr(L)+\Ogr(1)$
  \EndProcedure
  \underline{Complejidad:} $\Ogr(L)$
 \underline{Justificación:} $\Ogr(1)+\Ogr(1)+\Ogr(L)+\Ogr(1) = \Ogr(L)$

\end{algorithmic}
\end{algorithm}


\begin{algorithm}
\caption{La Que Más Envio}
\begin{algorithmic}[1]
  \Procedure{laQueMasEnvio(\textbf{in} $s$ : dcnet)  $\to res$ : compu}{}     
   \State res $\gets$   $\pi_1(\pi_3(net))$ \Comment $\Ogr(1)$
  \EndProcedure
  \underline{Complejidad:} $\Ogr(1)$
\end{algorithmic}
\end{algorithm}



\end{Algoritmos}



\clearpage

\end{document}
