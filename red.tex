\begin{Interfaz}
  
  \textbf{se explica con}: \tadNombre{Red}.

  \textbf{géneros}: \TipoVariable{red}.
  
  \section*{Operaciones básicas de \tadNombre{Red}}

  \InterfazFuncion{IniciarRed}{}{red}
  {$res \igobs iniciarRed()$}
  [$\Ogr(1)$]
  [genera la red vacía]

  \InterfazFuncion{AgregarComputadora}{\Inout{r}{\tadNombre{Red}}, \In{c}{\tadNombre{Compu}}, \In{is}{\tadNombre{conj(interfaz)}}}{}
  [$r \igobs r_0 \land (\forall c' : compu) (c' \in computadoras(r) \impluego ip(c) \neq ip(c'))$] 
  {$computadoras(r) = Ag(c, computadoras(r_0)) \yluego \\
     (\forall c_1, c_2 : compu) (c_1 \in computadoras(r_0) \land c_2 \in computadoras(r_0) \impluego \\ 
   conectadas(r_0, c_1, c_2) = conectadas(r, c_1, c_2) \land interfazUsada(r_0, c_1, c_2) = interfazUsada(r, c_1, c_2)$)}
  [$\Ogr(L)$]
  [agrega el la computadora $c$ a la red $r$.]
  [el elemento $c$ agrega por copia.]
  
  
  \InterfazFuncion{conectar}{\Inout{r}{\tadNombre{Red}}, \In{c_1}{\tadNombre{Compu}},  \In{i_1}{\tadNombre{Interfaz}}, \In{c_2}{\tadNombre{Interfaz}}, \In{i_2}{\tadNombre{Interfaz}}}{}
 [$r \igobs r_0 \land c_1 \in computadoras(r) \land c_2 \in computadoras(r) \land ip(c_1) \neq ip(c_2) \land \neg conectadas?(r, c_1, c_2) \land \neg usaInterfaz?(r,c_1, i_1) \land \neg usaInterfaz?(r, c_2, i_2)$]
  {$computadoras(r) = computadoras(r_0) \yluego \\
    (\forall c'_1, c'_2 : compu) (c'_1 \in computadoras(r_0) \land c'_2 \in computadoras(r_0) \land \{c_1, c_2\} \neq \{c'_1, c'_2\} \impluego \\ 
   conectadas(r_0, c'_1, c'_2) = conectadas(r, c'_1, c'_2) \land interfazUsada(r_0, c'_1, c'_2) = interfazUsada(r, c'_1, c'_2)) \land \\
   conectadas(r_0, c_1, c_2) \land interfazUsada(r, c_1, c_2) = i_1 \land interfazUsada(r, c_2, c_1) = i_2$}
  [$\Ogr(L)$]
  [conecta las computadoras $c_1$ y $c_2$ mediante las interfaces $i_1$ e $i_2$ respectivamente en la red $r$.]


  \InterfazFuncion{computadoras}{\In{r}{\tadNombre{Red}}}{\tadNombre{conj(compu)}}
  {$alias(res \igobs computadoras(r))$}
  [$\Ogr(n)$]
  [devuelve un conjunto con todas las computadoras de la red $r$.]
  [$res$ no es modificable.]


  \InterfazFuncion{conectadas?}{\In{r}{\tadNombre{Red}}, \In{c_1}{\tadNombre{Compu}}, \In{c_2}{\tadNombre{Compu}}}{\tadNombre{bool}}
  [$c_1 \in computadoras(r) \land c_2 \in computadoras(r)$]
  {$res \igobs conectadas?(r,c_1,c_2)$}
  [$\Ogr(L)$]
  [dice si las computadoras $c_1$ y $c_2$ están conectadas en la red $r$]


  \InterfazFuncion{interfazUsada}{\In{r}{\tadNombre{Red}}, \In{c_1}{\tadNombre{Compu}}, \In{c_2}{\tadNombre{Compu}}}{\tadNombre{bool}}
  [$conectadas?(r,c_1,c_2)$]
  {$res \igobs interfazUsada(r,c_1,c_2)$}
  [$\Ogr(L)$]
  [devuelve la interfaz mediante la cual $c_1$ se conecta a $c_2$  en la red $r$.]

	\newpage
  \InterfazFuncion{vecinos}{\In{r}{\tadNombre{Red}}, \In{c}{\tadNombre{Compu}}}{\tadNombre{conj(compu)}}
  [$c \in computadoras(r)$]
  {$alias(res \igobs vecinos(r,c))$}
  [$\Ogr(n+L)$]
  [devuelve las computadoras conectadas a $c$ en la red $r$.]
  [$res$ no es modificable.]


  \InterfazFuncion{usaInterfaz?}{\In{r}{\tadNombre{Red}}, \In{c}{\tadNombre{Compu}}, \In{i}{\tadNombre{Interfaz}}}{\tadNombre{bool}}
  [$c \in computadoras(r)$]
  {$res = usaInterfaz?(r,c,i)$}
  [$\Ogr(nL)$]
  [devuelve true si y solo si $c$ está conectada a otra pc mediante la interfaz $i$ en la red $r$.]


  \InterfazFuncion{caminosMinimos}{\In{r}{\tadNombre{Red}}, \In{c_1}{\tadNombre{Compu}}, \In{c_2}{\tadNombre{Compu}}}{\tadNombre{conj(secu(compu))}}
  [$c_1 \in computadoras(r) \land c_2 \in computadoras(r)$]
  {$res = caminosMinimos(r,c_1,c_2)$}
  [$\Ogr(n \cdot n^2!)$]
  [devuelve un conjunto con los caminos más cortos entre las computadoras $c_1$ y $c_2$ en la red $r$.]

%%

  \InterfazFuncion{hayCamino?}{\In{r}{\tadNombre{Red}}, \In{c_1}{\tadNombre{Compu}}, \In{c_2}{\tadNombre{Compu}}}{\tadNombre{bool}}
  [$c_1 \in computadoras(r) \land c_2 \in computadoras(r)$]
  {$res = hayCamino?(r,c_1,c_2)$} 
  [$\Ogr(n \cdot n^2!)$]
  [devuelve true si y solo si hay un camino entre las computadoras $c_1$ y $c_2$ en la red $r$.]



\end{Interfaz}

\begin{Representacion}
  
  
  
  \section*{Representación de la red}

  \begin{Estructura}{red}[redEstr]
    \begin{Tupla}[topo]
      \tupItem{vecinos}{diccTrie(compu, diccTrie(compu, interfaz))}
      \tupItem{\\ interfaces}{diccTrie(compu, conj(interfaces))}%
    \end{Tupla}

  \end{Estructura}

  \Rep[redEstr][e]{claves(e.vecinos) $=$ claves(e.compus) $\yluego$ \\
  ($\forall i_1, i_2$ : id) def?($i_2$, (obtener($i_1$, e.vecinos))) $\impluego$ def?($i_2$, e.vecinos) $\yluego$ def?($i_1$, obtener($i_2$, e.vecinos)) $\yluego$ \\
obtener($i_2$, obtener($i_1$, e.vecinos)) $\in$  interfaces(obtener($i_1$, e.compus))
$\land$\\ 
($\forall i, i_1, i_2$ : id) def?($i$, e.vecinos) $\impluego$ \\
(def?($i_1$, obtener($i$, e.vecinos)) $\land$ def?($i_2$, obtener($i$, e.vecinos)) $\land$ $i_1 \neq i_2$ $\impluego$ obtener($i_1$, obtener($i$, e.vecinos)) $\neq$ obtener($i_2$, obtener($i$, e.vecinos)))

}\mbox{}
  
   
   \AbsFc[redEstr]{red}[e]{
     r : red / computadoras(r) = significados(e.compus) $\land$ \\ 
     ($\forall c_1, c_2$ : computadora) conectadas($r, c_1, c_2$) = def?(id($c_1$), obtener(id($c_2$), e.vecinos)) $\land$ \\
     ($\forall c_1, c_2$: computadora) ($\forall i$ : interfaz) conectadas($r,c_1,c_2$) $\impluego$ interfazUsada($r, c_1, c_2$) = obtener(id($c_1$), obtener(id($c_2$), e.vecinos))
  }

\end{Representacion}

\newpage

\begin{Algoritmos}


\begin{algorithm}
  \caption{Algoritmos de \tadNombre{RED}}
\begin{algorithmic}[1]
  \Procedure{iIniciarRed}{\ } $\to$ $res$ : \texttt{redEstr}
  \State vecinos $\gets$ Vacia() \Comment $\Ogr(1)$
  \State interfaces $\gets$ Vacia() \Comment $\Ogr(1)$
  \State $res \gets$ (vecinos, interfaces) \Comment $\Ogr(1)$
  \EndProcedure
\end{algorithmic}
  \underline{Complejidad:} $\Ogr(1)$
\end{algorithm}


\begin{algorithm}
\caption{Agregar Computadora}
\begin{algorithmic}[1]
  \Procedure{iAgregarComputadora}{\texttt{in/out} $r$ : \texttt{redEstr}, \texttt{in} $c$ : \texttt{compu}, \texttt{in} $is$ : \texttt{conj(interfaz)}}
  \State definir(r.vecinos, $c$, Vacia()) \Comment $\Ogr(L)$
  \State definir(r.interfaces, $c$, $is$) \Comment $\Ogr(L)$
 \EndProcedure
\end{algorithmic}
	\underline{Complejidad:} $\Ogr(L)$
\end{algorithm}


\begin{algorithm}
\caption{Conectar}
\begin{algorithmic}[1]
  \Procedure{iConectar}{\texttt{in/out} $r$ : \texttt{redEstr}, \texttt{in} $c_1$ : \texttt{compu}, \texttt{in} $c_2$ :  \texttt{compu}, \texttt{in} $i_1$ : \texttt{interfaz}, \texttt{in} $i_2$ : \texttt{interfaz}}{}
  \State $x \gets$ obtener(r.vecinos, $c_1$) \Comment $\Ogr(L)$
  \State definir($x$, $c_2$, $i_1$) \Comment $\Ogr(L)$
  \State $x \gets$ obtener(r.vecinos, $c_2$) \Comment $\Ogr(L)$
  \State definir($x$, $c_1$, $i_2$) \Comment $\Ogr(L)$
 \EndProcedure
\end{algorithmic}
 \underline{Complejidad:} $\Ogr(L)$
\end{algorithm}



\begin{algorithm}
\caption{Computadoras}
\begin{algorithmic}[1]
  \Procedure{iComputadoras}{\texttt{in} $r$ : \texttt{redEstr}} $\to res$ : \texttt{conj(compu)} 
  \State $res \gets$ Claves($interfaces$) \Comment $\Ogr(n)$
 \EndProcedure
\end{algorithmic}
	\underline{Complejidad:} $\Ogr(n)$
\end{algorithm}


\begin{algorithm}
\caption{Conectadas?}
\begin{algorithmic}[1]
  \Procedure{iConectadas?}{\texttt{in} $r$ : \texttt{redEstr}, \texttt{in} $c_1$ : \texttt{compu}, \texttt{in} $c_2$ : \texttt{compu}}  $\to res$ : \texttt{bool}
  \State $x \gets$ Obtener($c_1$, $r$.vecinos) \Comment $\Ogr(L)$
  \State $res \gets$ def?($c_2$, $x$) \Comment $\Ogr(L)$
 \EndProcedure
\end{algorithmic}
	 \underline{Complejidad:} $\Ogr(L)$
\end{algorithm}


\begin{algorithm}
\caption{Interfaz Usada}
\begin{algorithmic}[1]
  \Procedure{iInterfazUsada}{\texttt{in} $r$ : \texttt{redEstr}, \texttt{in} $c_1$ : \texttt{compu}, \texttt{in} $c_2$ : \texttt{compu}} $\to res$ : \texttt{interfaz}
  \State $x \gets$ Obtener($c_1$, $r$.vecinos) \Comment $\Ogr(L)$
  \State $res \gets$ Obtener($c_2$, $x$) \Comment $\Ogr(L)$
 \EndProcedure
\end{algorithmic}
 \underline{Complejidad:} $\Ogr(L)$
\end{algorithm}


\begin{algorithm}
\caption{Vecinos}
\begin{algorithmic}[1]
  \Procedure{iVecinos}{\texttt{in} $r$ : \texttt{redEstr}, \texttt{in} $c$ :\texttt{compu}}  $\to res$ : \texttt{conj(compu)}
  \State $x \gets$ Obtener($c$, $r$.vecinos) \Comment $\Ogr(L)$
  \State $res \gets$ Claves($x$) \Comment $\Ogr(n)$
 \EndProcedure
\end{algorithmic}
 \underline{Complejidad:} $\Ogr(n+L)$
\end{algorithm}



\begin{algorithm}
\caption{Usa Interfaz?}
\begin{algorithmic}[1]
  \Procedure{iUsaInterfaz?}{\texttt{in} $r$ : \texttt{redEstr}, \texttt{in} $c$ : \texttt{compu},  \texttt{in} $i$ : interfaz} $\to res$ : \texttt{bool} 
  \State $x \gets$ Obtener($c$, $r$.vecinos) \Comment $\Ogr(L)$
  \State $i :\tadNombre{itdicc} \gets$ NuevoIterador($x$) \Comment $\Ogr(n)$
  \State $res \gets$ False \Comment $\Ogr(1)$
  \While{haySiguiente($i$)} \Comment Se repite $\Ogr(n)$ veces
    \If {Obtener(Siguiente($i$), $x$) = $i$}  \Comment $\Ogr(L)$
       \State $res \gets$ True  \Comment $\Ogr(1)$
    \EndIf
    \State Avanzar($i$) \Comment $\Ogr(1)$
  \EndWhile
 \EndProcedure
\end{algorithmic}
 \underline{Complejidad:} $\Ogr(nL)$\\
 \underline{Justificación:} $\Ogr(L) + \Ogr(n) + \Ogr(1) + \Ogr(n)(\Ogr(L) + \Ogr(1)) = \Ogr(n)\Ogr(L) = \Ogr(nL)$
\end{algorithm}


\begin{algorithm}
\caption{Caminos Minimos}
\begin{algorithmic}[1]
  \Procedure{iCaminosMinimos}{\texttt{in} $r$ : \texttt{redEstr}, \textbf{in} $c_1$ : \texttt{compu},  \texttt{in} $c_2$ : \texttt{compu}} $\to res$ : \texttt{conj(secu(compu))}  
   \State $k \gets$ 1 \Comment $\Ogr(1)$
   \State $n \gets$ Cardinal(Computadoras($r$))  \Comment $\Ogr(n)$
   \While{$k<n$ $\land$ esVacio?(CaminosDeLargoN($r$, $c_1$, $c_2$, k))} \Comment Se repite $\Ogr(n)$ veces y cuesta  $\Ogr(n(L + n^2!))$
     \State k++ \Comment $\Ogr(1)$
   \EndWhile
   \State $res \gets$ CaminosDeLargoN($r$, $c_1$, $c_2$, k) \Comment $\Ogr(n(L + n^2!))$
\EndProcedure
\end{algorithmic}
\underline{Complejidad:} $\Ogr(n^2 (L + n^2!))$ \\ 
\underline{Justificacion:} $\Ogr(1) + \Ogr(n) + \Ogr(n)\Ogr(n(L + n^2!)) + \Ogr(n(L + n^2!)) = \Ogr(n)\Ogr(n(L + n^2!)) = \Ogr(n^2(L + n^2!))$
\end{algorithm}


\begin{algorithm}
\caption{Hay Camino?}
\begin{algorithmic}[1]
  \Procedure{iHayCamino?}{\texttt{in} $r$ : \texttt{redEstr}, \texttt{in} $c_1$ : \texttt{compu},  \texttt{in} $c_2$ : \texttt{compu}} $\to res$ : \texttt{bool} 
   \State $res \gets$ $\neg$ esVacio?(CaminosMinimos($r$,$c_1$, $c_2$)) \Comment  $\Ogr(n^2 (L + n^2!))$ 
\EndProcedure
\end{algorithmic}
\underline{Complejidad:} $\Ogr(n^2 (L + n^2!))$
\end{algorithm}

\begin{algorithm}
\caption{Caminos De Largo N }
\begin{algorithmic}[1]
  \Procedure{iCaminosDeLargoN}{\texttt{in} $r$ : \texttt{redEstr}, \textbf{in} $c_1$ : \texttt{compu},  \texttt{in} $c_2$ : \texttt{compu}, \texttt{in} $n$ : \texttt{nat}} $\to res$ : \texttt{conj(secu(compu))} 
   \State conj(secu(compu)) $caminos \gets$ Vacio() \Comment $\Ogr(1)$
   \If{n = 0}
     \State secu(compus) $camino \gets$ Vacio \Comment $\Ogr(1)$
     \State AgregarAtras($camino$, $c_1$) \Comment $\Ogr(1)$
     \State Agregar($camino$, $caminos$) \Comment $\Ogr(1)$
   \Else
     \State $vec \gets$ vecinos($c_1$) \Comment $\Ogr(n+L)$
     \State $itVecinos \gets$ CrearIt($vec$) \Comment $\Ogr(1)$
     \While{haySiguiente?($itVecinos$)} \Comment Se va a repetir $\Ogr(n)$ veces
       \State $v \gets$ Siguiente($itVecinos$) \Comment $\Ogr(1)$
       \State $cams \gets$ CaminosDeLargoN($r$, $v$, $c_2$, $n-1$) \Comment $T(n-1)$
       \State $itCaminos \gets$ CreatIt($cams$) \Comment $\Ogr(1)$
       \While{haySiguiente?($itCaminos$)}  \Comment Se va a repetir $\Ogr(|caminos|) \subseteq \Ogr(n^2!)$ veces \footnotesize{Si el grafo de computadoras fuera completo (alcanzando asi la maxima cantidad de links posibles, que es $\frac{n(n-1)}{2})$ es claro que se puede elegir $\frac{n(n-1)}{2}!$ caminos distintos, por combinatoria} \normalsize
         \State $camino \gets$ Siguiente($itCaminos$) \Comment $\Ogr(1)$
         \If{Ultimo($camino$) = $c_2$} \Comment $\Ogr(1)$
           \State $camino_2 \gets$ camino \Comment $\Ogr(1)$
           \State AgregarAdelante($c_1$, $camino_2$) \Comment $\Ogr(1)$
           \State Agregar ($camino_2$, $caminos$) \Comment $\Ogr(1)$
         \EndIf
       \State Avanzar($itCaminos$) \Comment $\Ogr(1)$
       \EndWhile
    \State Avanzar($itVecinos$) \Comment $\Ogr(1)$
    \EndWhile    
   \EndIf
   
   \State $itCaminos \gets$ crearIt($caminos$) \Comment $\Ogr(1)$
   \While{haySiguiente?($itCaminos$)} \Comment Se va a repetir $\Ogr(n)$ veces
     \If{ultimo(Siguiente($itCaminos$)) $\neq$ $c_2$} \Comment $\Ogr(1)$
       \State EliminarSiguiente($itCaminos$) \Comment $\Ogr(1)$
      \EndIf
    \EndWhile
    \State $res \gets caminos$ \Comment $\Ogr(1)$

 \EndProcedure
\end{algorithmic}
\underline{Complejidad:} $\Ogr(n(L + n^2!))$ \\
\underline{Justificación:} $T(0) = \Ogr(1), T(n) = \Ogr(n+L) + \Ogr(n) \cdot (T(n-1) + \Ogr(n^2!)) + \Ogr(n)$ \\
Por lo tanto, $T(n) = \Ogr(n+L) + \Ogr(n)T(n-1) +  \Ogr(n \cdot n^2!)$ \\
Entonces, $T(n) =  \Ogr(n+L) + \Ogr(n)(\Ogr(n-1+L) + \Ogr(n-1)T(n-2) +  \Ogr((n-1) \cdot (n-1)^2!)) +  \Ogr(n \cdot n^2!) = \Ogr(n+L) + \Ogr(n(n-1)+nL) + \Ogr(n(n-1))T(n-2) +  \Ogr(n \cdot (n-1) \cdot (n-1)^2!) +  \Ogr(n \cdot n^2!) = \Ogr(nL) + \Ogr(n(n-1))T(n-2) +  \Ogr(n \cdot n^2!) = ... = \Ogr(nL) + \Ogr(n(n-1)(n-2) \cdot ... \cdot 1)T(0) +  \Ogr(n \cdot n^2!) = \Ogr(n(L + n^2!)) $
\end{algorithm}


\end{Algoritmos}







