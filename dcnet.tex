\begin{Interfaz}
  
  \textbf{se explica con}: \tadNombre{dcnet}.

  \textbf{géneros}: \TipoVariable{dcnet}.

  \section*{Operaciones básicas de DCNet}

  \InterfazFuncion{IniciarDCNet}{\In{r}{\tadNombre{red}}}{\tadNombre{dcnet}}
  [$\neg\emptyset?(computadoras(r))$]
  {$ res \igobs IniciarDCNet(r)  $ }
  [$\Ogr(n^4(L + n^2!))$]
  [Genera una DCNet con las computadoras y conexiones de la red pasada como parámetro.]

  \InterfazFuncion{CrearPaquete}{\Inout{s}{\tadNombre{dcnet}}, \In{p}{\tadNombre{paquete}}}{}
  [$s \igobs s_0 \land 
   \neg((\exists p':paquete) (paqueteEnTransito?(s_0,p') \land id(p') \igobs id(p))) \land\\ 
   \hspace*{3em} origen(p) \in computadoras(red(s_0)) \yluego
   destino(p) \in computadoras(red(s_0)) \yluego\\
   \hspace*{3em} hayCamino?(red(s), origen(p), destino(p))$]
  {$s \igobs crearPaquete(s_0, p)$}
  [$\Ogr(L + log(k))$]
  [Agrega el paquete $p$ a la cola de la computadora $p.origen$.]
  
  
  \InterfazFuncion{AvanzarSegundo}{\Inout{s}{\tadNombre{dcnet}}}{}
  [$s \igobs s_0$]  
  {$res \igobs avanzarSegundo(s_0)$}
  [$\Ogr(n \times (L + log(k)))$]
  [Avanza un segundo; todas las computadoras que tengan paquetes por enviar los envían.]
  
  
  \InterfazFuncion{red}{\In{s}{\tadNombre{dcnet}}}{\tadNombre{red}}
  [$ True $]  
  {$alias(res \igobs red(s_0))$}
  [$\Ogr(1)$]
  [Expresa la red que esta contenida en la DCNet.]
  [$res$ no es modificable]

  
  \InterfazFuncion{caminoRecorrido}{\In{s}{\tadNombre{dcnet}}, \In{p}{\tadNombre{paquete}}}{\tadNombre{secu(compu)}}
  [$ paqueteEnTransito?(s,p) $]
  {$alias(res \igobs caminoRecorrido(s, p))$}
  [$\Ogr(nlog(k))$]
  [Expresa el camino de computadoras recorrido por el paquete $p$ desde su inicio hasta posicion actual.]
	[$res$ no es modificable]

  \InterfazFuncion{cantidadEnviados}{\In{s}{\tadNombre{dcnet}}, \In{c}{\tadNombre{compu}}}{\tadNombre{nat}}
  [$ c \in computadoras(red(s)) $]  
  {$res \igobs cantidadEnviados(s, c)$}
  [$\Ogr(L)$]
  [Expresa la cantidad de mensajes enviados por la compu $c$.]

  
  \InterfazFuncion{enEspera}{\In{s}{\tadNombre{dcnet}}, \In{c}{\tadNombre{compu}}}{\tadNombre{conj(paquete)}}
  [$ c \in computadoras(red(s)) $]  
  {$alias(res \igobs enEspera(s, c))$}
  [$\Ogr(L)$]
  [Expresa el camino de computadoras recorrido por el paquete $p$ desde su inicio hasta posicion actual.]
	[$res$ no es modificable]

  \InterfazFuncion{PaqueteEnTransito}{\In{s}{\tadNombre{dcnet}}, \In{p}{\tadNombre{paquete}}}{\tadNombre{bool}}
  [$ True $]  
  {$alias(res \igobs paqueteEnTransito?(s, p))$}
  [$\Ogr(nlog(k))$]
  [Expresa si el paquete $p$ esta en alguna computadora.]
  [$res$ no es modificable]
  
  \InterfazFuncion{LaQueMasEnvio}{\In{s}{\tadNombre{dcnet}}}{\tadNombre{compu}}
  [$ True $]
  {$alis(res \igobs laQuemasEnvio(s))$}
  [$\Ogr(1)$]
  [Devuelve la computadora que mas paquetes envio.]
  [$res$ no es modificable]

\end{Interfaz}

\begin{Representacion}
  
  \section*{Representación de DCNet}

  \begin{Estructura}{dcnet}[net]
    \begin{Tupla}[net]
      \tupItem{proximaEnCamino}{diccTrie(compu, diccTrie(compu, compu))}%
      \tupItem{\\ paquetes}{diccTrie(compu,infoPaquetes)}%
      \tupItem{\\ caminosRecorridos}{lista(lista(compu))}
      \tupItem{\\ laQueMasEnvio}{tupla($cuantosEnvio$ : nat, $cualCompu$ : compu)}
      \tupItem{\\ red}{red}
    \end{Tupla}
  
  
  \begin{Tupla}[infoPaquetes]
      \tupItem{colas}{colaPrior($p$ : paq)}
      \tupItem{\\ diccPaqCamino}{diccAVL($p$ : paq, $camino$ : itLista )}
      \tupItem{\\ conjPaquetes}{conjAVL(paq)}
      \tupItem{\\ cantEnviados}{nat}                       
    \end{Tupla}
   \end{Estructura}
 $ \newline$
  Invariante de Representacion en palabras: $\newline$
  $\newline$
  1) Las computadoras del diccionario $paquetes$ estan en la red y viceversa $\newline$
  2) Los paquetes que estan en infopaquete son los mismos, tanto en diccPaqCamino, en conjPaquetes y en colas. $\newline$
  3) En caminos de diccPaqCamino hay un iterador que te da el camino que hizo el paquete $\newline$
  desde su origen hasta la compu actual $\newline$
  4) En colas de infopaquete de cualquier computadora nunca van a haber dos paquetes con la misma id y distinto origen o destino. $\newline$ 
  5) Si dos compus estan conectadas (una es de origen y otra de destino) entonces ingresando esas dos en proximaEnCamino podes averiguar a que compu ir ahora para lograr hacer un paso mas en tu camino hacia la compu de destino. $\newline$
  6) En la que mas envio esta una computadora que esta en la dcnet y que tiene efectivamente una cantidad mayor o igual de paquetes enviados que cualquier otra. $\newline$
  7) No hay colas con paquetes repetidos $\newline$
  8) No hay paquetes con la misma id en distintas computadoras $\newline$
  
  \Rep[net][n]{computadoras(n.red) = claves(n.paquetes) $\yluego$ \\
  									($\forall$ c : compu) [((c $\in$ claves(n.paquetes) $\Rightarrow$  definicion(n.paquetes, c).conjPaquetes = elementos(definicion(n.paquetes,c).colas) $\land$ definicion(n.paquetes, c).conjPaquetes) = claves(definicion(paquetes, c).diccPaqCamino) $\land$  \\
  									\hspace*{2em}(($\forall$ p : paquete) (p $\in$ claves(definicion(paquetes, c).diccPaqCamino))  $\rightarrow$ siguiente(definicion(definicion(n.paquetes,c).diccPaqCamino,p).camino) $\in$ caminosMinimos(n.red, p.origen, c)))] $\land$ \\
  									($\forall$ c : compu) [ c $\in$ claves(n.paquetes) $\rightarrow$ ($\forall$ p1, p2 : paquete) (id(p1) = id(p2) $\rightarrow$ $\neg$(p2 $in$ (elementos(definicion(n.paquetes,c).colas) - {p1}))) ] \\ 
  									($\forall$ c1, c2 : compu) (estanConectadas(n.red, c1,c2) $\Rightarrow$ \\
  									 \hspace*{2em} estaDefinida?(n.proximaEnCamino, c1) $\yluego$ \\
  									 \hspace* {2em} estaDefinida?(definicion(n.proximaEnCamino, c1),c2) $\yluego$ \\
  									 \hspace*{2em}	($\exists$ sc : secu(compu)) sc $\in$ caminosMinimos(n.red, c1, c2) $\land$\\
  									 \hspace*{3em} 	definicion(definicion(n.proximaEnCamino,c1),c2) = primero(fin(sc))) $\land$ \\
  									  (n.laQueMasEnvio).cualCompu $\in$ claves(paquetes) $\yluego$ \\
  									  definicion(n.paquetes, (n.laQueMasEnvio).cualCompu).cantEnviados =  (n.laQueMasEnvio).cuantosEnvio $\land$\\
  									   ($\forall$ c : compu) c $\in$ claves(n.paquetes) $\Rightarrow$ (n.laQueMasEnvio).cuantosEnvio  $\geq$ definicion(n.paquetes, c).cantEnviados $\yluego$ sinrepetidos(definicion(paquetes, c).colas)     \\
  									   ($\forall$ c1, c2 : compu) ((c1 $\in$ claves(n.paquetes) $\land$ c2 $\in$ claves(n.paquetes))  $\rightarrow$ ( ($\forall$ p1 : paquete) p1 $\in$ definicion(n.paquetes, c1).conjPaquetes $\rightarrow$ $\neg$($\exists$ p2 : paquete) p2 $\in$ definicion(n.paquetes, c1).conjPaquetes $\land$ id(p2) = id(p1)))} \mbox{}

 
  \AbsFc[net]{dcnet}[n]{dnt: dcnet | red(dnt) = n.red  $\land$ (($\forall$ p : paquete) (paqueteEnTransito?(dnt,p) $\Rightarrow$ (($\exists$ c : compu) p $\in$ $\pi_2$(definicion(n, c))) $\yluego$
 ($\forall$ p : paquete) ($\exists$ c : compu) c in computadoras(dnt) $\Rightarrow$ siguiente(definicion($\pi_2$(definicion(n.paquetes,c)),p) = caminorecorrido(c,p)  $\yluego$
($\forall$ c : compu) (c $\in$ computadoras(red(dcn)) $\Rightarrow$ (cantidadEnviados(dcn, c) = $\pi_3$(definicion(n.paquetes, c) $\yluego$ 
($\forall$ c : compu) c $\in$ computadoras(dnt) $\Rightarrow$ enEspera(dnt, c) = $\pi_2$(definicion(n.paquetes,c))}

\end{Representacion}



\begin{Algoritmos}


\begin{algorithm}
\caption{Iniciar DCNet}
\begin{algorithmic}[1]
  \Procedure{IniciarDCNet(\textbf{in} $r$ : red) $\to res$ : dcnet}{}
  \State $red \gets r$ \Comment $\Ogr(1)$
  \State $caminosRecorridos \gets$ Vacia() \Comment $\Ogr(1)$ 
  \State itConj $it \gets$ CrearIt($compus$) \Comment $\Ogr(1)$
  \State $laQueMasEnvio \gets$ (0, Siguiente($it$)) \Comment $\Ogr(1)$
  \State conj(compu) $compus \gets$ Computadoras($red$) \Comment $\Ogr(n)$
  \State $proximaEnCamino \gets$ Vacio() \Comment $\Ogr(1)$
  \While{HaySiguiente?(it)} \Comment Se repite $\Ogr(n)$ veces
    \State \tadNombre{itConj} $it2 \gets$ CrearIt($compus$) \Comment $\Ogr(1)$
    \State diccTrie(compu, puntero(compu)) $diccActual \gets$ Vacio() \Comment $\Ogr(1)$
	\State \tadNombre{tupla} tupinfopaquete $\gets$ tupla(vacio(), vacio(), vacio(), 0) \Comment $\Ogr(1)$	
	\State definir($paquetes$, siguiente(it), tupinfopaquete) \Comment $\Ogr(L)$
    \While{HaySiguiente?(it2)} \Comment Se repite $\Ogr(n)$ veces
      \State conj(lista(compu)) $camMinimos \gets$ CaminosMinimos(red, SiguienteClave(it), SiguienteClave(it2)) \Comment $\Ogr(n^2(L + n^2!))$
      \State itConj $it3 \gets$ CrearIt(camMinimos) \Comment $\Ogr(1)$
      \State $caminoMinimo \gets$ siguiente(it3) \Comment $\Ogr(1)$
      \State Fin(caminoMinimo) \Comment $\Ogr(1)$
      \State puntero(compu) $siguiente \gets$ \& Primero($caminoMinimo$) \Comment $\Ogr(1)$
      \State Definir($diccActual$, SiguienteClave(it2), id($siguiente$)) \Comment $\Ogr(L)$
      \State Avanzar(it2)\Comment $\Ogr(1)$
    \EndWhile
    \State definir($proximaEnCamino$, SiguienteClave($it$), $diccActual$) \Comment $\Ogr(L)$
    \State Avanzar($it$) \Comment $\Ogr(1)$
  \EndWhile
   
  \EndProcedure 
\end{algorithmic}
\underline{Complejidad:} $\Ogr(n^4(L + n^2!))$ \\
 \underline{Justificación:} $\Ogr(n)\times(\Ogr(L) + \Ogr(n)\times(\Ogr(n^2 (L+n^2!)) + \Ogr(L)) + O(L)) = \Ogr(nL) + \Ogr(n^4(L + n^2!)) =   \Ogr(n^4(L + n^2!))$
\end{algorithm}


\begin{algorithm}
\caption{Crear Paquete}
\begin{algorithmic}[1]
  \Procedure{CrearPaquete(\textbf{in/out} $s$ : dcnet, \textbf{in} $p$ : paquete)}{}
   \If {p.origen != p.destino} \Comment $\Ogr(L)$
   \State lista(compu) $nuevoCaminoRecorrido \gets$ Vacio() \Comment $\Ogr(1)$
   \State AgregarAtras(nuevoCaminoRecorrido, origen(p))\Comment  $\Ogr(1)$
   \State itLista(lista(compu)) $it \gets$ AgregarAtras(caminosRecorridos, nuevoCaminoRecorrido)\Comment $\Ogr(1)$
   \State $losPaquetes \gets $ Obtener($paquetes$, p.origen) \Comment $\Ogr(L)$
   \State Encolar(losPaquetes.cola, $p$) \Comment $\Ogr(log \ k)$
   \State Agregar(losPaquetes.conjPaquetes, $p$) \Comment $\Ogr(log \ k)$
   \State Definir(diccPaqCamino, $p$, it) \Comment $\Ogr(log \ k)$
   \EndIf
   \EndProcedure
\end{algorithmic}
\underline{Complejidad:} $\Ogr(L + log(k))$\\
\underline{Justificación:} $3 \times \Ogr(1)+2 \times \Ogr(L)+3\times \Ogr(log(k)) = \Ogr(L + log(k))$
\end{algorithm}


\begin{algorithm}
\caption{Avanzar Segundo}
\begin{algorithmic}[1]
  \Procedure{AvanzarSegundo(\textbf{in/out} $s$ : dcnet)}{}
  
  \State lista(tupla(paq, itLista(lista(id)))) $paquetesAEnviar \gets$ Vacio() \Comment  $\Ogr(1)$
  \State itDiccTrie $it \gets$ CrearIt($paquetes$) \Comment  $\Ogr(n)$
  \While{HaySiguientes?(it)} \Comment Se repite $\Ogr(n)$  
  	\State infoCompu $s \gets$ SiguienteSignificado($it$) \Comment  $\Ogr(1)$ 
  	\If{$\neg$Vacía?($s$.$cola$)} \Comment  $\Ogr(1)$
  		\State paquete $estePaquete \gets$ Desencolar($s$.$cola$) \Comment  $\Ogr(log(k))$
  		\State Eliminar($s$.conjPaquetes, $estepaquete$) \Comment  $\Ogr(log(k))$
		\State AgregarAtras(paquetesAEnviar(tupla($estePaquete$, definicion(s.diccPaqCamino, $estePaquete$)))) 
		\State\Comment  $\Ogr(1)$
		\State Eliminar(s.diccPaqCamino, $estePaquete$)  \Comment  $\Ogr(log(k))$
		\State cantEnviados++ \Comment  $\Ogr(1)$
		\If{cantEnviados > $s$.laQueMasEnvio.cuantosEnvio} \Comment  $\Ogr(1)$
		   \State $laQueMasEnvio.cuantosEnvio \gets$ $s$.cantEnviados \Comment  $\Ogr(1)$
		   \State $laQueMasEnvio.cualCompu \gets$ ultimo(Siguiente($\pi_1$($estePaquete$))) \Comment  $\Ogr(1)$
		\EndIf 
	\EndIf
	\State Avanzar($it$) \Comment  $\Ogr(1)$
  \EndWhile 
   
  \State itLista(tupla(paq, itLista(lista(compu)))) $it_2 \gets$ CrearIt($paquetesAEnviar$) \Comment  $\Ogr(1)$
  \While{HaySiguientes?($it_2$)} \Comment  $se repite \Ogr(n) veces$
      \State tupla(paq, itLista(lista(compu)) $p \gets$ Siguiente($it_2$) \Comment  $\Ogr(1)$
      \State compu $proximaCompu \gets$ Obtener(Obtener($proximaEnCamino$, $Ultimo(Siguiente(\pi_1(p)))$), destino($p$)) 
      \State \Comment  $\Ogr(L + L)$
      \If {proximaCompu != $\pi_0(p)$} \Comment  $\Ogr(1)$
	\State $paquetesDeProximaCompu \gets$ Obtener($paquetes$, proximaCompu) \Comment  $\Ogr(log(L))$
      	\State AgregarAtras(Siguiente($\pi_1(p)$), proximaCompu) \Comment  $\Ogr(1)$
      	\State Encolar($paquetesDeProximaCompu$.$cola$, $\pi_0(p)$) \Comment  $\Ogr(log(k))$
      	\State Agregar($paquetesDeProximacompu$.$conjPaquetes$, $\pi_0$(p)) \Comment  $\Ogr(log(k))$
	\State Definir($paquetesDeProximacompu.diccPaqCamino$,$\pi_0$(p),$\pi_1$(p)) \Comment  $\Ogr(log(k))$
      \EndIf
      \State Avanzar($it_2$) \Comment  $\Ogr(1)$
  \EndWhile
 
 \EndProcedure
 \end{algorithmic}
 \underline{Complejidad:} $\Ogr(n \times (L + log(k)))$\\
 \underline{Justificación:} $\Ogr(n)+\Ogr(n) \times \Ogr(log(k))+\Ogr(n)\times (\Ogr(L) + \Ogr(log(k)) = \Ogr(n) \times (\Ogr(L) + \Ogr(log(k)))=
 \Ogr(n \times (L + log(k)))$
\end{algorithm}



\begin{algorithm}
\caption{Red}
\begin{algorithmic}[1]
 \Procedure{red(\textbf{in} $s$ : dcnet)}{}
 \State res $\gets$ red \Comment $\Ogr(1)$
 \EndProcedure
   \underline{Complejidad:} $\Ogr(1)$
\end{algorithmic}
\end{algorithm}

\begin{algorithm}
\caption{Camino Recorrido}
\begin{algorithmic}[1]
  \Procedure{caminoRecorrido(\textbf{in} $s$ : dcnet, \textbf{in} $p$ : paquete) $\to res$ : secu(compu)}{}
   \State itDiccTrie $it$ $\gets$ CrearIt(paquetes) \Comment $\Ogr(1)$
   \While{$\neg$($p \in$ siguienteDefinicion(it).conjPaquete)} \Comment n veces $\Ogr(log \ k)$
   \State avanzar(it) \Comment $\Ogr(1)$
   \EndWhile
   \State res $\gets$ siguiente(Obtener(siguienteDefinicion(it).diccPaqCamino,p)) \Comment $\Ogr(log \ k)$
  \EndProcedure
   \underline{Complejidad:} $\Ogr(n.log \ k)$
 \underline{Justificación:} $\Ogr(1)+\Ogr(n.log \ k)+\Ogr(1)+\Ogr(log \ k) = \Ogr(n.log \ k)$
\end{algorithmic}
\end{algorithm}


\begin{algorithm}
\caption{Cantidad Enviados}
\begin{algorithmic}[1]
  \Procedure{cantidadEnviados(\textbf{in} $s$ : dcnet, \textbf{in} $c$ : compu)  $\to res$ : nat}{}
   \State res $\gets$ (Obtener(s.paquetes, c)).cantEnviados \Comment $\Ogr(L)$
  \EndProcedure
   \underline{Complejidad:} $\Ogr(L)$
\end{algorithmic}
\end{algorithm}


\begin{algorithm}
\caption{Paquete En Transito}
\begin{algorithmic}[1]
  \Procedure{paqueteEnTransito(\textbf{in} $s$ : dcnet, \textbf{in} $p$ : paquete)  $\to res$ : bool}{}
   \State itDiccTrie $it$ $\gets$ CrearIt(paquetes) \Comment $\Ogr(1)$
   \While{HaySiguiente?(it) $\yluego$ $\neg$($p \in$ siguienteDefinicion(it).conjPaquetes)} \Comment Se repite n veces $\Ogr(log \ k)$ 
   \State avanzar(it) \Comment $\Ogr(1)$
	\EndWhile  
	\If {HaySiguiente?(it)} \Comment $\Ogr(1)$
	\State res $\gets$   True \Comment $\Ogr(1)$
	\Else  
	\State res $\gets$ False \Comment $\Ogr(1)$
	\EndIf 
  \EndProcedure
    \underline{Complejidad:} $\Ogr(n.log \ k)$
 \underline{Justificación:} $\Ogr(1)+\Ogr(n.log \ k )+\Ogr(1)+\Ogr(1) = \Ogr(n.log \ k)$
\end{algorithmic}
\end{algorithm}




\begin{algorithm}
\caption{En Espera}
\begin{algorithmic}[1]
  \Procedure{enEspera(\textbf{in} $s$ : dcnet, \textbf{in} $c$ : compu)  $\to res$ : conj(compu)}{}
    \State res $\gets$ $(Obtener(s.paquetes,c)).conjPaquetes$ \Comment $\Ogr(1)+\Ogr(1)+\Ogr(L)+\Ogr(1)$
  \EndProcedure
  \underline{Complejidad:} $\Ogr(L)$
 \underline{Justificación:} $\Ogr(1)+\Ogr(1)+\Ogr(L)+\Ogr(1) = \Ogr(L)$

\end{algorithmic}
\end{algorithm}


\begin{algorithm}
\caption{La Que Más Envio}
\begin{algorithmic}[1]
  \Procedure{laQueMasEnvio(\textbf{in} $s$ : dcnet)  $\to res$ : compu}{}     
   \State res $\gets$   ($s$.laQueMasEnvio).cualCompu \Comment $\Ogr(1)$
  \EndProcedure
  \underline{Complejidad:} $\Ogr(1)$
\end{algorithmic}
\end{algorithm}



\end{Algoritmos}

