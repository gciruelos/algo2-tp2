\begin{Interfaz}
  \textbf{parámetros formales}\hangindent=2\parindent\\
  \parbox{1.7cm}{\textbf{géneros}} $\sigma$\\
  \parbox[t]{1.7cm}{\textbf{función}}\parbox[t]{\textwidth-2\parindent-1.7cm}{    	
    \InterfazFuncion{Copiar}{\In{s}{$\sigma$}}{$\sigma$}
    {$res \igobs s$}
    [$\Theta(copy(s))$]
    [función de copia de $\sigma$]
    }
      
  \textbf{se explica con}: \tadNombre{diccionario($string, \sigma$)}.

  \textbf{géneros}: \TipoVariable{diccTrie($\sigma$)}.

  \section*{Operaciones básicas DiccTrie($\sigma$)}

  \InterfazFuncion{Vacío}{}{diccTrie($\sigma$)}
  {$res$ \igobs vacío()}
  [$\Theta(1)$]
  [genera un diccionario vacío.]

  \InterfazFuncion{Definir}{\Inout{d}{diccTrie($\sigma$)}, \In{k}{$string$}, \In{s}{$\sigma$}}{}
  [$d \igobs d_0$]  
  {$d$ \igobs definir($k, s, d_0$)}
  [$\Theta(log(L))$, donde $L$ es el largo de $k$]
  [define la clave $k$ con el sinificado $s$.]
    
  \InterfazFuncion{Definido?}{\In{d}{diccTrie($\sigma$)}, \In{k}{$string$}}{bool}
  {$res$ \igobs def?($k$, $d$) }
  [$\Theta(log(L))$]
  [devuelve $true$ si la clave $k$ está definida en el diccionario.]
  
  \InterfazFuncion{Obtener}{\In{d}{diccTrie($\sigma$)}, \In{k}{$string$}}{$\sigma$}
  [def?($k$, $d$)]  
  {$res$ \igobs obtener($k$, $d$)}
  [$\Theta(log(L))$, donde $L$ es el largo de $k$]
  [devuelve el significado de la clave $k$.]
  
  \InterfazFuncion{Claves}{\In{d}{diccTrie($\sigma$)}}{conj(string)}
  {$res$ \igobs claves($d$)}
  [$\Theta()$]
  [devuelve un conjunto con todas las claves del diccionario.]  
  
  \section*{Operaciones del iterador}

  \InterfazFuncion{CrearIt}{\In{l}{lista($\alpha$)}}{itLista($\alpha$)}
  {$res$ $\igobs$ crearItBi(\secuencia{}, $l$) $\land$ alias(SecuSuby($it$) $=$ $l$)}
  [$\Theta(1)$]
  [crea un iterador bidireccional de la lista, de forma tal que al pedir \NombreFuncion{Siguiente} se obtenga el primer elemento de $l$.]
  [el iterador se invalida si y sólo si se elimina el elemento siguiente del iterador sin utilizar la función \NombreFuncion{EliminarSiguiente}.]
  
  \InterfazFuncion{CrearItUlt}{\In{l}{lista($\alpha$)}}{itLista($\alpha$)}
  {$res$ $\igobs$ crearItBi($l$, \secuencia{}) $\land$ alias(SecuSuby($it$) $=$ $l$)}
  [$\Theta(1)$]
  [crea un iterador bidireccional de la lista, de forma tal que al pedir \NombreFuncion{Anterior} se obtenga el último elemento de $l$.]  
  [el iterador se invalida si y sólo si se elimina el elemento siguiente del iterador sin utilizar la función \NombreFuncion{EliminarSiguiente}.]

\end{Interfaz}

\begin{Representacion}
  
  \section*{Representación de DiccTrie($\kappa, \sigma$)}

  \begin{Estructura}{lista$(\alpha)$}[lst]
    \begin{Tupla}[lst]
      \tupItem{primero}{puntero(nodo)}%
      \tupItem{longitud}{nat}%
    \end{Tupla}

    \begin{Tupla}[nodo]
      \tupItem{dato}{$\alpha$}%
      \tupItem{anterior}{puntero(nodo)}%
      \tupItem{siguiente}{puntero(nodo)}%
    \end{Tupla}
  \end{Estructura}

  \Rep[lst][l]{($l$.primero $=$ NULL) $=$ ($l$.longitud $=$ $0$) $\yluego$ ($l$.longitud $\neq$ $0$ \impluego \\
    Nodo($l$, $l$.longitud) $=$ $l$.primero $\land$ \\
    ($\forall i$: nat)(Nodo($l$,$i$)\DRef siguiente $=$ Nodo($l$,$i+1$)\DRef anterior) $\land$ \\
    ($\forall i$: nat)($1 \leq i <$ $l$.longitud $\implies$ Nodo($l$,$i$) $\neq$ $l$.primero)}\mbox{}

  ~      

  \tadOperacion{Nodo}{lst/l,nat}{puntero(nodo)}{$l$.primero $\neq$ NULL}
  \tadAxioma{Nodo($l$,$i$)}{\IF $i = 0$ THEN $l$.primero ELSE Nodo(FinLst($l$), $i-1$) FI}

  ~

  \tadOperacion{FinLst}{lst}{lst}{}
  \tadAxioma{FinLst($l$)}{Lst($l$.primero\DRef siguiente, $l$.longitud $-$ $\min$\{$l$.longitud, $1$\})}

  ~

  \tadOperacion{Lst}{puntero(nodo),nat}{lst}{}
  \tadAxioma{Lst($p,n$)}{$\langle p, n\rangle$}

  ~
 
  \AbsFc[lst]{secu($\alpha$)}[l]{\IF $l$.longitud $=$ $0$ THEN \secuencia{} ELSE \secuencia{$l$.primero\DRef dato}[Abs(FinLst($l$))] FI}

  \section*{Representación del iterador}

  \begin{Estructura}{itLista($\alpha$)}[iter]
    \begin{Tupla}[iter]
      \tupItem{siguiente}{puntero(nodo)}%
      \tupItem{lista}{puntero(lst)}%
    \end{Tupla}
  \end{Estructura}

  \Rep[iter][it]{Rep($\ast$($it$.lista)) $\yluego$ ($it$.siguiente $=$ NULL $\oluego$ ($\exists i$: nat)(Nodo($\ast it$.lista, $i$) $=$ $it$.siguiente)}

  ~

  \Abs[iter]{itBi($\alpha$)}[it]{b}{Siguientes($b$) $=$ Abs(Sig($it$.lista, $it$.siguiente)) $\land$\\
    Anteriores($b$) $=$ Abs(Ant($it$.lista, $it$.siguiente))}

  ~

  \tadOperacion{Sig}{puntero(lst)/l,puntero(nodo)/p}{lst}{Rep($\langle l, p\rangle$)}
  \tadAxioma{Sig($i, p$)}{Lst($p$, $l$\DRef longitud $-$ Pos($\ast l$, $p$))}

  ~

  \tadOperacion{Ant}{puntero(lst)/l,puntero(nodo)/p}{lst}{Rep($\langle l, p\rangle$)}
  \tadAxioma{Ant($i, p$)}{Lst(\IF $p$ $=$ $l$\DRef primero THEN NULL ELSE $l$\DRef primero FI, Pos($\ast l$, $p$))}

  ~

  {\small Nota: cuando $p$ $=$ NULL, Pos devuelve la longitud de la lista, lo cual está bien, porque significa que el iterador no tiene siguiente.}
  \tadOperacion{Pos}{lst/l,puntero(nodo)/p}{puntero(nodo)}{Rep($\langle l, p\rangle$)}
  \tadAxioma{Pos($l$,$p$)}{\IF $l$.primero $=$ $p$ $\lor$ $l$.longitud $=$ $0$ THEN $0$ ELSE $1$ $+$ Pos(FinLst($l$), $p$) FI}


\end{Representacion}