\begin{Interfaz}
  
  \textbf{se explica con}: \tadNombre{dcnet}.

  \textbf{géneros}: \TipoVariable{dcnet}.

  \section*{Operaciones básicas de DCNet}

  \InterfazFuncion{IniciarDCNet}{\In{r}{\tadNombre{red}}}{\tadNombre{dcnet}}
  [ $True$ ]
  {$ res \igobs IniciarDCNet(r)  $ }
  [$\Theta(1)$]
  [Genera una DCNet con las computadoras y conexiones de la red pasada como parámetro.]

  \InterfazFuncion{CrearPaquete}{\Inout{s}{\tadNombre{dcnet}}, \In{p}{\tadNombre{paquete}}}{}
  [$s \igobs s_0 \land 
   \neg((\exists p':paquete) (paqueteEnTransito?(s_0,p') \land id(p') \igobs id(p)) \land\\ 
   \hspace*{3em} origen(p) \in computadoras(red(s_0)) \yluego
   destino(p) \in computadoras(red(s_0)) \yluego\\
   \hspace*{3em} hayCamino?(red(s), origen(p), destino(p))$]
  {$s \igobs crearPaquete(s_0, p)$}
  [$\Theta()$]
  [Agrega el paquete $p$ a la cola de la computadora $p.origen$.]
  
  \InterfazFuncion{AvanzarSegundo}{\Inout{s}{\tadNombre{dcnet}}}{}
  [$s \igobs s_0$]  
  {$res \igobs avanzarSegundo(s_0)$}
  [$\Theta()$]
  [Avanza un segundo; todas las computadoras que tengan paquetes por enviar los envían.]
  
  
  \InterfazFuncion{red}{\In{s}{\tadNombre{dcnet}}}{\tadNombre{red}}
  [$ True $]  
  {$res \igobs red(s_0)$}
  [$\Theta(1)$]
  [Expresa la red que esta contenida en la DCNet.]

  
  \InterfazFuncion{caminoRecorrido}{\In{s}{\tadNombre{dcnet}}, \In{p}{\tadNombre{paquete}}}{\tadNombre{secu(compu)}}
  [$ paqueteEnTransito?(s,p) $]
  {$res \igobs caminoRecorrido(s, p)$}
  [$\Theta()$]
  [Expresa el camino de computadoras recorrido por el paquete $p$ desde su inicio hasta posicion actual.]


  \InterfazFuncion{cantidadEnviados}{\In{s}{\tadNombre{dcnet}}, \In{c}{\tadNombre{compu}}}{\tadNombre{nat}}
  [$ c \in computadoras(red(s)) $]  
  {$res \igobs cantidadEnviados(s, c)$}
  [$\Theta()$]
  [Expresa la cantidad de mensajes enviados por la compu $c$.]

  
  \InterfazFuncion{enEspera}{\In{s}{\tadNombre{dcnet}}, \In{c}{\tadNombre{compu}}}{\tadNombre{conj(paquete)}}
  [$ c \in computadoras(red(s)) $]  
  {$res \igobs enEspera(s, c)$}
  [$\Theta()$]
  [Expresa el camino de computadoras recorrido por el paquete $p$ desde su inicio hasta posicion actual.]


  \InterfazFuncion{PaqueteEnTransito}{\In{s}{\tadNombre{dcnet}}, \In{p}{\tadNombre{paquete}}}{\tadNombre{bool}}
  [$ True $]  
  {$res \igobs paqueteEnTransito?(s, p)$}
  [$\Theta()$]
  [Expresa si el paquete $p$ esta en alguna computadora.]
  
  
  \InterfazFuncion{LaQueMasEnvio}{\In{s}{\tadNombre{dcnet}}}{\tadNombre{compu}}
  [$ True $]
  {$res \igobs laQuemasEnvio(s)$}
  [$\Theta(1)$]
  [Devuelve la computadora que mas paquetes envio.]
  

\end{Interfaz}

\begin{Representacion}
  
  \section*{Representación de DCNet}

  \begin{Estructura}{dcnet}[net]
    \begin{Tupla}[net]
      \tupItem{proximaEnCamino}{dicc(compu, dicc(compu, compu))}%
      \tupItem{\\ paquetes}{diccTrie(compu,infoPaquetes)}%
      \tupItem{\\ caminosRecorridos}{lista(lista(compu))}
      \tupItem{\\ laQueMasEnvio}{tupla($cuantosEnvio$ : nat, $cualCompu$ : compu)}
      \tupItem{\\ red}{red}
    \end{Tupla}
  
  
  \begin{Tupla}[infoPaquetes]
      \tupItem{colas}{colaPrior($p$ : paq)}
      \tupItem{\\ diccPaqCamino}{diccAVL($p$ : paq, $camino$ : itLista )}
      \tupItem{\\ conjPaquetes}{conj(paq)}
      \tupItem{\\ cantEnviados}{nat}                       
    \end{Tupla}
  \end{Estructura}


  \Rep[net][n]{($\forall$ c : compu) ((c $\in$ claves(paquetes) $\rightarrow$ (($\forall$ p : paquete) p $\in$ definicion(paquetes, c).conjPaquetes $\rightarrow$ p $\in$ definicion(paquetes,c).colas $\land$ p $\in$ claves(diccPaqCamino) y tienen el mismo tamaño los tres (falta definir $\in$ para colas de prioridad \\ proximaEnCamino relacionado con red, para cualquier par de compus conectadas la proximaEnCamino es igual a al  primer elemento de alguna secuencia de caminosMinimos(n.red, c1, c2), si dos compus no estan conectadas no se define el camino | el numero de la que mas envio corresponde al de cantEnviados en infopaquete de cualCompu en laQueMasEnvio y no existe otra con una cantEnviados mayor | colaprior y conjPaquetes tienen los mismos elementos y diccpaqcamino tambien pero aparte tiene un iterador que te lleva a una secu de compus que contiene algun caminoMinimo que va desde el origan hasta la posicion actual del paquete |  }\mbox{}

 
  \AbsFc[net]{dcnet}[n]{dnt: dcnet | red(dnt) = n.red  $\land$ (($\forall$ p : paquete) (paqueteEnTransito?(dnt,p) $\Rightarrow$ (($\exists$ c : compu) p $\in$ $\pi_2$(definicion(n, c))) $\yluego$
 ($\forall$ p : paquete) ($\exists$ c : compu) c in computadoras(dnt) $\Rightarrow$ siguiente(definicion($\pi_2$(definicion(n.paquetes,c)),p) = caminorecorrido(c,p)  $\yluego$
($\forall$ c : compu) (c $\in$ computadoras(red(dcn)) $\Rightarrow$ (cantidadEnviados(dcn, c) = $\pi_3$(definicion(n.paquetes, c) $\yluego$ 
($\forall$ c : compu) c $\in$ computadoras(dnt) $\Rightarrow$ enEspera(dnt, c) = $\pi_2$(definicion(n.paquetes,c))}

\end{Representacion}



\begin{Algoritmos}


\begin{algorithm}
\caption{Iniciar DCNet}
\begin{algorithmic}[1]
  \Procedure{IniciarDCNet(\textbf{in} $r$ : red) $\to res$ : dcnet}{}
  \State $red \gets r$ \Comment $\Ogr(1)$
  \State $caminosRecorridos \gets$ Vacia() \Comment $\Ogr(1)$ 
  \State $laQueMasEnvio \gets$ (0, NULL) \Comment $\Ogr(1)$
  \State conj(compu) $compus \gets$ Computadoras($red$) \Comment $\Ogr(1)$
   \State itConj $it \gets$ CrearIt($compus$) \Comment $\Ogr(1)$
  \State $proximaEnCamino \gets$ Vacio() \Comment $\Ogr(1)$
  \While{HaySiguiente?(it)} \Comment $\Ogr(1)$
    \State \tadNombre{itConj} $it2 \gets$ CrearIt($compus$) \Comment $\Ogr(1)$
    \State diccTrie(compu, puntero(compu)) $diccActual \gets$ Vacio() \Comment $\Ogr(1)$
	\State definir(paquetes, siguiente(it), tupla(vacio(), vacio(), vacio(), 0)) \Comment $\Ogr(log \ n)$
    \While{HaySiguiente?(it2)} \Comment $\Ogr(n)$
      \State conj(lista(compu)) $camMinimos \gets$ CaminosMinimos(red, SiguienteClave(it), SiguienteClave(it2)) \Comment $\Ogr(1)$
      \State itConj $it3 \gets$ CrearIt(camMinimos) \Comment $\Ogr(1)$
      \State $caminoMinimo \gets$ siguiente(it3)
      \State Fin(caminoMinimo)
      \State puntero(compu) $siguiente \gets$ \& Primero($caminoMinimo$)
      \State Definir($diccActual$, SiguienteClave(it2), id($siguiente$))
      \State Avanzar(it2)            
    \EndWhile
    \State definir($proximaEnCamino$, id(SiguienteClave($it$), $diccActual$)))
    \State Avanzar($it$)
  \EndWhile
   
  \EndProcedure
\end{algorithmic}
\end{algorithm}


\begin{algorithm}
\caption{Crear Paquete}
\begin{algorithmic}[1]
  \Procedure{CrearPaquete(\textbf{in/out} $s$ : dcnet, \textbf{in} $p$ : paquete)}{}
   \If {p.origen != p.destino}
   \State lista(compu) $nuevoCaminoRecorrido \gets$ Vacio()
   \State AgregarAtras(nuevoCaminoRecorrido, origen(p))
   \State itLista(lista(compu)) $it \gets$ AgregarAtras(caminosRecorridos, nuevoCaminoRecorrido)
   \State $losPaquetes \gets $ Obtener($paquetes$, p.origen)
   \State Encolar(losPaquetes.cola, $p$)
   \State Agregar(losPaquetes.conjPaquetes, $p$)
   \State Definir(diccPaqCamino, $p$, it)
   \EndIf
   \EndProcedure
\end{algorithmic}
\end{algorithm}


\begin{algorithm}
\caption{Avanzar Segundo}
\begin{algorithmic}[1]
  \Procedure{AvanzarSegundo(\textbf{in/out} $s$ : dcnet)}{}
  
  \State lista(tupla(paq, itLista(lista(id)))) $paquetesAEnviar \gets$ Vacio()
  
  \State itDiccTrie $it \gets$ CrearIt($paquetes$)
  \While{HaySiguientes?(it)} \Comment $\Ogr(n)$
  	\State infoCompu $s \gets$ SiguienteSignificado($it$) 
  	\If{$\neg$Vacía?($s$.$cola$)}
  		\State paquete $estePaquete \gets$ Desencolar($s$.$cola$)
  		\State Eliminar($s$.conjPaquetes, $estepaquete$)
		\State AgregarAtras(paquetesAEnviar(tupla($estePaquete$, definicion(s.diccPaqCamino, $estePaquete$))))
		\State Eliminar(s.diccPaqCamino, $estePaquete$) 
		\State cantEnviados++
		\If{cantEnviados > $s$.laQueMasEnvio.cuantosEnvio}
		   \State $laQueMasEnvio.cuantosEnvio \gets$ $s$.cantEnviados
		   \State $laQueMasEnvio.cualCompu \gets$ ultimo(Siguiente($pi_1$($estePaquete$)))
		\EndIf
	\EndIf
	\State Avanzar($it$)
  \EndWhile 
   
  \State itLista(tupla(paq, itLista(lista(compu)))) $it_2 \gets$ CrearIt($paquetesAEnviar$)
  \While{HaySiguientes?($it_2$)}
      \State tupla(paq, itLista(lista(compu)) $p \gets$ Siguiente($it_2$)
      \State compu $proximaCompu \gets$ Obtener(Obtener($proximaEnCamino$, $Ultimo(Siguiente(\pi_1(p)))$), destino($p$))
      \If {proximaCompu != $\pi_0(p)$}
	\State $paquetesDeProximaCompu \gets$ Obtener($paquetes$, proximaCompu)
      	\State AgregarAtras(Siguiente($\pi_1(p)$), proximaCompu)
      	\State Encolar($paquetesDeProximaCompu$.$cola$, $\pi_0(p)$)
      	\State Agregar($paquetesDeProximacompu$.$conjPaquetes$, $\pi_0$(p))
	\State Definir($paquetesDeProximacompu.diccPaqCamino$,$\pi_0$(p),$\pi_1$(p))      
      \EndIf
      \State Avanzar($it_2$)
  \EndWhile
 
 \EndProcedure
\end{algorithmic}
\end{algorithm}



\begin{algorithm}
\caption{Red}
\begin{algorithmic}[1]
 \Procedure{red(\textbf{in} $s$ : dcnet)}{}
 \State res $\gets$ red
 \EndProcedure
\end{algorithmic}
\end{algorithm}

\begin{algorithm}
\caption{Camino Recorrido}
\begin{algorithmic}[1]
  \Procedure{caminoRecorrido(\textbf{in} $s$ : dcnet, \textbf{in} $p$ : paquete) $\to res$ : secu(compu)}{}
   \State itDiccTrie $it$ $\gets$ CrearIt(paquetes)
   \While{$\neg$($p \in$ enEspera(siguiente(it)))} 
   \State avanzar(it)
   \EndWhile
   \State res $\gets$ siguiente(Obtener($\pi_1$(Obtener(paquetes,siguiente(it)),p))
  \EndProcedure
\end{algorithmic}
\end{algorithm}


\begin{algorithm}
\caption{Cantidad Enviados}
\begin{algorithmic}[1]
  \Procedure{cantidadEnviados(\textbf{in} $s$ : dcnet, \textbf{in} $c$ : compu)  $\to res$ : nat}{}
   \State res $\gets$ $\pi_3$(Obtener(paquetes, c))
  \EndProcedure
\end{algorithmic}
\end{algorithm}


\begin{algorithm}
\caption{Paquete En Transito}
\begin{algorithmic}[1]
  \Procedure{paqueteEnTransito(\textbf{in} $s$ : dcnet, \textbf{in} $p$ : paquete)  $\to res$ : bool}{}
   \State itDiccTrie $it$ $\gets$ CrearIt(paquetes)
   \While{HaySiguiente?(it) $\yluego$ $\neg$($p \in$ enEspera(siguiente(it)))} 
   \State avanzar(it)
	\EndWhile
	\If {HaySiguiente?(it)}
	\State res $\gets$   True
	\Else 
	\State res $\gets$ False
	\EndIf 
  \EndProcedure
\end{algorithmic}
\end{algorithm}




\begin{algorithm}
\caption{En Espera}
\begin{algorithmic}[1]
  \Procedure{enEspera(\textbf{in} $s$ : dcnet, \textbf{in} $c$ : compu)  $\to res$ : conj(compu)}{}
    \State res $\gets$ $\pi_2(Obtener(\pi_1(net),c))$ \Comment $\Ogr(1)+\Ogr(1)+\Ogr(L)+\Ogr(1)$
  \EndProcedure
  \underline{Complejidad:} $\Ogr(L)$
 \underline{Justificación:} $\Ogr(1)+\Ogr(1)+\Ogr(L)+\Ogr(1) = \Ogr(L)$

\end{algorithmic}
\end{algorithm}


\begin{algorithm}
\caption{La Que Más Envio}
\begin{algorithmic}[1]
  \Procedure{laQueMasEnvio(\textbf{in} $s$ : dcnet)  $\to res$ : compu}{}     
   \State res $\gets$   $\pi_1(\pi_3(net))$ \Comment $\Ogr(1)$
  \EndProcedure
  \underline{Complejidad:} $\Ogr(1)$
\end{algorithmic}
\end{algorithm}



\end{Algoritmos}

